\section{Interfaces}\label{sec:PythonInterfaces}
In general, this section defines various combinations of default attributes and
the roles/interfaces that use them. Mostly, this information directly
maps to the definitions in the C API specification. However, there are some
methods described in this section that are discussed because of the
differences between their C and Python implementations.

\subsection{Operating System, Hardware Interface}
\label{sec:PythonOperatingSystemInterface}

These methods are related to \texttt{pwr.Obj} objects of type
\texttt{pwr.ObjType.NODE}, so they are encapsulated in the \texttt{pwr.Obj}
class.

%==============================================================================%
\subsubsection{Method StateTransitDelay} \label{meth:StateTransitDelay}

This method returns the transition delay (\texttt{pwr.Time}) given the callers startState
and endState input parameters. See section \ref{class:OperState} on page
\pageref{class:OperState} for details on the \texttt{pwr.OperState} class): 

\begin{center}\begin{minipage}{.95\linewidth}\begin{lstlisting}
startState = myPwrObj.GetPerfState()
startState = myPwrObj.perfstate         # Shortcut
endState = pwr.OperState(pwr.SleepState.SHALLOW, pwr.PerfState.FASTEST)
latencyPowerTime = myPwrObj.StateTransitDelay(startState, endState)
#
# Where:
#   startState is a pwr.OperState state 
#   endState   is a pwr.OperState state
# Returns:
#   latencyPowerTime is the pwr.Time latency
# This method raises a pwr.PwrError exception when something goes wrong.
# The possible exception errors are: 
#   pwr.ReturnCode.FAILURE
\end{lstlisting}\end{minipage}\end{center}

Note that the example above uses the method \texttt{GetPerfState} documented on page
\pageref{meth:GetPerfState}.

\subsection{Monitor and Control, Hardware Interface}\label{sec:PythonMonitorControlInterface}
(No new methods described here)

\subsection{Application, Operating System Interface}\label{sec:PythonApplicationInterface}

%==============================================================================%
\subsubsection{Method AppTuningHint} \label{meth:AppTuningHint}

This method supplies power hints to a power object, using the hints enumerated by the
\texttt{pwr.RegionHint} and \texttt{pwr.RegionIntensity} enumerations.

\begin{center}\begin{minipage}{.95\linewidth}\begin{lstlisting}
pwrRegionHint = pwr.RegionHint.MEM_BOUND      # see enum RegionHint
pwrRegionIntensity = pwr.RegionIntensity.LOW  # see enum RegionIntensity
myPwrObj.AppTuningHint(pwrRegionHint, pwrRegionIntensity)
#
# Where:
#   pwrRegionHint is a pwr.RegionHint type
#   pwrRegionIntensity is a pwr.RegionIntensity level
# This method raises a pwr.PwrError exception when something goes wrong.
# The possible exception errors are: 
#   pwr.ReturnCode.FAILURE
#   pwr.ReturnCode.NOT_IMPLEMENTED
\end{lstlisting}\end{minipage}\end{center}

%==============================================================================%
\subsubsection{Method SetSleepStateLimit} \label{meth:SetSleepStateLimit}

This method sets the sleep-state limit from the enumeration for a power object.

\begin{center}\begin{minipage}{.95\linewidth}\begin{lstlisting}
pwrSleepState = pwr.SleepState.NO  # see enum SleepState
myPwrObj.SetSleepStateLimit(pwrSleepState)
myPwrObj.sleepstate = pwrSleepState         # Shortcut
#
# Where:
#   pwrSleepState is a pwr.SleepState type
# This method raises a pwr.PwrError exception when something goes wrong.
# The possible exception errors are: 
#   pwr.ReturnCode.FAILURE
#   pwr.ReturnCode.NOT_IMPLEMENTED
\end{lstlisting}\end{minipage}\end{center}

%==============================================================================%
\subsubsection{Method WakeUpLatency} \label{meth:WakeUpLatency}

This method gets the wake up latency for the sleep-state (\texttt{DEEPEST} in this example) to
transition from, returning the (\texttt{pwr.Time}) latency of the transition.

\begin{center}\begin{minipage}{.95\linewidth}\begin{lstlisting}
latencyPwrTime = myPwrObj.WakeUpLatency(pwr.SleepState.DEEPEST)
#
# Where:
#   pwrSleepState is a pwr.SleepState type
# Returns:
#   latencyPwrTime a pwr.Time object representing the latency time
# This method raises a pwr.PwrError exception when something goes wrong.
# The possible exception errors are: 
#   pwr.ReturnCode.FAILURE
#   pwr.ReturnCode.NOT_IMPLEMENTED
\end{lstlisting}\end{minipage}\end{center}

%==============================================================================%
\subsubsection{Method RecommendSleepState} \label{meth:RecommendSleepState}

This method recommends a sleep-state for a power object, returning the deepest sleep-state
(\texttt{pwr.SleepState}) to be used as a limit.

\begin{center}\begin{minipage}{.95\linewidth}\begin{lstlisting}
pwrSleepState = myPwrObj.RecommendSleepState(latencyPwrTime) 
#
# Where:
#   latencyPwrTime is a pwr.Time latency value
# Returns:
#   pwrSleepState is a pwr.SleepState recommendation type
# This method raises a pwr.PwrError exception when something goes wrong.
# The possible exception errors are: 
#   pwr.ReturnCode.FAILURE
#   pwr.ReturnCode.NOT_IMPLEMENTED
\end{lstlisting}\end{minipage}\end{center}

%==============================================================================%
\subsubsection{Method SetPerfState} \label{meth:SetPerfState}

This method requests that a power object performance level be set to a desired (\texttt{FASTEST}
in this example) \texttt{pwr.PerfState} level.

\begin{center}\begin{minipage}{.95\linewidth}\begin{lstlisting}
myPwrObj.SetPerfState(pwr.PerfState.FASTEST)
myPwrObj.perfstate = pwr.PerfState.FASTEST     # Shortcut
#
# Where:
#   pwrPerfState is the requested pwr.PerfState type
# This method raises a pwr.PwrError exception when something goes wrong.
# The possible exception errors are: 
#   pwr.ReturnCode.FAILURE
#   pwr.ReturnCode.NOT_IMPLEMENTED
\end{lstlisting}\end{minipage}\end{center}

%==============================================================================%
\subsubsection{Method GetPerfState} \label{meth:GetPerfState}

This method retrieves the performance level of a power object, returning the current
performance state, \texttt{pwr.PerfState} object.

\begin{center}\begin{minipage}{.95\linewidth}\begin{lstlisting}
pwrPerfState = myPwrObj.GetPerfState()
pwrPerfState = myPwrObj.perfstate       # Shortcut
#
# Returns:
#   pwrPerfState is the returned pwr.PerfState type
# This method raises a pwr.PwrError exception when something goes wrong.
# The possible exception errors are: 
#   pwr.ReturnCode.FAILURE
#   pwr.ReturnCode.NOT_IMPLEMENTED
\end{lstlisting}\end{minipage}\end{center}

%==============================================================================%
\subsubsection{Method GetSleepState} \label{meth:GetSleepState}

Similarly for retrieving the sleep-state of a power object, the \texttt{GetSleepState} method may be used.
\begin{center}\begin{minipage}{.95\linewidth}\begin{lstlisting}
pwrSleepState = myPwrObj.GetSleepState()
pwrSleepState = myPwrObj.sleepstate       # Shortcut
#
# Returns:
#   pwrSleepState is the returned pwr.SleepState type
# This method raises a pwr.PwrError exception when something goes wrong.
# The possible exception errors are: 
#   pwr.ReturnCode.FAILURE
#   pwr.ReturnCode.NOT_IMPLEMENTED
\end{lstlisting}\end{minipage}\end{center}

\subsection{User, Resource Manager
Interface}\label{sec:PythonUserResourceManagerInterface}
(No new methods described here)

\subsection{Resource Manager, Operating System
Interface}\label{sec:PythonResourceManagerOSInterface}
(No new methods described here)

\subsection{Resource Manager, Monitor and Control
Interface}\label{sec:PythonResourceManagerMonitorControlInterface}
(No new methods described here)

\subsection{Administrator, Monitor and Control
Interface}\label{sec:PythonAdministratorMonitorControlInterface}
(No new methods described here)

\subsection{HPCS Manager, Resource Manager
Interface}\label{sec:PythonHPCSManagerInterface}
(No new methods described here)

\subsection{Accounting, Monitor and Control
Interface}\label{sec:PythonAccountingMonitorControlInterface}
(No new methods described here)

\subsection{User Monitor and Control
Interface}\label{sec:PythonUserMonitorControlInterface}
(No new methods described here)

\section{Conclusion}\label{sec:PythonConclusion}
This concludes the Python-specific appendix to the Power API specification.

