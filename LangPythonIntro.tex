\lstset{language=Python,keepspaces=true}
Acceptance of the Python language is growing within the HPCS System and Power
Management communities for its ease of scripting and its interoperability with
existing other applications and frameworks.

Python differs from C in many ways, including that it is an
object-oriented language and implements garbage collection.  Because of these differences, Python
implementations of the API have some fundamental differences with the way they
are used. The functionality is the same as the C API specification, but the
``style'' with which it is used is different in several respects, as described
in the remainder of this appendix.

\section{Introduction}\label{sec:PythonIntroduction}
The general structure of the C API specification is followed throughout this
appendix. Differences between the Python language bindings and the C API specification
are clearly noted. The subsections of this Python specification mirror the C
API specification and are similarly labeled, e.g. section
\ref{sec:PythonOpaqueTypes} corresponds to \ref{sec:OpaqueTypes} in the C API.

%==============================================================================%
\subsubsection{Python PEP-8 Standard Compliance}
\label{sec:PythonPEP8StandardCompliance:GrpCreate}

In general this Python API specification follows the PEP8 guidelines
regarding coding standards and naming conventions. These guidelines are
followed unless they create undue differences with how the C API specification
defines the names of various entities. An example of this is the C function:
\texttt{PWR_CntxtGetEntryPoint()}. In this Python Specification,
\texttt{GetEntryPoint()} is an instance method of the \texttt{pwr.Cntxt} class.
PEP8 suggests method and function names start with a lower case letter, e.g.
\texttt{getEntryPoint()}; but for consistency with the C API, this Python
specification uses \texttt{GetEntryPoint()}. A brief summary of the
PEP8 naming conventions follows:

\begin{itemize}[noitemsep,nolistsep]
\item{Packages and Modules have short names with all lowercase.}

\item{Classes are named with ``CapWords'' names, also known as
``CamelCase''. The first letter is capitalized.}

\item{Exceptions follow the same rules as classes with the additional
rule that ``Error'' be appended, such as ``CamelCaseError''.}

\item{Global Variables, Method and Function names, and Instance Variables
are named with ``lower_case_with_underscores'' or ``mixedCase'' (with the
first character being lower case).}

\item{Indentation and spacing are flexible within Python. See the PEP8 standard
for an at-length discussion. C function parameters are aligned with the opening
delimiter so Python methods and functions and their invocations are indented
that way whenever possible.}

\item{Non-Public or external variables are named with a
``_leadingUnderscore''. Virtual methods for which a child class method is
required may or may not have a leading underscore depending on whether that
method or its derivatives is meant to be publicly exposed. This document only
defines the public methods available via the ``opaque'' API}

\item{Constants are named with ``ALL_CAPS''.}

\item{Keyword collisions are handled by appending an underscore to the
colliding keyword, such as ``open_''.}

\item{Always use ``self'' as the first argument for instance methods, and use
``cls'' as the first argument for class methods.}

\end{itemize}

\emph{Implementation Note: Throughout this appendix, module function and class
method names begin with an upper-case character.}


