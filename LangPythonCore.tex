\section{Core (Common) Interface Methods}
\label{sec:PythonCoreInterfaceFunctions}

Core Interface Methods fall into the following categories:
\begin{itemize}[noitemsep,nolistsep]
\item{\textbf{Initialization}}
\item{\textbf{Navigation}}
\item{\textbf{Group}}
\item{\textbf{Attribute}}
\item{\textbf{Metadata}}
\item{\textbf{Statistics}}
\end{itemize}

For background information on the methods described in this section, please refer to
the C API function descriptions starting at page \pageref{chap:Common}. Many of
the Interface methods defined are implemented as class instance constructor
methods in Python versions of this API. Because of the garbage collection
capabilities of Python, some of the ``Destroy'' methods are not needed. Those
differences are noted in the following sections. If an error occurs
instantiating an object, a \texttt{pwr.PwrError} exception is raised.

\subsection{Initialization} \label{sec:PythonInitialization}

%==============================================================================%
\subsubsection{Method Cntxt} \label{meth:Cntxt}

A context is an instance of the \texttt{pwr.Cntxt} class:
\begin{center}\begin{minipage}{.95\linewidth}\begin{lstlisting}

class Cntxt():
    def __init__(self, cntxtType, cntxtRole, cntxtName):
        ...
    def GetEntryPoint():
        ...
\end{lstlisting}\end{minipage}\end{center}

Note: the \texttt{try/except} clause has been added for example purposes, but
is not included in all the code examples throughout this document.  See general
discussion about Python Error Handling in section \ref{sec:PythonErrorHandling}
on page \pageref{sec:PythonErrorHandling}.

To instantiate a default power context for a user role:

\begin{center}\begin{minipage}{.95\linewidth}\begin{lstlisting}
try:
    myPwrCntxt = pwr.Cntxt(pwr.CntxtType.DEFAULT, pwr.Role.RM, "Default")
except pwr.PwrError as e:
    print str(e.errno)
    print e.errmsg
    print e.strerror
#
# Where:
#   cntxtType is a pwr.CntxtType
#   pwrRole is a pwr.Role type
#   cntxtName is a Python str
# Returns:
#   myPwrCntxt is a pwr.Cntxt context
# This method raises a pwr.PwrError exception when something goes wrong.
# The possible exception errors are:
#   pwr.ReturnCode.FAILURE
\end{lstlisting}\end{minipage}\end{center}

%==============================================================================%
\subsubsection{Method CntxtDestroy NOT_IMPLEMENTED} \label{meth:CntxtDestroy}

Because Python implements garbage collection, there is no need to de-initialize or
destroy a context, and a \texttt{CntxtDestroy} method need not be implemented.

\subsection{Hierarchy Navigation Methods}
\label{sec:PythonHierarchyNavigationMethods}

%==============================================================================%
\subsubsection{Method GetEntryPoint} \label{meth:GetEntryPoint}

Once a context has been established, the entry point in the object tree can
be queried. Each context has its own entry point. Calling the
\texttt{GetEntryPoint} method on the users context returns the context specific
entry point.

\begin{center}\begin{minipage}{.95\linewidth}\begin{lstlisting}
myPwrObj = myPwrCntxt.GetEntryPoint()
myPwrObj = myPwrCntxt.entrypoint       # Shortcut
#
# Returns:
#   pwr.Obj object or None
# This method raises a pwr.PwrError exception when something goes wrong.
# The possible exception errors are:
#   pwr.ReturnCode.FAILURE
\end{lstlisting}\end{minipage}\end{center}

Once the entry point object is obtained, it can be queried to get its Type,
Name, Parent, and Children. These query methods return either an object or a
\texttt{pwr.Grp} (in the case of \texttt{GetChildren()}), or either \texttt{None}
or an empty \texttt{pwr.Grp} if the object(s) are non-existent. Not having a
parent or any children is not considered an error condition. Note that the
Python handling of non-existent parents and children is different than how
these conditions are handled in the C API, where a non-zero int is returned. In
Python returning an empty group or \texttt{None} enables code to handle
hierarchy navigation more naturally then if an exception was to be raised. The
\texttt{GetChildren()} method has a generator method, \texttt{GenerateChildren()}.

%==============================================================================%
\subsubsection{Method GetType} \label{meth:GetType}

This method returns the \texttt{pwr.ObjType} of an object.

\begin{center}\begin{minipage}{.95\linewidth}\begin{lstlisting}
objType = myPwrObj.GetType()
objType = myPwrObj.objType    # Shortcut
#
# Returns:
#   pwr.ObjType or pwr.ObjType.INVALID upon failure
# This method raises a pwr.PwrError exception when something goes wrong.
# The possible exception errors are:
#   pwr.ReturnCode.FAILURE
\end{lstlisting}\end{minipage}\end{center}

%==============================================================================%
\subsubsection{Method GetName} \label{meth:GetName}

This method returns the name of an object.

\begin{center}\begin{minipage}{.95\linewidth}\begin{lstlisting}
objName = myPwrObj.GetName()
objName = myPwrObj.name        # Shortcut
#
# Returns:
#   String containing myPwrObj's name
# This method raises a pwr.PwrError exception when something goes wrong,
#   such as if myPwrObj does not actually represent a pwr.Obj instance.
# The possible exception errors are:
#   pwr.ReturnCode.FAILURE
\end{lstlisting}\end{minipage}\end{center}

%==============================================================================%
\subsubsection{Method GetParent} \label{meth:GetParent}

Note that unlike the corresponding C API function on page
\pageref{func:ObjGetParent} the Python \texttt{GetParent} Method returns \texttt{None}
when the base object has no parent. This allows for handling this condition in
Python without needing a \texttt{try/except} block.

\begin{center}\begin{minipage}{.95\linewidth}\begin{lstlisting}
objParent = myPwrObj.GetParent()
objParent = myPwrObj.parent       # Shortcut
#
# Returns:
#   pwr.Obj type or None if there is no parent.
# This method raises a pwr.PwrError exception when something goes wrong.
# The possible exception errors are:
#   pwr.ReturnCode.FAILURE
\end{lstlisting}\end{minipage}\end{center}

%==============================================================================%
\subsubsection{Method GetChildren and GenerateChildren} \label{meth:GetChildren}

Note that unlike the corresponding C API function on page
\pageref{func:ObjGetChildren}, the Python \texttt{GetChildren} method returns an empty
group when the base object has no children. This allows for handling this
condition in Python without needing a \texttt{try/except} block.

\begin{center}\begin{minipage}{.95\linewidth}\begin{lstlisting}
# Return a pwr.Grp of the children:
objChildrenGrp = myPwrObj.GetChildren()  # Returns a PwrGrp Group.
objChildrenGrp = myPwrObj.children       # Shortcut
# Generator of pwr.Obj children. Yields pwr.Obj members of the group.
for childPwrObj in myPwrObj.GenerateChildren():
    # Iterate on childPwrObj...
#
# Returns:
#   objChildrenGrp is a pwr.Grp containing pwr.Obj
#    type objects of children. An empty pwr.Grp may be returned
#    when there are no children.
# This method raises a pwr.PwrError exception when something goes wrong.
# The possible exception errors are:
#   pwr.ReturnCode.FAILURE
\end{lstlisting}\end{minipage}\end{center}

%==============================================================================%
\subsubsection{Method GetObjByName} \label{meth:GetObjByName}

A \texttt{pwr.Obj} object can be obtained using its name. Because the
the naming system used for this method may be vendor-specific, this method is
necessarily vendor implementation-specific and should not be considered
generally portable. Vendor-specific details should be documented by the API
implementor/vendor.

\begin{center}\begin{minipage}{.95\linewidth}\begin{lstlisting}
namedPwrObj = myPwrCntxt.GetObjByName(objName)
#
# Where:
#   objName is a Python string containing the power object's name
# Returns:
#   namedPowerObj is a pwr.Obj or None upon failure
# This method raises a pwr.PwrError exception when something goes wrong.
# The possible exception errors are:
#   pwr.ReturnCode.FAILURE
#   pwr.ReturnCode.NOT_IMPLEMENTED
\end{lstlisting}\end{minipage}\end{center}

\emph{Implementation Note: Object names are vendor-implementation-dependent and
are not defined in this API. If the name of an object or group is not supported,
a \texttt{pwr.PwrError} error code with \texttt{pwr.ReturnCode.NOT_IMPLEMENTED}
is returned.}

\subsection{Group Methods} \label{sec:PythonGroupMethods}

All Power API groups are associated with a context, therefore the group
creation and retrieval methods are encapsulated as \texttt{pwr.Cntxt} class
methods. See page \pageref{sec:Group} for the C API's full text description of
Group operations.

%==============================================================================%
\subsubsection{Method GrpCreate} \label{meth:GrpCreate}

If a \texttt{pwr.PwrError} does not get raised during creation of this group,
an empty \texttt{pwr.Grp} group is returned. No specific ``Destroy'' method is
needed for any \texttt{pwr.Grp} groups. Python's garbage collection handles the
clean up of \texttt{pwr.Grp} groups that are no longer referenced.

\begin{center}\begin{minipage}{.95\linewidth}\begin{lstlisting}
myPwrGrp = myPwrCntxt.GrpCreate()
#
# This method raises a pwr.PwrError exception when something goes wrong.
# The possible exception errors are:
#   pwr.ReturnCode.FAILURE
\end{lstlisting}\end{minipage}\end{center}

A Power API Group is an encapsulated Python list of pwr.Obj objects.
This encapsulation offers strict type-checking over that of standard Python
lists, but gives inheritance of all the power of Python lists to the
\texttt{pwr.Grp}:

\begin{center}\begin{minipage}{.95\linewidth}\begin{lstlisting}
myGroup = myPwrCntxt.GrpCreate()
someOtherList = [1,2,3]
print isinstance(myGroup, list)            # Prints: "True"
print isinstance(myGroup, pwr.Grp)         # Prints: "True"
print isinstance(someOtherList, list)      # Prints: "True"
print isinstance(someOtherList, pwr.Grp)   # Prints: "False"
\end{lstlisting}\end{minipage}\end{center}

%==============================================================================%
\subsubsection{Method iter(Grp)} \label{meth:GenerateGroupObjs}

The following standard Python function generates an iterator over the objects
in a \texttt{pwr.Grp}. See section \ref{sec:PythonIteratorsGenerators}
on page \pageref{sec:PythonIteratorsGenerators} for more background on
``generators'':

\begin{center}\begin{minipage}{.95\linewidth}\begin{lstlisting}
for pwrObj in  iter(myPwrGrp):
    # Iterate on pwrObj...
#
# This method raises a pwr.PwrError exception when something goes wrong.
# The possible exception errors are:
#   pwr.ReturnCode.FAILURE
\end{lstlisting}\end{minipage}\end{center}

%==============================================================================%
\subsubsection{Method AddObj} \label{meth:AddObj}

This method adds a \texttt{pwr.Obj} to a group. As noted in the C API
description on page \pageref{func:GrpAddObj} attempting to add an object that
is already in a group is not allowed and will result in no insertion.  The
following shows examples of adding a \texttt{pwr.Obj} to a group.

\begin{center}\begin{minipage}{.95\linewidth}\begin{lstlisting}
myPwrGrp.AddObj(pwrObj)
myPwrGrp = myPwrGrp + pwrObj         # Shortcut
myPwrGrp = myPwrGrp + [pwrObj, ...]  # Shortcut
myPwrGrp += pwrObj                   # Shortcut
myPwrGrp += [pwrObj, ...]            # Shortcut
#
# Where:
#   pwrObj is a pwr.Obj object
# This method raises a pwr.PwrError exception when something goes wrong.
# The possible exception errors are:
#   pwr.ReturnCode.FAILURE
\end{lstlisting}\end{minipage}\end{center}

%==============================================================================%
\subsubsection{Method RemoveObj} \label{meth:RemoveObj}

This removes a \texttt{pwr.Obj} from the group.

\begin{center}\begin{minipage}{.95\linewidth}\begin{lstlisting}
myPwrGrp.RemoveObj(pwrObj)
myPwrGrp = myPwrGrp - pwrObj         # Shortcut
myPwrGrp = myPwrGrp - [pwrObj, ...]  # Shortcut
myPwrGrp -= pwrObj                   # Shortcut
myPwrGrp -= [pwrObj, ...]            # Shortcut
#
# Where:
#   pwrObj is a pwr.Obj object
# This method raises a pwr.PwrError exception when something goes wrong.
# The possible exception errors are:
#   pwr.ReturnCode.FAILURE
\end{lstlisting}\end{minipage}\end{center}

%==============================================================================%
\subsubsection{Method GetNumObjs} \label{meth:GetNumObjs}

The following returns the number of objects in a group:

\begin{center}\begin{minipage}{.95\linewidth}\begin{lstlisting}
myPwrGrpNumObjs = myPwrGrp.GetNumObjs()
myPwrGrpNumObjs = len(myPwrGrp)          # Shortcut
#
# Returns:
#   myPwrGrpNumObjs is an integer
# This method raises a pwr.PwrError exception when something goes wrong.
# The possible exception errors are:
#   pwr.ReturnCode.FAILURE
\end{lstlisting}\end{minipage}\end{center}

%==============================================================================%
\subsubsection{Method GetObjByIndx NOT_IMPLEMENTED} \label{meth:GetObjByIndx}

In Python API implementations, there is no need for a Power API method to index
a group's objects. Python's built-in list class, which forms the foundation of
a \texttt{pwr.Grp} has all the necessary indexing and iteration methods needed.
The C API's \texttt{PWR_GrpGetObjByIndx()} function, is documented in
\ref{func:GrpGetObjByIndx} on page \pageref{func:GrpGetObjByIndx}.

\begin{center}\begin{minipage}{.95\linewidth}\begin{lstlisting}
# Python's built-in iterator
for pwrObj in iter(myPwrGrp):
    print pwrObj.GetName()

# Trick:  To index an item in a group:
pwrObj3 = list(iter(myPwrGrp))[3]
\end{lstlisting}\end{minipage}\end{center}


%==============================================================================%
\subsubsection{Method Duplicate} \label{meth:Duplicate}

The following duplicates the \texttt{myPwrGrp} group creating the new
\texttt{duplicateGrp}:

\begin{center}\begin{minipage}{.95\linewidth}\begin{lstlisting}
duplicateGrp = myPwrGrp.Duplicate()
duplicateGrp = pwr.Grp(myPwrGrp)     # Shortcut: copy constructor.
#
# This method raises a pwr.PwrError exception when something goes wrong.
# The possible exception errors are:
#   pwr.ReturnCode.FAILURE
\end{lstlisting}\end{minipage}\end{center}

%==============================================================================%
\subsubsection{Method Union and GenerateUnion} \label{meth:Union}

The following example creates a new group \texttt{unionGrp} containing all the
objects that exist in either or both of the \texttt{myPwrGroup} and the
\texttt{someOtherPwrGrp} group. The associated \texttt{GenerateUnion()} method is also shown:

\begin{center}\begin{minipage}{.95\linewidth}\begin{lstlisting}
unionGrp = myPwrGrp.Union(someOtherPwrGrp)
unionGrp = myPwrGrp | someOtherPwrGrp       # Shortcut
unionGrp |= someOtherPwrGrp                 # Shortcut
# Generator of pwr.Objs:
for pwrObj in myPwrGrp.GenerateUnion(someOtherPwrGrp):
    # Iterate on pwrObj...
#
# Where:
#   someOtherPwrGroup is a pwr.Grp object to merge with
# This method raises a pwr.PwrError exception when something goes wrong.
# The possible exception errors are:
#   pwr.ReturnCode.FAILURE
\end{lstlisting}\end{minipage}\end{center}

%==============================================================================%
\subsubsection{Method Intersection and GenerateIntersection}
\label{meth:Intersection}

The following creates a new group containing only objects that exist in both the
\texttt{myPwrGroup} and \texttt{someOtherPwrGrp} groups. The associated
\texttt{GenerateIntersection()} method is also shown:

\begin{center}\begin{minipage}{.95\linewidth}\begin{lstlisting}
intersectionGrp = myPwrGrp.Intersection(someOtherPwrGrp)
intersectionGrp = myPwrGrp & someOtherPwrGrp               # Shortcut
intersectionGrp &= someOtherPwrGrp                         # Shortcut
# Generator of pwr.Objs:
for pwrObj in myPwrGrp.GenerateIntersection(someOtherPwrGrp):
    # Iterate on pwrObj...
#
# Where:
#   someOtherPwrGroup is a pwr.Grp object to merge with
# This method raises a pwr.PwrError exception when something goes wrong.
# The possible exception errors are:
#   pwr.ReturnCode.FAILURE
\end{lstlisting}\end{minipage}\end{center}

%==============================================================================%
\subsubsection{Method Difference and GenerateDifference} \label{meth:Difference}

The following creates a new group containing all the objects of the current
group or another but do not exist in both groups. The associated
\texttt{GenerateDifference()} method is also shown:

\begin{center}\begin{minipage}{.95\linewidth}\begin{lstlisting}
differenceGrp = myPwrGrp.Difference(someOtherPwrGrp)
differenceGrp = myPwrGrp - someOtherGrp              # Shortcut
differenceGrp -= someOtherGrp                        # Shortcut
# Generator of pwr.Objs:
for pwrObj in myPwrGrp.GenerateDifference(someOtherPwrGrp):
    # Iterate on pwrObj...
#
# Where:
#   someOtherPwrGroup is a pwr.Grp object to merge with
# This method raises a pwr.PwrError exception when something goes wrong.
# The possible exception errors are:
#   pwr.ReturnCode.FAILURE
\end{lstlisting}\end{minipage}\end{center}

%==============================================================================%

\subsubsection{Method SymDifference} \label{meth:SymDifference}

The following creates new group containing members in the current group or
another but not members that are in both groups, that is, the symmetric
difference of the current group and another. This can be implemented as the
\texttt{Union()} minus the \texttt{Intersection()} of two groups.

\begin{center}\begin{minipage}{.95\linewidth}\begin{lstlisting}
symDifferenceGrp = myPwrGrp.SymDifference(someOtherPwrGrp)
symDifferenceGrp = myPwrGrp ^ someOtherGrp              # Shortcut
symDifferenceGrp ^= someOtherGrp                        # Shortcut

#
# Where:
#   someOtherPwrGroup is a pwr.Grp object to merge with
# This method raises a pwr.PwrError exception when something goes wrong.
# The possible exception errors are:
#   pwr.ReturnCode.FAILURE
#   pwr.ReturnCode.BAD_VALUE
\end{lstlisting}\end{minipage}\end{center}

%==============================================================================%
\subsubsection{Method GetGrpByName} \label{meth:GetGrpByName}

For general details see section \ref{func:CntxtGetGrpByName} on page
\pageref{func:CntxtGetGrpByName}. As noted in that description, valid group
names are vendor-specific. Use of this function should be considered
non-portable. Vendor-specific details should be documented by the API
implementor/vendor. An example of getting a group by name follows:

\begin{center}\begin{minipage}{.95\linewidth}\begin{lstlisting}
groupName = "vendor_supported_group_name_string"
myPwrGrp  = myPwrCntxt.GetGrpByName(groupName)
#
# Where:
#   groupName: vendor specific string designating group name
# Returns:
#   myPwrGrp is a pwr.Grp object or None if none found
# This method raises a pwr.PwrError exception when something goes wrong.
# The possible exception errors are:
#   pwr.ReturnCode.FAILURE
\end{lstlisting}\end{minipage}\end{center}

\subsection{Attribute Methods} \label{sec:PythonAttributeMethods}

%==============================================================================%
\subsubsection{Method pwr.Obj.AttrGetValue}
\label{meth:ObjAttrGetValue}

The \texttt{pwr.Obj} \texttt{AttrGetValue} method returns a Python named tuple
describing a measurement. A measurement is a \texttt{namedtuple} type from the
\texttt{collections} standard Python library module. Its
contents can best be described by listing the definition of the named tuple and
providing an example of how to access its members:

\begin{center}\begin{minipage}{.95\linewidth}\begin{lstlisting}
# Definition of the named tuple used to contain a measurement:
InfoFromGet = collections.namedtuple("InfoFromGet",
                                     "attr value obj timestamp rc")
\end{lstlisting}\end{minipage}\end{center}

\begin{center}\begin{minipage}{.95\linewidth}\begin{lstlisting}
# To return a single measurement:
attrName = pwr.AttrName.TEMP
measInfo = myPwrObj.AttrGetValue(attrName)

# Access the results:
measurementAttr = measInfo.attr
measurementValue = measInfo.value
measurementPwrObj = measInfo.obj
measurementTime = measInfo.timestamp
measurementError = measInfo.rc
#
# When a general failure occurs, a pwr.PwrError exception is raised.
# The possible exception errors are:
#   pwr.ReturnCode.FAILURE
#   pwr.ReturnCode.BAD_VALUE
\end{lstlisting}\end{minipage}\end{center}

%==============================================================================%
\subsubsection{Method pwr.Obj.AttrSetValue}
\label{meth:ObjAttrSetValue}

Similarly, there is an attribute ``Set'' method for the object, which is capable
of setting the value of one or more attributes on that object. This method
uses a named tuple similar to the one defined in
\ref{meth:ObjAttrGetValue} on page \pageref{meth:ObjAttrGetValue} to feed a
list of one or more attribute-value pairs to the attribute ``Set`` methods. A
definition of this named tuple and an example of how to create it follow for
providing an input to the Attribute ``Set'' methods:

\begin{center}\begin{minipage}{.95\linewidth}\begin{lstlisting}
# Definition of the named tuple used to contain a setting:
InfoForSet = collections.namedtuple("InfoForSet", "attr value")
\end{lstlisting}\end{minipage}\end{center}

Another named tuple definition is used to extract any error information
that the attribute operation(s) may yield:

\begin{center}\begin{minipage}{.95\linewidth}\begin{lstlisting}
# Definition of the named tuple used to contain error information
ErrorFromSet = collections.namedtuple("ErrorFromSet", "attr obj rc")
\end{lstlisting}\end{minipage}\end{center}

In the below example, the Attribute ``Set'' method, along with the named tuples for setting it and error handling is shown:

\begin{center}\begin{minipage}{.95\linewidth}\begin{lstlisting}
# To set a single attribute and handle any error that may occur:
setting = InfoForSet(attr=pwr.AttrName.CSTATE, value=3)
# The for-loop will catch any possible ErrorFromSet named tuples
# that the Set operation may yield.
for setError in myPwrObj.AttrSetValue(setting):
    errorAttribute = setError.attr
    errorPwrObj = setError.obj
    errorReturnCode = setError.rc
    # Process error here...
#
# In case of general failures, the possible exception errors are:
#   pwr.ReturnCode.FAILURE
#   pwr.ReturnCode.BAD_VALUE
\end{lstlisting}\end{minipage}\end{center}

%==============================================================================%
\subsubsection{Method pwr.Obj.AttrGetValues}
\label{meth:ObjAttrGetValues}

The \texttt{pwr.Obj} \texttt{AttrGetValues} method returns a list containing Python
measurement named tuples. Returning a list keeps consistency with the
\texttt{pwr.Grp} AttrGetValue() and \texttt{pwr.Grp} AttrGetValues()
methods which return the results from multiple measurements or queries as items
in a list. For each of the AttrGetValue(s) methods there is a generator
method which returns a memory-efficient iterator as opposed to a Python list.
Please refer to \ref{meth:ObjAttrGetValue} on page
\pageref{meth:ObjAttrGetValue} for details.

\begin{center}\begin{minipage}{.95\linewidth}\begin{lstlisting}
# To return a measurement for each attribute in the list:
attrList = [pwr.AttrName.TEMP, pwr.AttrName.VOLTAGE]
measList = myPwrObj.AttrGetValues(attrList)
for measInfo in measList:
    # Access the results:
    measurementAttr = measInfo.attr
    measurementValue = measInfo.value
    measurementPwrObj = measInfo.obj
    measurementTime = measInfo.timestamp
    measurementError = measInfo.rc

# To iterate on the results yielded by the generator method:
for measInfo in myPwrObj.AttrGenerateValues(attrList):
    # Access the results:
    measurementAttr = measInfo.attr
    measurementValue = measInfo.value  # etc.
\end{lstlisting}\end{minipage}\end{center}

%==============================================================================%
\subsubsection{Method pwr.Obj.AttrSetValues}
\label{meth:ObjAttrSetValues}

The \texttt{AttrSetValues} method sets the values for multiple attributes on a
\texttt{pwr.Obj}, yielding any errors that may have occurred.
Please refer to
\ref{meth:ObjAttrSetValue} on page \pageref{meth:ObjAttrSetValue} for details.

\begin{center}\begin{minipage}{.95\linewidth}\begin{lstlisting}
# To set multiple attributes and handle any errors that may occur:
setting1 = InfoForSet(attr=pwr.AttrName.CSTATE, value=3)
setting2 = InfoForSet(attr=pwr.AttrName.PSTATE, value=2)
settingList = [setting1, setting2]
for setError in myPwrObj.AttrSetValues(settingList):
    errorAttribute = setError.attr
    errorPwrObj = setError.obj
    errorReturnCode = setError.rc
    # Process error here...
\end{lstlisting}\end{minipage}\end{center}

%==============================================================================%
\subsubsection{Method pwr.Obj.AttrIsValid} \label{meth:ObjAttrIsValid}

To determine the validity of an attribute on a particular \texttt{pwr.Obj}
object:

\begin{center}\begin{minipage}{.95\linewidth}\begin{lstlisting}
pwrAttr = pwr.AttrName.ENERGY
attrGood = myPwrObj.AttrIsValid(pwrAttr)
attrGood = myPwrObj.ENERGY.isvalid       # Shortcut
#
# Where:
#   pwrAttr is a pwr.AttrName type
# Returns:
#   True or False
#
\end{lstlisting}\end{minipage}\end{center}

%==============================================================================%
\subsubsection{Method pwr.Grp.AttrGetValue}
\label{meth:GrpAttrGetValue}

The \texttt{pwr.Grp} \texttt{AttrGetValue} method returns a list containing Python
named tuples containing the resulting measurements of an attribute across a
\texttt{pwr.Grp}. Returning a list keeps consistency with the
\texttt{pwr.Obj} AttrGetValues() and \texttt{pwr.Grp} AttrGetValues()
methods which return the results from multiple measurements or queries as items
in a list. The InfoFromGet() named tuple is used in the same way
as with the pwrObj AttrGetValue(s) methods for containing
the ``measurement'' information. Please refer to \ref{meth:ObjAttrGetValue} on
page \pageref{meth:ObjAttrGetValue} for details. For this method there is a
generator method which returns a memory-efficient iterator as opposed to a
Python list.

\begin{center}\begin{minipage}{.95\linewidth}\begin{lstlisting}
# Return measurements for the given attribute for all group members
measList = myPwrGrp.AttrGetValue(pwr.AttrName.TEMP)
for measInfo in measList:
    # Access the results:
    measurementAttr = measInfo.attr
    measurementValue = measInfo.value
    measurementPwrObj = measInfo.obj
    measurementTime = measInfo.timestamp
    measurementError = measInfo.rc

# To iterate on the results yielded by the generator method:
for measInfo in myPwrGrp.AttrGenerateValues(pwr.AttrName.TEMP):
    # Access the results:
    measurementAttr = measInfo.attr
    measurementValue = measInfo.value  # etc.
#
# When a failure occurs, a pwr.PwrError exception is raised.
# The possible exception errors are:
#   pwr.ReturnCode.FAILURE
#   pwr.ReturnCode.BAD_VALUE
\end{lstlisting}\end{minipage}\end{center}

%==============================================================================%
\subsubsection{Method pwr.Grp.AttrSetValue}
\label{meth:GrpAttrSetValue}

The \texttt{pwr.Grp} \texttt{AttrSetValue} method sets the value of an attribute on all the
\texttt{pwr.Obj} objects across a \texttt{pwr.Grp}.  The named tuple definitions
InfoForSet() and ErrorFromSet() are used in the same way as with the pwrObj
AttrSetValue(s) methods for extraction and construction of the ``settings''
and error named tuples. Please refer to \ref{meth:ObjAttrGetValue} on page
\pageref{meth:ObjAttrGetValue} and \ref{meth:ObjAttrSetValue} on page
\pageref{meth:ObjAttrSetValue} for details.

\begin{center}\begin{minipage}{.95\linewidth}\begin{lstlisting}
# Set a single attribute for all objects in a group
# and handle any errors that may occur:
setting = pwr.InfoForSet(attr=pwr.AttrName.CSTATE, value=3)
for setError in myPwrGrp.AttrSetValue(setting):
    errorAttribute = setError.attr
    errorPwrObj = setError.obj
    errorReturnCode = setError.rc
    # Process error on particular object
#
# In case of general failures, the possible exception errors are:
#   pwr.ReturnCode.FAILURE
#   pwr.ReturnCode.BAD_VALUE
\end{lstlisting}\end{minipage}\end{center}

%==============================================================================%
\subsubsection{Method pwr.Grp.AttrGetValues}
\label{meth:GrpAttrGetValues}

The \texttt{pwr.Grp.AttrGetValues()} method returns a list containing Python
measurement named tuples. Returning a list maintains consistency with the
\texttt{pwr.Obj.AttrGetValues()} and \texttt{pwr.Grp.AttrGetValue()}
methods which return the results from multiple measurements or queries as items
in a list. For each of the AttrGetValue(s) methods, there is a generator
method which returns a memory-efficient iterator as opposed to a Python list:

\begin{center}\begin{minipage}{.95\linewidth}\begin{lstlisting}
# To return a list of measurements of a list of attributes across the
# pwr.Obj members of a pwr.Grp, and access the results:
attrList = [pwr.AttrName.TEMP, pwr.AttrName.POWER]
measList = myPwrGrp.AttrGetValues(attrList)
for measInfo in measList:
    # Access the results:
    measurementAttr = measInfo.attr
    measurementValue = measInfo.value
    measurementPwrObj = measInfo.obj
    measurementTime = measInfo.timestamp
    measurementError = measInfo.rc

# To iterate on the results yielded by the generator method:
for measInfo in myPwrGrp.AttrGenerateValues(attrList):
    # Access the results:
    measurementAttr = measInfo.attr
    measurementValue = measInfo.value  # etc.
\end{lstlisting}\end{minipage}\end{center}

%==============================================================================%
\subsubsection{Method pwr.Grp.AttrSetValues}
\label{meth:GrpAttrSetValues}

This method sets values for multiple attributes on all the objects in a
group.

\begin{center}\begin{minipage}{.95\linewidth}\begin{lstlisting}
# To set multiple attributes across all objects of a group
# and handle any errors that may occur:
setting1 = pwr.InfoForSet(attr=pwr.AttrName.CSTATE, value=3)
setting2 = pwr.InfoForSet(attr=pwr.AttrName.PSTATE, value=2)
settingList = [setting1, setting2]
for setError in myPwrGrp.AttrSetValues(settingList):
    errorAttribute = setError.attr
    errorPwrObj = setError.obj
    errorReturnCode = setError.rc
    # Process error on particular attr for particular object.
\end{lstlisting}\end{minipage}\end{center}

\subsection{Metadata Methods} \label{sec:PythonMetadataMethods}
The C API metadata functions (see page \pageref{sec:METADATA}) are represented
in Python API implementations as class methods to the \texttt{pwr.Obj}
object.

%==============================================================================%
\subsubsection{Method AttrGetMeta} \label{meth:AttrGetMeta}

This method returns a metadata value associated with a \texttt{pwr.Obj} attribute.
\begin{center}\begin{minipage}{.95\linewidth}\begin{lstlisting}
attrName = pwr.AttrName.TEMP
metaName = pwr.MetaName.MAX
metaValue = myPwrObj.AttrGetMeta(attrName, metaName)
metaValue = myPwrObj.TEMP.MAX    # Shortcut
#
# Where:
#   attrName is the pwr.AttrName attribute to get the meta info for
#   metaName is the pwr.MetaName meta information to set
# Returns:
#   metaValue: the meta information requested
# This method raises a pwr.PwrError exception when something goes wrong.
# The possible exception errors are:
#   pwr.ReturnCode.FAILURE
#   pwr.ReturnCode.NO_ATTRIB
#   pwr.ReturnCode.NO_META
\end{lstlisting}\end{minipage}\end{center}

%==============================================================================%
\subsubsection{Method AttrSetMeta} \label{meth:AttrSetMeta}

This method writes a metadata value to the \texttt{pwr.Obj}'s attribute's metadata.
\begin{center}\begin{minipage}{.95\linewidth}\begin{lstlisting}
attrName = pwr.AttrName.CSTATE
metaName = pwr.MetaName.SAMPLE_RATE
myPwrObj.AttrSetMeta(attrName, metaName, 100)
myPwrObj.CSTATE.SAMPLE_RATE = 100   # Shortcut
#
# Where:
#   attrName is the pwr.AttrName attribute to get the meta info for
#   metaName is the pwr.MetaName meta information to set
#   metaValue: the meta information to set
# This method raises a pwr.PwrError exception when something goes wrong.
# The possible exception errors are:
#   pwr.ReturnCode.FAILURE
#   pwr.ReturnCode.NO_ATTRIB
#   pwr.ReturnCode.NO_META
#   pwr.ReturnCode.READ_ONLY
#   pwr.ReturnCode.BAD_VALUE
\end{lstlisting}\end{minipage}\end{center}

%==============================================================================%
\subsubsection{Method GetMetaValueAtIndex} \label{meth:GetMetaValueAtIndex}

This method returns a two-item tuple with the metadata value and a string representation of that value.
\begin{center}\begin{minipage}{.95\linewidth}\begin{lstlisting}
attrName = pwr.AttrName.CSTATE
metaValue, metaString = myPwrObj.GetMetaValueAtIndex(attrName, 1)
metaValue, metaString = myPwrObj.CSTATE[1]   # Shortcut
#
# Where:
#   attrName is a pwr.AttrName type
#   index is the index of the meta data item
# Returns:
#   metaValue is the meta information requested
#   metaString is the string version of the meta information
# This method raises a pwr.PwrError exception when something goes wrong.
# The possible exception errors are:
#   pwr.ReturnCode.FAILURE
#   pwr.ReturnCode.NO_ATTRIB
#   pwr.ReturnCode.BAD_INDEX
\end{lstlisting}\end{minipage}\end{center}

\subsection{Statistics Methods} \label{sec:PythonStatisticsMethods}

Statistics are applied either to Python \texttt{pwr.Obj} or a \texttt{pwr.Grp}
objects. Because of this, the various statistics methods are either
encapsulated by the \texttt{pwr.Obj} or the \texttt{pwr.Grp} classes. See
section \ref{sec:StatisticsFunctions} starting on page
\pageref{sec:StatisticsFunctions} for C API documentation on statistics.

%==============================================================================%
\subsubsection{Method pwr.Obj.GetStat} \label{meth:ObjGetStat}

Return a named tuple describing the requested historic statistic. Refer to
\ref{meth:ObjAttrGetValue} on page \pageref{meth:ObjAttrGetValue} for details
of the \texttt{InfoFromGet()} named tuple to access the information returned. The
C API equivalent of this method is documented in section
\ref{func:ObjGetStat} on page \pageref{func:ObjGetStat}.

\begin{center}\begin{minipage}{.95\linewidth}\begin{lstlisting}
# To return a single historic statistic:
attrName = pwr.AttrName.POWER
attrStat = pwr.AttrStat.AVG
endTime = Time(time.time())                                   # current time.
timePeriod = timePeriod(start=(endTime-3600.0), end=endTime)  # one hour.
statInfo = myPwrObj.GetStat(attrName, attrStat, timePeriod)
# Where:
#     attrName is a pwr.AttrName attribute name
#     attrStat is the pwr.AttrStat statistic to gather
#     timePeriod is the desired time of the statistic
# To access the results:
statisticValue = statInfo.value
statisticTimePeriod = statInfo.timestamp
statisticErrorCode = statInfo.rc
#
# When a general failure occurs, a pwr.PwrError exception is raised.
# The possible exception errors are:
#   pwr.ReturnCode.FAILURE
#   pwr.ReturnCode.BAD_VALUE
\end{lstlisting}\end{minipage}\end{center}

%==============================================================================%
\subsubsection{Method pwr.Grp.GetStats} \label{class:GrpGetStats}

This method returns a list containing Python named tuples describing historic statistics
across the objects of a \texttt{pwr.Grp}.
The C API equivalent of this method is documented in section
\ref{func:GrpGetStats} on page \pageref{func:GrpGetStats}.

\begin{center}\begin{minipage}{.95\linewidth}\begin{lstlisting}
# To return historic statistics over the objects of a pwr.Grp:
attrName = pwr.AttrName.POWER
attrStat = pwr.AttrStat.AVG
endTime = Time(time.time())                                   # current time.
timePeriod = timePeriod(start=(endTime-3600.0), end=endTime)  # one hour.
statList = myPwrGrp.GetStats(attrName, attrStat, timePeriod)
# Where:
#     attrName is a pwr.AttrName attribute name
#     attrStat is the pwr.AttrStat statistic to gather
#     timePeriod: is the desired TimePeriod of the statistic, or None
# To access the results:
for statInfo in statList:
    # Access the results:
    statisticValue = statInfo.value
    statisticPwrObj = statInfo.obj
    statisticTimePeriod = statInfo.timestamp
    statisticErrorCode = statInfo.rc
    # Process statistic...
\end{lstlisting}\end{minipage}\end{center}

%==============================================================================%
\subsubsection{Class Stat} \label{class:CreateStat}

A \texttt{pwr.Stat} instance provides real-time statistics functionality and may be associated with a \texttt{pwr.Obj} or \texttt{pwr.Grp} object.

%==============================================================================%
\subsubsection{Method pwr.Obj.CreateStat } \label{meth:ObjCreateStat}

This method creates a \texttt{pwr.Obj.Stat} object:

\begin{center}\begin{minipage}{.95\linewidth}\begin{lstlisting}
attrName  = pwr.AttrName.POWER
attrStat  = pwr.AttrStat.AVG
myPwrStat = myPwrObj.CreateStat(attrName, attrStat)
#
# Where:
#   attrName is the pwr.AttrName attribute to get the statistics for
#   attrAttrStat is a pwr.AttrStat object
# Returns:
#   myPwrStat : a pwr.Stat object
# This method raises a pwr.PwrError exception when something goes wrong.
# The possible exception errors are:
#   pwr.ReturnCode.FAILURE
\end{lstlisting}\end{minipage}\end{center}

%==============================================================================%
\subsubsection{Method pwr.Grp.CreateStat } \label{meth:GrpCreateStat}

This method creates a \texttt{pwr.Grp.Stat} object.

\begin{center}\begin{minipage}{.95\linewidth}\begin{lstlisting}
attrName = pwr.AttrName.TEMP
attrStat = pwr.AttrStat.MAX
myGrpPwrStat = myPwrGrp.CreateStat(attrName, attrStat)
#
# Where:
#   attrName is the pwr.AttrName attribute to get the statistics for
#   attrAttrStat is a pwr.AttrStat object
# Returns:
#   myGrpPwrStat: a pwr.Stat object
# This method raises a pwr.PwrError exception when something goes wrong.
# The possible exception errors are:
#   pwr.ReturnCode.FAILURE
\end{lstlisting}\end{minipage}\end{center}

%==============================================================================%
\subsubsection{Method pwr.Stat.Start} \label{meth:StatStart}

This method starts the collection of real-time statistics on either the
\texttt{pwr.Obj.Stat} or \texttt{pwr.Grp.Stat} object:

\begin{center}\begin{minipage}{.95\linewidth}\begin{lstlisting}
myPwrStat.Start()  # Start gathering real-time stats
#
# This method raises a pwr.PwrError exception when something goes wrong.
# The possible exception errors are:
#   pwr.ReturnCode.FAILURE
\end{lstlisting}\end{minipage}\end{center}

%==============================================================================%
\subsubsection{Method pwr.Stat.Stop} \label{meth:StatStop}

This method stops the collection of real-time statistics on either the
\texttt{pwr.Obj.Stat} or \texttt{pwr.Grp.Stat} object:

\begin{center}\begin{minipage}{.95\linewidth}\begin{lstlisting}
myPwrStat.Stop()  # Stop gathering real-time stats
#
# this method raises a pwr.PwrError exception when something goes wrong.
# The possible exception errors are:
#   pwr.ReturnCode.FAILURE
\end{lstlisting}\end{minipage}\end{center}

%==============================================================================%
\subsubsection{Method pwr.Stat.Clear} \label{meth:StatClear}

This method resets the collection of real-time statistics on either the
\texttt{pwr.Obj.Stat} or \texttt{pwr.Grp.Stat} object:

\begin{center}\begin{minipage}{.95\linewidth}\begin{lstlisting}
myPwrStat.Clear()
#
# This method raises a pwr.PwrError exception when something goes wrong.
# The possible exception errors are:
#   pwr.ReturnCode.FAILURE
\end{lstlisting}\end{minipage}\end{center}

%==============================================================================%
\subsubsection{Method pwr.Stat.GetValue} \label{meth:StatGetValue}

This method returns a named tuple describing the requested real-time statistic. Refer to
\ref{meth:ObjAttrGetValue} on page \pageref{meth:ObjAttrGetValue} for details
of the \texttt{InfoFromGet()} named tuple to access the information returned. The
C API equivalent of this method is documented in section
\ref{func:StatGetValue} on page \pageref{func:StatGetValue}.

\begin{center}\begin{minipage}{.95\linewidth}\begin{lstlisting}
# To return a single real-time statistic:
myObjPwrStat = myPwrObj.CreateStat(pwr.AttrName.TEMP, pwr.AttrStat.MAX)
myObjPwrStat.Start()  # Start gathering real-time stats
# (Do something useful...)
myObjPwrStat.Stop()   # Stop gathering real-time stats
statInfo = myObjPwrStat.GetValue()
# To access the results:
statisticValue = statInfo.value
statisticTimePeriod = statInfo.timestamp
statisticErrorCode = statInfo.rc
#
# When a general failure occurs, a pwr.PwrError exception is raised.
# The possible exception errors are:
#   pwr.ReturnCode.FAILURE
#   pwr.ReturnCode.BAD_VALUE
\end{lstlisting}\end{minipage}\end{center}

%==============================================================================%
\subsubsection{Method pwr.Stat.GetValues} \label{meth:StatGetValues}

This method returns a list containing Python named tuples describing real-time statistics across
the objects of the \texttt{pwr.Grp} referenced in this \texttt{pwr.Stat} object.
The C API equivalent of this method is documented in section
\ref{func:StatGetValues} on page \pageref{func:StatGetValues}.

\begin{center}\begin{minipage}{.95\linewidth}\begin{lstlisting}
# To return a real-time statistic across the objects of a pwr.Grp:
myGrpPwrStat = myPwrGrp.CreateStat(pwr.AttrName.TEMP, pwr.AttrStat.MAX)
myGrpPwrStat.Start()  # Start gathering real-time stats
# (Do something useful...)
myGrpPwrStat.Stop()  # Stop gathering real-time stats
# Collect the statistics:
statList = myGrpPwrStat.GetValues()
for statInfo in statList:
    # Access the results:
    statisticValue = statInfo.value
    statisticPwrObj = statInfo.obj
    statisticTimePeriod = statInfo.timestamp
    statisticErrorCode = statInfo.rc
    # Process statistic...
#
# When a general failure occurs, a pwr.PwrError exception is raised.
# The possible exception errors are:
#   pwr.ReturnCode.FAILURE
#   pwr.ReturnCode.BAD_VALUE
\end{lstlisting}\end{minipage}\end{center}

%==============================================================================%
\subsubsection{Method pwr.Stat.GetReduce} \label{meth:StatGetReduce}

A reduction of a real-time statistic can be retrieved with the
\texttt{pwr.Stat.GetReduce()} method. For a description of the reduction
operation refer to the C API description of \texttt{PWR_StatGetReduce()} on page
\pageref{func:StatGetReduce}.

\begin{center}\begin{minipage}{.95\linewidth}\begin{lstlisting}
# Get a reduction of real-time attribute values
reduceOp = pwr.AttrStat.AVG
reduceInfo = myGrpPwrStat.GetReduce(reduceOp)
reduceValue = reduceInfo.value
reduceTimePeriod = reduceInfo.timestamp
reduceErrorCode = reduceInfo.rc
# Where:
#     reduceOp: AttrStat reduction operation to get
# Returns:
#     A named tuple of the attr, value, obj (group), timestamp, and rc.
# This method will raise a PwrError exception when something goes wrong.
# The possible exception errors are:
#     ReturnCode.FAILURE
\end{lstlisting}\end{minipage}\end{center}

%==============================================================================%
\subsubsection{Method pwr.Grp.GetReduce} \label{meth:GrpGetReduce}

A reduction of a historic statistic can be retrieved with the
\texttt{pwr.Grp.GetReduce()} method. For a description of the reduction
operation refer to the C API description of \texttt{PWR_GrpGetReduce()} on page
\pageref{func:GrpGetReduce}.

\begin{center}\begin{minipage}{.95\linewidth}\begin{lstlisting}
# Get a reduction of historic attribute values
import time
attrName = pwr.AttrName.TEMP
attrStat = pwr.AttrStat.MAX
reduceOp = pwr.AttrStat.AVG
endTime = Time(time.time())                                   # current time.
timePeriod = timePeriod(start=(endTime-3600.0), end=endTime)  # 1 hour period
reduceInfo = myPwrGrp.GetReduce(attrName, attrStat, reduceOp, timePeriod)
reduceValue = reduceInfo.value
reduceTimePeriod = reduceInfo.timestamp
reduceErrorCode = reduceInfo.rc
# Where:
#     attrName: AttrName attribute for which to gather statistics
#     attrStat: AttrStat historic statistic to gather
#     reduceOp: AttrStat reduction operation over the objects in the group.
#     timePeriod: pwr.TimePeriod period to get statistic.
# Returns:
#     A named tuple of the attr, value, obj (group), timestamp, and rc.
# This method will raise a PwrError exception when something goes wrong.
# The possible exception errors are:
#     ReturnCode.FAILURE
\end{lstlisting}\end{minipage}\end{center}


%==============================================================================%
\subsubsection{Method pwr.Stat.Destroy NOT_IMPLEMENTED}
\label{meth:StatDestroy}

There is no need for a ``Destroy'' method due to Python's garbage collection implementation.

\subsection{Version Functions} \label{sec:PythonVersionFunctions}
The top-level Version functions are not associated with any object. They return
an integer detailing a particular segment of the version of the API. There also
is an included \texttt{version} variable available to obtain a string version of the
major/minor version number of the API

%==============================================================================%
\subsubsection{Method GetMajorVersion} \label{meth:GetMajorVersion}

\begin{center}\begin{minipage}{.95\linewidth}\begin{lstlisting}
majorVersion = pwr.GetMajorVersion()
majorVersion = pwr.majorVersion      # Shortcut
versionStr = pwr.version             # Gives "1.2" as example.
#
# This method raises a pwr.PwrError exception when something goes wrong.
# The possible exception errors are:
#   pwr.ReturnCode.FAILURE
\end{lstlisting}\end{minipage}\end{center}

%==============================================================================%
\subsubsection{Method GetMinorVersion} \label{meth:GetMinorVersion}

\begin{center}\begin{minipage}{.95\linewidth}\begin{lstlisting}
minorVersion = pwr.GetMinorVersion()
minorVersion = pwr.minorVersion      # Shortcut
#
# These methods raises a pwr.PwrError exception when something goes wrong.
# The possible exception errors are:
#   pwr.ReturnCode.FAILURE
\end{lstlisting}\end{minipage}\end{center}

\subsection{Big List of Attributes} \label{sec:PythonBigListOfAttributes}

The list of attributes for the default context is the same as the C API section
\ref{sec:BLOA} starting on page \pageref{sec:BLOA}, with the attributes
enumerated as defined in \ref{class:AttrName} on page \pageref{class:AttrName}.

\subsection{Big List of Metadata} \label{sec:PythonBigListOfMetadata}

The list of metadata names for the default context is the same as in the C API
section \ref{sec:BLOM} on page \pageref{sec:BLOM}, with the attributes
enumerated as defined in \ref{class:MetaName} on page \pageref{class:MetaName}.
