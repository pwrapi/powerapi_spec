\documentclass[12pt]{report}
\usepackage{caption}
\usepackage{longtable,tabu}
\usepackage{enumitem}
\usepackage{hyperref}			
\usepackage{graphicx}
\usepackage{marginnote}
\usepackage[nohyphen]{underscore}
%\usepackage{imakeidx}
\usepackage{makeidx}
\usepackage{color}
\usepackage[toc,page]{appendix}
\usepackage{listings}
\newcommand\Mark[1]{\textsuperscript#1}
\lstset{ %
	language=C,                		% choose the language of the code
	columns= [l]fullflexible,
	basicstyle=\ttfamily\footnotesize,      % the size of the fonts that are used for the code
	commentstyle=\color{blue},      	% format of commentstyle 
	numberstyle=\footnotesize,      	% the size of the fonts that are used for the line-numbers
	stepnumber=1,                   	% the s	tep between two line-numbers. If it is 1 each line will be numbered
	numbersep=5pt,                  	% how far the line-numbers are from the code
	backgroundcolor=\color{white},  	% choose the background color. You must add \usepackage{color}
	keepspaces=false,
	showspaces=false,               	% show spaces adding particular underscores
	showstringspaces=false,         	% underline spaces within strings
	showtabs=false,                 	% show tabs within strings adding particular underscores
	frame=single,           		% adds a frame around the code
	tabsize=2,          			% sets default tabsize to 2 spaces
	captionpos=b,           		% sets the caption-position to bottom
	breaklines=true,        		% sets automatic line breaking
	breakatwhitespace=false,    		% sets if automatic breaks should only happen at whitespace
	escapeinside={\%*}{*)}          	% if you want to add a comment within your code
}
\makeindex
% Breaks index assembly
%\makeindex[intoc]
% Breaks index assembly
%\usepackage[totoc]{idxlayout}			% Force index into table of contents.

%%%%%%%%%%%%%%%%%%%%%%%%%%%%%%%%%%%%%%%%%%%%%%%%%%%%%%%%%%%%%%%%%%%%%%%%%%%%%%%%
%%% Function prototype macros.
%%%%%%%%%%%%%%%%%%%%%%%%%%%%%%%%%%%%%%%%%%%%%%%%%%%%%%%%%%%%%%%%%%%%%%%%%%%%%%%%
%
% Function prototype descriptions are now completely parameterized.  An example for PWR_ObjGetName():
%
%        \begin{prototype}{ObjGetName}
%        \longdescription{
%                Copies the name of the specified object into the user provided buffer.
%                See page \pageref{func:CntxtGetObjByName} to get the object based on the
%                unique name using \CntxtGetObjByName.
%        }
%        \returntype{int}
%
%        \parameter{\PWR{Obj} object}    {Input} {The object that the user wishes to determine the name of.}
%        \parameter{char* dest}          {Input} {The address of the user provided buffer.}
%        \parameter{size_t len}          {Input} {The length of the user provided buffer.}
%
%        \returnval{\PWR{RET_SUCCESS}}           {Upon SUCCESS, the buffer will contain the name of the object, the string will include a terminating null byte.}
%        \returnval{\PWR{RET_WARN_TRUNC}}        {Call succeeded, but the length of object name was longer than the provided buffer and the name was truncated.}
%        \returnval{\PWR{RET_FAILURE}}           {Upon FAILURE.}
%        \end{prototype}
%
% This approach allows all of the formatting to be contained in a single 
% set of functions, rather than hard-coded in each of the function prototype
% subsections.  Thanks to Victor Eijkhout and Gilles Castel for coding 
% assistance.  --BLR

\def\PWR#1{\texttt{PWR\_{#1}}}%                         % Adds PWR_ and texttt formatting.
\def\null{}
\def\undecoratedname{}                                  % Undecorated function name, e.g., ObjGetName (not PWR_ObjGetName)
\def\longdesc{}                                         % Paragraph describing function details.
\def\returns{}                                          % Return type of the function
\def\params{}                                           % Table-formatted accumulator for function parameters and descriptions.
\def\retvals{}                                          % Table-formatted accumulator for available return values.
\def\signatureparams{}                                  % Comma-formatted list of parameter types and names.
\def\parameternote{}					% Text to be hung in the right margin (at the moment)
%\reversemarginpar					% Make the left margin the default for margin notes.


\newenvironment{prototype}[1]{
        % This stanza is placed at the beginning of the environment.
        \def\undecoratedname{#1}
}{
        % This stanza is placed at the end of the environment.

	% Subsection Header
        \subsubsection{Function Prototype for \PWR{\undecoratedname}()}\label{func:\undecoratedname}
	\index{\undecoratedname@\PWR{\undecoratedname} \textit{(function)}}

	% Long description of the function prototype.
	\longdesc

        \vspace{0.1in}

	% Function signature
	\begin{center}
		\begin{tabular}{| p{13.2cm} |}
			\hline
			\noindent\small{\texttt{\returns\ \PWR{\undecoratedname}(\signatureparams)}}\\
			\hline
		\end{tabular}
	\end{center}

	% Table of function parameters
	\ifx \params\null
		%
	\else
		\begin{center}
			\begin{tabular}{ | p{0.9cm} p{4.8cm} | p{6.7cm} |}
				\hline
				\multicolumn{2}{|l|}{\textbf{Arguments}} & \textbf{Description}\marginnote{\scriptsize\parameternote}\\
				\hline
				\params%
				\hline
			\end{tabular}
		\end{center}
	\fi	


	% Table of return values
	\ifx \retvals\null
		%
	\else
		\begin{center}
			\begin{tabular}{ | p{6.1cm} | p{6.8cm} |}
				\hline
				\textbf{Return Code(s)} & \textbf{Description} \\
				\hline
				\retvals%
				\hline
			\end{tabular}
		\end{center}
	\fi
}

% Parameter accumulator
\newcommand\parameter[3]{
	
	% Puts parameters into table format.
	\edef\params{%
		\unexpanded\expandafter{\params}%
		\unexpanded{#2 & \texttt{#1} & #3\\}%
	}%  

	% Puts parameters into comma-separated signature format.
	\ifx\signatureparams\empty
		\edef\signatureparams{%
			\unexpanded\expandafter{\signatureparams}%
			\unexpanded{#1}%
		}
	\else
		\edef\signatureparams{%
			\unexpanded\expandafter{\signatureparams}%
			, \unexpanded{#1}%
		}
	\fi
}

% Stores the long description of the function prototype.
\def\longdescription#1{
        \def\longdesc{#1}
}

% Parameter marginal note.
\def\pnote#1{
	\def\parameternote{#1}
}

% Return value accumulator.
\newcommand\returnval[2]{
        \edef\retvals{%
                \unexpanded\expandafter{\retvals}%
                \unexpanded{\texttt{#1} & #2\\}%
        }%
}

% Stores the return type of the function prototype.
\def\returntype#1{
        \def\returns{#1}
}

% Allow global modification of how we represent Input and Output variables.
\def\pInput{\tiny{\textbf{IN}}}
\def\pOutput{\tiny{\textbf{OUT}}}
\def\pInputOutput{\tiny{\textbf{IN/OUT}}}

%%%%%%%%%%%%%%%%%%%%%%%%%%%%%%%%%%%%%%%%%%%%%%%%%%%%%%%%%%%%%%%%%%%%%%%%%%%%%%%%
%%% Attribute (and metadata) table macros.
%%%%%%%%%%%%%%%%%%%%%%%%%%%%%%%%%%%%%%%%%%%%%%%%%%%%%%%%%%%%%%%%%%%%%%%%%%%%%%%%
%
% At this writing, each particular attribute keeps its get/set characteristics
% across all interfaces.  

\def\attributetablename{}
\def\attributetablelabel{}
\def\attriblist{}

\newenvironment{attributetable}[2]{
        % This stanza is placed at the beginning of the environment.
        \def\attributetablename{#1}
	\def\attributetablelabel{#2}
}{
        % This stanza is placed at the end of the environment.
	\begin{longtable}{ | p{6.3cm} | p{6.6cm} |}
	\caption{\attributetablename}\label{\attributetablelabel}\\
	\hline											%%			
	\textbf{Attribute, Get/Set, Type}& \textbf{Description}\\  				%% First header.
	\hline											%%
	\endfirsthead					
	\multicolumn{2}{r}{{ \small{\tablename\ \thetable{} -- continued from previous page}}}\\%% Remaining headers.
	\hline											%%
	\textbf{Attribute, Get/Set, Type} & \textbf{Description}\\ 				%% First header.
	\hline											%%
	\endhead
	\hline											%% 
	\multicolumn{2}{r}{\small{{Continued on next page}}} \\					%% All but last footer.
	\endfoot
	\hline											%% Last footer.
	\endlastfoot										
	\attriblist
	\hline
	\end{longtable}
}
\def\attribute#1#2#3#4{
	\edef\attriblist{%
		\unexpanded\expandafter{\attriblist}%
		\unexpanded{ \hline \PWR{ATTR\_#1} \newline .\quad #2 \newline .\quad #3 & #4 \index{ATTR\_#1@\PWR{ATTR\_\uppercase{#1}}\textit{(attribute)}}\\ }%
	}%
}

% This is a near-duplicate of the attribute table above.
% We'll use the same variables.
\newenvironment{metadatatable}[2]{								
        % This stanza is placed at the beginning of the environment.				
        \def\attributetablename{#1}
	\def\attributetablelabel{#2}
}{
        % This stanza is placed at the end of the environment.
	\begin{longtable}{ | p{6.3cm} | p{6.6cm} |}
	\caption{\attributetablename}\label{\attributetablelabel}\\
	\hline											%%			
	\textbf{Metadata, Get/Set, Type}& \textbf{Description}\\  				%% First header.
	\hline											%%
	\endfirsthead					
	\multicolumn{2}{r}{{ \small{\tablename\ \thetable{} -- continued from previous page}}}\\%% Remaining headers.
	\hline											%%
	\textbf{Metadata, Get/Set, Type} & \textbf{Description}\\ 				%% First header.
	\hline											%%
	\endhead
	\hline											%% 
	\multicolumn{2}{r}{\small{{Continued on next page}}} \\					%% All but last footer.
	\endfoot
	\hline											%% Last footer.
	\endlastfoot										
	\attriblist
	\hline
	\end{longtable}
}
\def\metadata#1#2#3#4{
	\edef\attriblist{%
		\unexpanded\expandafter{\attriblist}%
		\unexpanded{ \hline \PWR{MD\_#1} \newline .\quad #2 \newline .\quad #3 & #4 \index{MD\_#1@\PWR{MD\_\uppercase{#1}}\textit{(metadata)}}\\ }%
	}%
}

\def\ATTR#1{ATTR\_{\uppercase{#1}}}%	% convert foo into ATTR_FOO
\def\MD#1{MD\_{\uppercase{#1}}}%	% convert bar into MD_BAR
\def\aG{Get}
\def\aS{Set}
\def\aGS{Get/Set}
\def\uint{\small{\texttt{uint64\_t}}}
\def\dbl{\small{\texttt{double}}}
\def\SaA{\small{Same type as attribute}}
\def\ptrchar{\texttt{char *}}



%%%%%%%%%%%%%%%%%%%%%%%%%%%%%%%%%%%%%%%%%%%%%%%%%%%%%%%%%%%%%%%%%%%%%%%%%%%%%%%%
%%% Environments for lists of #defines, typedefs, enums and structs.
%%%%%%%%%%%%%%%%%%%%%%%%%%%%%%%%%%%%%%%%%%%%%%%%%%%%%%%%%%%%%%%%%%%%%%%%%%%%%%%%
%
% These were originally kept in a listings environment, but that prevents 
% additional markup such as labels and index entries.

% Accumulator for typesdefs and defines
\def\typedefsanddefines{}

\newenvironment{typedefs}{
}{
	\begin{center}
		%\begin{tabular}{| p{3cm} p{3cm} p{6cm} |}
		\begin{tabular}{| lll |}
			\hline
			\typedefsanddefines
			\hline
		\end{tabular}
	\end{center}
}


\newcommand\typedef[2]{
	\edef\typedefsanddefines{%
		\unexpanded\expandafter{\typedefsanddefines}%
		\unexpanded{%
			\texttt{typedef} &% 
			\texttt{#1} &%
			\PWR{#2}\label{typedef:#2} \index{#2@\PWR{#2} \textit{(typedef)}}\\}%
	}%
}%

\newcommand\pounddefine[2]{
	\edef\typedefsanddefines{%
		\unexpanded\expandafter{\typedefsanddefines}%
		\unexpanded{%
			\texttt{\#define} &%
			\PWR{#1} &%
			\texttt{#2}\label{define:#2} \index{#2@\PWR{#2} \textit{(\#define)}}\\}%
	}%
}%

% #defines that have underscores in them have to be treated individually.
\newcommand\pounddefineMAJORVERSION{%
	\edef\typedefsanddefines{%
		\unexpanded\expandafter{\typedefsanddefines}%
		\unexpanded{%
			\texttt{\#define} &%
			\PWR{MAJOR\_VERSION} &%
			\texttt{2}\label{define:MAJORVERSION} \index{MAJOR\_VERSION@\PWR{MAJOR\_VERSION} \textit{(\#define)}}\\}%
	}%
}%
	
\newcommand\pounddefineMINORVERSION{%
	\edef\typedefsanddefines{%
		\unexpanded\expandafter{\typedefsanddefines}%
		\unexpanded{%
			\texttt{\#define} &%
			\PWR{MINOR\_VERSION} &%
			\texttt{0}\label{define:MINORVERSION} \index{MINOR\_VERSION@\PWR{MINOR\_VERSION} \textit{(\#define)}}\\}%
	}%
}%

\newcommand\pounddefineMAXSTRINGLEN{%
	\edef\typedefsanddefines{%
		\unexpanded\expandafter{\typedefsanddefines}%
		\unexpanded{%
			\texttt{\#define} &%
			\PWR{MAX\_STRING\_LEN} &%
			\textit{vendor-defined}\label{define:MAXSTRINGLEN} \index{MAX\_STRING\_LEN@\PWR{MAX\_STRING\_LEN} \textit{(\#define)}}\\}%
	}%
}%

\newcommand\pounddefineCNTXTDEFAULT{%
	\edef\typedefsanddefines{%
		\unexpanded\expandafter{\typedefsanddefines}%
		\unexpanded{%
			\texttt{\#define} &%
			\PWR{CNTXT\_DEFAULT} &%
			\textit{0}\label{define:CNTXTDEFAULT} \index{CNTXT\_DEFAULT2@\PWR{CNTXT\_DEFAULT} \textit{(\#define)}}\\}%
	}%
}%


\newcommand\pounddefineCNTXTVENDOR{%
	\edef\typedefsanddefines{%
		\unexpanded\expandafter{\typedefsanddefines}%
		\unexpanded{%
			\texttt{\#define} &%
			\PWR{CNTXT\_VENDOR} &%
			\textit{0}\label{define:CNTXTVENDOR} \index{CNTXT\_VENDOR@\PWR{CNTXT\_VENDOR} \textit{(\#define)}}\\}%
	}%
}%
%%%%%%%%%%%%%%%%%%%%%%%%%%%%%%%%%%%%%%%%%%%%%%%%%%%%%%%%%%%%%%%%%%%%%%%%%%%%%%%%
%%% Label, Reference and Index macros
%%%%%%%%%%%%%%%%%%%%%%%%%%%%%%%%%%%%%%%%%%%%%%%%%%%%%%%%%%%%%%%%%%%%%%%%%%%%%%%%

% References will be automatically hyperlinked.  
\newcommand\funcref[1]{\hyperref[func:#1]{\PWR{#1}}\texttt{()}\marginnote{\scriptsize{p.~\pageref{func:#1}}}}
\newcommand\funcrefx[1]{\hyperref[func:#1]{\PWR{#1}}}

\newcommand\typeref[1]{\hyperref[type:#1]{\PWR{#1}}\marginnote{\scriptsize{p.~\pageref{type:#1}}}}
\newcommand\typerefx[1]{\hyperref[type:#1]{\PWR{#1}}}

\newcommand\defref[1]{\hyperref[define:#1]{\PWR{#1}}\marginnote{\scriptsize{p.~\pageref{define:#1}}}}
\newcommand\defrefx[1]{\hyperref[define:#1]{\PWR{#1}}}

\newcommand\enumref[1]{\hyperref[enum:#1]{\PWR{#1}}\marginnote{\scriptsize{p.~\pageref{enum:#1}}}}
\newcommand\enumrefx[1]{\hyperref[enum:#1]{\PWR{#1}}}

\newcommand\structref[1]{\hyperref[struct:#1]{\PWR{#1}}\marginnote{\scriptsize{p.~\pageref{struct:#1}}}}
\newcommand\structrefx[1]{\hyperref[struct:#1]{\PWR{#1}}}

% #defines with underscores (beyond PWR_) can't be used as labels or references with 
% the underscore package.  These are the special cases.
\newcommand\MAJORVERSIONref{\hyperref[define:MAJORVERSION]{\texttt{PWR\_MAJOR\_VERSION}\marginnote{\scriptsize{p.~\pageref{define:MAJORVERSION}}}}}
\newcommand\MAJORVERSIONrefx{\hyperref[define:MAJORVERSION]{\PWR{MAJOR\_VERSION}}}

\newcommand\MINORVERSIONref{\hyperref[define:MINORVERSION]{\texttt{PWR\_MINOR\_VERSION}\marginnote{\scriptsize{p.~\pageref{define:MINORVERSION}}}}}
\newcommand\MINORVERSIONrefx{\hyperref[define:MINORVERSION]{\PWR{MINOR\_VERSION}}}

\newcommand\MAXSTRINGLENref{\hyperref[define:MAXSTRINGLEN]{\texttt{PWR\_MINOR\_VERSION}\marginnote{\scriptsize{p.~\pageref{define:MAXSTRINGLEN}}}}}
\newcommand\MAXSTRINGLENrefx{\hyperref[define:MAXSTRINGLEN]{\PWR{MAX\_STRING\_LEN}}}

\newcommand\CNTXTDEFAULTref{\hyperref[define:CNTXTDEFAULT]{\texttt{PWR\_CNTXT\_DEFAULT}\marginnote{\scriptsize{p.~\pageref{define:CNTXTDEFAULT}}}}}
\newcommand\CNTXTDEFAULTrefx{\hyperref[define:CNTXDEFAULT]{\PWR{CNTXT\_DEFAULT}}}

\newcommand\CNTXTVENDORref{\hyperref[define:CNTXTVENDOR]{\texttt{PWR\_CNTXT\_VENDOR}\marginnote{\scriptsize{p.~\pageref{define:CNTXTVENDOR}}}}}
\newcommand\CNTXTVENDORrefx{\hyperref[define:CNTXTVENDOR]{\PWR{CNTXT\_VENDOR}}}

%%%%%%%%%%%%%%%%%%%%%%%%%%%%%%%%%%%%%%%%%%%%%%%%%%%%%%%%%%%%%%%%%%%%%%%%%%%%%%%%
%%% The big list of attributes
%%%%%%%%%%%%%%%%%%%%%%%%%%%%%%%%%%%%%%%%%%%%%%%%%%%%%%%%%%%%%%%%%%%%%%%%%%%%%%%%

% The remainder of the macros are for text repeated across several tables.
\newcommand{\aPstateDesc}{%
	\attribute{PstateDesc}		{ \aGS }{ \uint }{ The current P-state for the object specified (typically processors but for use with other component types when applicable).  }%
}
\newcommand{\aCstateDesc}{%		
	\attribute{CstateDesc}		{ \aGS }{ \uint }{ The current C-state for the object specified (typically processors but for use with other component types when applicable).  }%
}
\newcommand{\aCstateLimitDesc}{%
	\attribute{CstateLimitDesc}	{\aGS}{\uint}{The lowest C-state allowed for the object specified (typically processors but for use with other component types when applicable).}%
}
\newcommand{\aSstateDesc}{%
	\attribute{SstateDesc}		{\aGS}{\uint}{The current S-state for the object specified (typically processors but for use with other component types when applicable).}%
}
\newcommand{\aPowerDesc}{%		
	\attribute{PowerDesc}		{\aG }{\dbl }{Discrete power value in watts. The power value should be the value measured as close as possible to the time of the function call.}%
}
\newcommand{\aCurrentDesc}{%
	\attribute{CurrentDesc}		{\aG }{\dbl }{Discrete current value in amps. The current value should be the value measured as close as possible to the time of the function call.}%
}
\newcommand{\aVoltageDesc}{%
	\attribute{VoltageDesc}		{\aG }{\dbl }{Discrete voltage value in volts. The voltage value should be the value measured as close as possible to the time of the function call.}%
}
\newcommand{\aMaxPowerDesc}{%		
	\attribute{MaxPowerDesc}	{\aGS}{\dbl }{Maximum power limit (ceiling, upper bound) for the specified object (as in power cap) in watts.}%
}
\newcommand{\aMinPowerDesc}{%
	\attribute{MinPowerDesc}	{\aGS}{\dbl }{Minimum power limit (floor, lower bound) for the specified object in watts.}%
}
\newcommand{\aFreqLimitMinDesc}{%	
	\attribute{FreqLimitMinDesc}	{\aGS}{\dbl }{Minimum operating frequency limit for the specified object in Hz (cycles per second).}%
}
\newcommand{\aFreqLimitMaxDesc}{%
	\attribute{FreqLimitMaxDesc}	{\aGS}{\dbl }{Maximum operating frequency limit for the specified object in Hz (cycles per second).}%
}
\newcommand{\aFreqDesc}{%
	\attribute{FreqDesc}		{\aGS}{\dbl }{The current operating frequency value for the specified object in Hz (cycles per second).}
}
\newcommand{\aEnergyDesc}{%
	\attribute{EnergyDesc}		{\aG }{\dbl }{The cumulative energy used by the specified object in joules. Note that two attribute get calls are typically required to obtain the energy consumed by the specified object. Subtracting the energy value obtained from the first call from the energy value obtained from the second call produces the energy used for the object from the timestamp of the first value through the timestamp of the second value.}%
}
\newcommand{\aTempDesc}{%
	\attribute{TempDesc}		{\aG }{\dbl }{The current temperature value for the specified object in degrees Celsius.}%
}
\newcommand{\aOSIdDesc}{%	
	\attribute{OSIdDesc}		{\aG }{\dbl }{The operating system ID that corresponds to the object. For example, a runtime system may need to figure out which Power API \texttt{PWR\_OBJ\_CORE} objects correspond to the cores that it is controlling. This attribute provides a linkage between Power API objects and operating system IDs.}%
}
\newcommand{\aThrottledIdDesc}{%		
	\attribute{ThrottledIdDesc}	{\aG }{\dbl }{The cumulative time in nanoseconds that the specified object's performance was purposefully slowed in order to meet some constraint, such as a power cap.  }%
}
\newcommand{\aThrottledCountIdDesc}{%	
	\attribute{ThrottledCountIdDesc}{\aG }{\dbl }{The cumulative count of the number of times that the specified object's performance was purposefully slowed in order to meet some constraint, such as a power cap.}%
}
\newcommand{\aGovDesc}{%
	\attribute{GovDesc}		{\aG }{\dbl }{Power related governor capability exposed through the operating system interface.}
}

%%%%%%%%%%%%%%%%%%%%%%%%%%%%%%%%%%%%%%%%%%%%%%%%%%%%%%%%%%%%%%%%%%%%%%%%%%%%%%%%
%%% The big list of metadata
%%%%%%%%%%%%%%%%%%%%%%%%%%%%%%%%%%%%%%%%%%%%%%%%%%%%%%%%%%%%%%%%%%%%%%%%%%%%%%%%

%\texttt{PWR\_MD\_NUM}                  & Get     & uint64_t   & Number of values supported. This is only relevant for attributes with a discrete set of values (e.g., \texttt{PWR\_ATTR\_PSTATE}). Other attributes return 0. \\
\newcommand{\mNum}{%
	\metadata{num}			{\aG}{\uint}{Number of values supported. This is only relevant for attributes with a discrete set of values (e.g., \texttt{PWR\_ATTR\_PSTATE}). Other attributes return 0.}%
}
%------------------------------------------------------------------------------------------------------------------------------------------------------------------------------------------------------------------------------------------------
%\texttt{PWR\_MD\_MIN}                  & Get     & SaA        & Minimum value supported. \\
\newcommand{\mMin}{%
	\metadata{min}			{\aG}{\SaA}{Minimum value supported.}%
}
%------------------------------------------------------------------------------------------------------------------------------------------------------------------------------------------------------------------------------------------------
%\texttt{PWR\_MD\_MAX}                  & Get     & SaA        & Maximum value supported. \\
\newcommand{\mMax}{%
	\metadata{max}			{\aG}{\SaA}{Maximum value supported.}%
}
%------------------------------------------------------------------------------------------------------------------------------------------------------------------------------------------------------------------------------------------------
%\texttt{PWR\_MD\_PRECISION}            & Get     & uint64_t   & Number of significant digits in values. \\
\newcommand{\mPrecision}{%
	\metadata{precision}		{\aG}{\uint}{Number of significant digits in values.}%
}
%------------------------------------------------------------------------------------------------------------------------------------------------------------------------------------------------------------------------------------------------
%\texttt{PWR\_MD\_ACCURACY}             & Get     & double     & Estimated percent error +/- of measured vs. actual values. \\
\newcommand{\mAccuracy}{%
	\metadata{precision}		{\aG}{\dbl}{Estimated percent error +/- of measured vs. actual values.}%
}
%------------------------------------------------------------------------------------------------------------------------------------------------------------------------------------------------------------------------------------------------
%\texttt{PWR\_MD\_UPDATE\_RATE}         & Set/Get & double     & Rate values become visible to user, in updates per second. Getting or setting a value at a rate higher than this is not useful. \\
\newcommand{\mUpdateRate}{%
	\metadata{update\_rate}		{\aGS}{\dbl}{Rate values become visible to user, in updates per second. Getting or setting a value at a rate higher than this is not useful.}%
}
%------------------------------------------------------------------------------------------------------------------------------------------------------------------------------------------------------------------------------------------------
%\texttt{PWR\_MD\_SAMPLE\_RATE}         & Set/Get & double     & Rate of underlying sampling, in samples per second. This is only relevant for values derived over time (e.g., \texttt{PWR\_ATTR\_ENERGY}). \\
\newcommand{\mSampleRate}{%
	\metadata{sample\_rate}		{\aGS}{\dbl}{Rate of underlying sampling, in samples per second. This is only relevant for values derived over time (e.g., \texttt{PWR\_ATTR\_ENERGY}).}% 
}
%------------------------------------------------------------------------------------------------------------------------------------------------------------------------------------------------------------------------------------------------
%\texttt{PWR\_MD\_TIME\_WINDOW}         & Set/Get & \texttt{PWR\_Time}  & The time window used to calculate the value returned or relevant to an attribute. For example, the ``instantaneous'' \texttt{PWR\_ATTR\_POWER} values reported may actually be averaged over a short time window. Power caps are also enforced with respect to a target time window. \\
\newcommand{\mTimeWindow}{%
	\metadata{time\_window}		{\aGS}{\PWR{Time}}{The time window used to calculate the value returned or relevant to an attribute. For example, the ``instantaneous'' \texttt{PWR\_ATTR\_POWER} values reported may actually be averaged over a short time window. Power caps are also enforced with respect to a target time window.}% 
}
%------------------------------------------------------------------------------------------------------------------------------------------------------------------------------------------------------------------------------------------------
%\texttt{PWR\_MD\_TS\_LATENCY}          & Get     & \texttt{PWR\_Time}  & Estimate of the time required to get or set an attribute. This is useful to estimate completion time for an operation \textit{a priori}. A value of zero should be returned when the get/set is instantaneous.\\
\newcommand{\mTSLatency}{%
	\metadata{ts\_latency}		{\aG}{\PWR{Time}}{Estimate of the time required to get or set an attribute. This is useful to estimate completion time for an operation \textit{a priori}. A value of zero should be returned when the get/set is instantaneous.}%
}
%------------------------------------------------------------------------------------------------------------------------------------------------------------------------------------------------------------------------------------------------
%\texttt{PWR\_MD\_TS\_ACCURACY}         & Get     & \texttt{PWR\_Time}  & Estimated accuracy of returned timestamps, represented as +/- the \texttt{PWR\_Time} value returned. \\
\newcommand{\mTSAccuracy}{%
	\metadata{ts\_accuracy}		{\aG}{\PWR{Time}}{Estimated accuracy of returned timestamps, represented as +/- the \texttt{PWR\_Time} value returned.}% 
}
%------------------------------------------------------------------------------------------------------------------------------------------------------------------------------------------------------------------------------------------------
%\texttt{PWR\_MD\_MAX\_LEN}             & Get     & uint64_t   & The maximum string length that will be returned by the metadata interface. All other string lengths (metadata items ending in ``_LEN'') will be less than or equal to this value.  The value of \texttt{PWR\_MD\_MAX\_LEN} will be less than or equal to \texttt{PWR\_MAX\_STRING\_LEN}. \\
\newcommand{\mMaxLen}{%
	\metadata{max\_len}		{\aG}{\uint}{The maximum string length that will be returned by the metadata interface. All other string lengths (metadata items ending in \texttt{\_LEN}) will be less than or equal to this value.  The value of \texttt{PWR\_MD\_MAX\_LEN} will be less than or equal to \texttt{PWR\_MAX\_STRING\_LEN}.}% 
}
%------------------------------------------------------------------------------------------------------------------------------------------------------------------------------------------------------------------------------------------------
%\texttt{PWR\_MD\_NAME\_LEN}            & Get     & uint64_t   & Length of the attribute name string, in bytes. This is the buffer length needed to store the string returned when \texttt{PWR\_MD\_NAME} is requested. \\
\newcommand{\mNameLen}{%
	\metadata{name\_len}		{\aG}{\uint}{Length of the attribute name string, in bytes. This is the buffer length needed to store the string returned when \texttt{PWR\_MD\_NAME} is requested.}%
}
%------------------------------------------------------------------------------------------------------------------------------------------------------------------------------------------------------------------------------------------------
%\texttt{PWR\_MD\_NAME}                 & Get     & char *     & Attribute name string. This is a C-style NULL-terminated ASCII string. This provides a human readable name for the attribute. The string length is given by \texttt{PWR\_MD\_NAME\_LEN}. \\
\newcommand{\mName}{%
	\metadata{name}			{\aG}{\uint}{Attribute name string. This is a C-style NULL-terminated ASCII string. This provides a human readable name for the attribute. The string length is given by \texttt{PWR\_MD\_NAME\_LEN}.}%
}
%------------------------------------------------------------------------------------------------------------------------------------------------------------------------------------------------------------------------------------------------
%\texttt{PWR\_MD\_DESC\_LEN}            & Get     & uint64_t   & Length of the attribute description string, in bytes. This is the buffer length needed to store the string returned when \texttt{PWR\_MD\_DESC} is requested. \\
\newcommand{\mDescLen}{%
	\metadata{desc\_len}		{\aG}{\uint}{Length of the attribute description string, in bytes. This is the buffer length needed to store the string returned when \texttt{PWR\_MD\_DESC} is requested.}%
}
%------------------------------------------------------------------------------------------------------------------------------------------------------------------------------------------------------------------------------------------------
%\texttt{PWR\_MD\_DESC}                 & Get     & char *     & Attribute description string. This is a C-style NULL-terminated ASCII string. This provides a human readable description of the attribute that is more descriptive than the attribute's name alone. The string length is given by \texttt{PWR\_MD\_DESC\_LEN}. \\
\newcommand{\mDesc}{%
	\metadata{desc}			{\aG}{\ptrchar}{Attribute description string. This is a C-style NULL-terminated ASCII string. This provides a human readable description of the attribute that is more descriptive than the attribute's name alone. The string length is given by \texttt{PWR\_MD\_DESC\_LEN}.}%
}
%------------------------------------------------------------------------------------------------------------------------------------------------------------------------------------------------------------------------------------------------
%\texttt{PWR\_MD\_VALUE\_LEN}           & Get     & uint64_t   & Maximum length of the value strings returned by \texttt{PWR\_MetaValueAtIndex}. This can be used to discover the buffer size that needs to be passed to \texttt{PWR\_MetaValueAtIndex} via the \texttt{value\_str} argument. \\
\newcommand{\mValueLen}{%
	\metadata{value\_len}		{\aG}{\uint}{Maximum length of the value strings returned by \texttt{PWR\_MetaValueAtIndex}. This can be used to discover the buffer size that needs to be passed to \texttt{PWR\_MetaValueAtIndex} via the \texttt{value\_str} argument.}%
}
%------------------------------------------------------------------------------------------------------------------------------------------------------------------------------------------------------------------------------------------------
%\texttt{PWR\_MD\_VENDOR\_INFO\_LEN}    & Get     & uint64_t   & Length of the vendor information string, in bytes. This is the buffer length needed to store the string returned when \texttt{PWR\_MD\_VENDOR\_INFO} is requested. \\
\newcommand{\mVendorInfoLen}{%
	\metadata{vendor\_info\_len}	{\aG}{\uint}{Length of the vendor information string, in bytes. This is the buffer length needed to store the string returned when \texttt{PWR\_MD\_VENDOR\_INFO} is requested.}%
}
%------------------------------------------------------------------------------------------------------------------------------------------------------------------------------------------------------------------------------------------------
%\texttt{PWR\_MD\_VENDOR\_INFO}         & Get     & char *     & Vendor provided information string. This is a C-style NULL-terminated ASCII string. This may be used to convey part numbers, configuration, or other non-standard information. The string length is given by \texttt{PWR\_MD\_VENDOR\_INFO\_LEN}. \\
\newcommand{\mVendorInfo}{%
	\metadata{vendor\_info}		{\aG}{\ptrchar}{Vendor provided information string. This is a C-style NULL-terminated ASCII string. This may be used to convey part numbers, configuration, or other non-standard information. The string length is given by \texttt{PWR\_MD\_VENDOR\_INFO\_LEN}.}% 
}
%------------------------------------------------------------------------------------------------------------------------------------------------------------------------------------------------------------------------------------------------
%\texttt{PWR\_MD\_MEASURE\_METHOD}      & Get     & uint64_t   & Denotes the measurement method: an actual measurement (returned value = 0) or a model based estimate (return value = 1). Other values $> 1$ may be used to denote multiple vendor specific models in the situation where multiple models may exist. \\
\newcommand{\mMeasureMethod}{%
	\metadata{measure\_method}	{\aG}{\ptrchar}{Denotes the measurement method: an actual measurement (returned value = 0) or a model based estimate (return value = 1). Other values $> 1$ may be used to denote multiple vendor specific models in the situation where multiple models may exist.}%
}
%------------------------------------------------------------------------------------------------------------------------------------------------------------------------------------------------------------------------------------------------

% End Macro definitions
   
\newcommand{\majorversion}{2}
\newcommand{\minorversion}{0}
%
% Set the title, and author
%
\title{High Performance Computing - Power Application Programming Interface Specification \\ Version \majorversion.\minorversion }

\author{Chair: Ryan E. Grant\Mark{1} 
\\ Editor: Barry Rountree\Mark{2} 
\\ Secretary: Jeff Hanson\Mark{3} 
\\ \\ Contributors:  
\and Chris Cantalupo\Mark{4}, Jonathan Eastep\Mark{4}, 
\and Scott Hara\Mark{5}, Siddhartha Jana\Mark{4},\\ 
\and Jaymin Jasoliya\Mark{4}, Matthew Kappel\Mark{6}, \\ 
\and James H. Laros III\Mark{1}, Steve Leak \Mark{7}, \\ 
\and Michael Levenhagen\Mark{1}, Steve Martin\Mark{6}, \\
\and Ramakumar Nagappan\Mark{4}, Kevin Pedretti\Mark{1}, \\
\and Todd Rosedahl\Mark{8}, Andy Warner\Mark{3}, Andrew Younge\Mark{1}} 

%
\date{June 2018}
% ---------------------------------------------------------------------------- %
%
% Start the document
%
\begin{document}
\maketitle

\noindent Contributing Organizations: \\
\Mark{1} Sandia National Laboratories \\
\Mark{2} Lawrence Livermore National Laboratory \\
\Mark{3} Hewlett Packard Enterprise (HPE) \\
\Mark{4} Intel \\
\Mark{5} Qualcomm \\
\Mark{6} Cray \\
\Mark{7} NERSC \\
\Mark{8} IBM \\


\begin{abstract}
Measuring and controlling the power and energy consumption of high performance computing systems by various components in the software stack is an active research area 
~\cite{6604474,6604496,
5488433,
Liu:2012:RCA:2318857.2254779,
Chen:2013:DSP:2561828.2561853,
Yang:2013:IDP:2503210.2503264,
Wallace:2013:CLUSTER:2503210.2503264,
Shoukourian2013,
conf:icdcn:GeorgiouCGAJH14,
Trader:2013:report:GreenComputing,
6337489,
Vishnu:2013:JSC:s11227-011-0699-9,
Mills:2013:EES:2536430.2536432,
Bertran:2013:IBM:2012.2227580,
Georgiou:2013:Bull,
6604481,
6604508}.
Implementations in lower level software layers are beginning to emerge in some production systems, which is very welcome.
To be most effective, a portable interface to measurement and control features would significantly facilitate participation by all levels of the software stack.
We present a proposal for a standard power Application Programming Interface (API) that endeavors to cover the entire software space, from generic hardware interfaces to the input from the computer facility manager.

    \end{abstract}
    \chapter{Acknowledgment}

The Power API Community Specification is managed via the Power API Committee, an open specifications body operating under the Energy-Efficient High Performance Working Group (EEHPC-WG).
The community version of the specification was developed based on the Power API Specification, originally developed at Sandia National Laboartories and supported through the Advanced Simulation and Computing (ASC) program funded by U.S. Department of Energy's National Nuclear Security Agency.

The Sandia National Laboratories version of the specification was retired to support the community-lead version. The original specification can be found at \texttt{powerapi.sandia.gov}. 

The original publication describing the design and operation of the Power API~\cite{grant2016standardizing} is: Grant, R.E., Levenhagen, M., Olivier, S.L., DeBonis, D., Pedretti, K.T. and Laros III, J.H., 2016. Standardizing power monitoring and control at exascale. Computer, 49(10), pp.38-46. 

We wish to thank our colleagues, Steve Hammond, Ryan Elmore, and Kris Munch at the National Renewable Energy Laboratory (NREL) for their contributions to the use case model which was the progenitor of this work.
This effort was greatly enhanced by interactions with staff throughout Sandia as well as many external organizations. 

The addition of the Python language bindings in version 2.0 of the Power API specification would not have been possible without contributions from Steve Martin (Cray), Matthew Kappel (Cray) and Leo Maurer (Cray), Paul Falde (Cray) and valuable feedback from Johnathan Woodring (Los Alamos National Laboratory)

Feedback and additions to the application hints interface were provided by Chris Cantalupo and Steve Sylvester of Intel.

The following individuals contributed to the specification during the version 1.X series: Sue Kelly (Sandia National Laboratories) and David DeBonis (Sandia National Laboratories). 

Prior to the first open release of this specification a select group of individuals agreed to review an early draft of the specification and provide feedback. 
We would like to recognize the very significant contributions these individuals made and thank them for their time and efforts. 
The following individuals participated in an all day face-to-face review of the specification and provided written feedback (listed in alphabetical order): David Jackson (Adaptive Computing), Steve Martin (Cray), Indrani Paul (AMD), Phil Pokorny (Penguin Computing), Avi Purkayastha (National Renewable Energy Laboratory), Muralidhar Rajappa (Intel), and Jeff Stuecheli (IBM).
The following individuals provided written feedback of the specification (listed in alphabetical order): Dorian Arnold (University of New Mexico), Natalie Bates (EEHPC), and Chung-Hsing Hsu (Oak Ridge National Laboratory).
We hope to continue these important collaborations and develop new ones in an effort to represent and serve the HPC community as best we can.



	\tableofcontents

	\chapter{Introduction}						%PWR-free
	\input{Introduction}

	\chapter{Theory of Operation}\label{chap:Theory}
	
%=============================================================================%
%=============================================================================%
%=============================================================================%
%=============================================================================%
%=============================================================================%
%=============================================================================%
%=============================================================================%
\section{Overview}
This section discusses many of the foundational concepts leveraged throughout the Power API specification.
It should be noted that many terms commonly used when discussing object oriented languages are used in this section and the document as a whole.
The use of these terms in no way implies that the Power API specification must be implemented using an object oriented language.
%The authors of this specification in fact, recommend the contrary.
We have attempted to achieve two goals, listed in order of priority: 1) programmer portability, where the programmer is the user of the API, and 2) the latitude of the implementor who will often become the user of the API benefiting from our first priority. 

%=============================================================================%
%=============================================================================%
%=============================================================================%
%=============================================================================%
%=============================================================================%
%=============================================================================%
%=============================================================================%
\section{Power API Initialization}\label{sec:PowerAPIInit}


Using any of the Power API interfaces requires initialization. 
Initialization returns a context.
In the specification, the context is defined as an opaque pointer.
This approach was taken to allow the maximum amount of flexibility to the implementor.
The context returned will contain (act as the entry point to) the system description that is exposed to the user, all policy and privilege information, basically everything the user of the API requires to perform the functionality specified by the API.
The system description is not required to be changed or updated during the life of a specific context.
Initialization is accomplished by calling \funcref{CntxtInit}.
Resources created, like groups, by the user during the life of the context should be cleaned up (destroyed) by the user when no longer needed. 
The implementation is required to clean up all context resources when the user calls \funcref{CntxtDestroy}.

%=============================================================================%
%=============================================================================%
%=============================================================================%
%=============================================================================%
%=============================================================================%
%=============================================================================%
%=============================================================================%
\section{Roles}\label{sec:Roles}

The Power API specification leverages the concept of Roles. 
Roles represent the different types of users that exist which include:
\begin{itemize}[noitemsep,nolistsep] %
\item{\textbf{Application}  The application or application library executing on the compute resource. May also include run-time components running in user space.}
\item{\textbf{Monitor and Control}  Cluster management or Reliability Availability and Serviceability (RAS) systems, for example.}
\item{\textbf{Operating System} Linux or specialized Light Weight Kernels which are found on HPC platforms and potentially portions of run-time systems. }
\item{\textbf{User} The user of the HPC facility, typically using one platform but potentially using multiple platforms. }
\item{\textbf{Resource Manager} This can include work load managers, schedulers, allocators and even portions of run-time systems. }
\item{\textbf{Administrator} The system administrator or HPC facility/platform manager. }
\item{\textbf{HPCS Manager} The individual or individuals responsible for managing policy for the HPC facility and platforms, for example. }
\item{\textbf{Accounting} Individual or software that produces reports of metrics for the HPC facility and associated platforms. }
\end{itemize}
These brief definitions are not meant to be exhaustive.
Roles are analogous with the \textit{Actors} discussed in section \ref{sec:UseCase}.
In some cases roles become the system that other roles interact with.
For example, we specify an interface between the Application role (HPCS Application in figure \ref{fig:UCDTopLevel}) and the Operating System (HPCS Operating System in figure \ref{fig:UCDTopLevel}).
The Operating System is the system (in UML terminology) that the Application role is interacting with. 
Notice in figure \ref{fig:UCDTopLevel} that the specification also includes an interface between the Operating System role and the Hardware (HPCS Hardware in figure \ref{fig:UCDTopLevel}).
These and other interfaces are described in chapter \ref{chap:Interfaces}.
The user of the API is required to specify what role they will assume when interacting with the system upon initialization of the API.

%The implementation is required to associate an integer precedence with each role, zero (0) being the highest precedence.
%Roles may have the same precedence number which indicates the roles have equal precedence.
%Other factors, such as user name and permissions can be used to make precedence determinations at finer granularity.
%This feature is provided as a mechanism for the implementation to determine which operation has priority especially in cases where operations conflict.

%For example, the Administrator role (assigned precedence 0) sets a power cap for node 0 at 200W. 
%The Resource Manager role (assigned precedence 1) would like to manage the power cap for an application which uses a number of nodes including node 0.
%The Application role (assigned precedence 2) would like to fine tune the power cap of one of the nodes it is executing on (node 0). 
%The implementation can use the precedence numbers in a variety of ways given the assumptions above.
%In this scenario, the Administrator role is allowed to set the power cap for node 0 at 200W.
%If the Resource Manager requests a power cap of 210W for node 0 the operation would be denied given a role with higher precedence requested a lower power cap.
%If the Resource Manager requests a power cap of 180W for node 0, the operation is allowed.
%Likewise the Application role is permitted to request a lower power cap for node 0 but not a higher power cap.
%First note, the specification does not proscribe these rules, it only provides the precedence mechanism as a way to implement them.
%Also note, this example could be accomplished by simply determining precedence based on which role is requesting the operation.
%The precedence concept is provided in cases where, for example, the implementation requires two roles to have equal precedence.
%If we add to the above scenario the Operating System role (precedence 0), the Operating System has the ability to request that the power cap for node 0 be raised to 210W.
%Once more the specification is not proscribing this rule, only illustrating the need for an additional mechanism that can be used by the implementation to enforce similar scenarios.
%
%\laros{Add a reference here for the call that can be used to get the implementation assigned precedence numbers for each role.}

Roles are also provided as a mechanism for the implementation to express priority or precedence in circumstances where, for example, conflicting operations are requested. 
%More detail on this topic is provided in section \ref{sec:Initialization} (Initialization).


%=============================================================================%
%=============================================================================%
%=============================================================================%
%=============================================================================%
%=============================================================================%
%=============================================================================%
%=============================================================================%
\section{System Description}\label{sec:PowerAPIBaseSysDesc}

\begin{figure}
	\begin{center}
		\includegraphics[width=0.80\linewidth,height=.60\paperheight]{FIGURES/PowerAPIMachineHierarchy5dot1}
	\end{center}
	\caption{Hierarchical Depiction of System Objects}
	\label{fig:BaseSystemMap}
\end{figure}

The system description is the \textit{view} of the system exposed to the user upon initialization via the context that is returned.
Figure \ref{fig:BaseSystemMap} depicts an example of a system description showing a hierarchical arrangement of objects where there is only one platform.
All object types listed in the specification must be defined by any implementation, but do not have to be used in the system description.
The implementation chooses which objects will be employed in the system description and how they will be arranged.
An object can only have a single parent but may have multiple children.
A system description may describe more than a single platform as a facility object where the facility object has multiple objects of type \texttt{Platform} and the facility represents the top of the hierarchy. 
Due to the large number of variations in software, hardware and security profiles between platforms and the difficulty in providing a universal Power API implementation that would work between any and all platforms, control over platform objects is confined to within that platform object. 
Power API contexts initialized below the facility level (platform or lower in the system hierarchy) can only read attributes at the facility level, not modify them.
For example, a compute node in a given platform cannot query or modify the CPU frequency of a object on another platform. 
However, that compute node could query (it can only read attributes, not modify them at the facility level) the facility object to inquire about the overall facility power usage if security policies allow such queries. 
A user with a role initialized at the facility level is meant to interact with high-level platform objects, it is not intended for low-level object control (i.e. controlling per socket power budgets). Such low level controls should be done with roles initialized at the platform level.
While figure \ref{fig:BaseSystemMap} depicts a homogeneous system description, homogeneity is \textit{not} a requirement. 
In practice a system description can be heterogeneous and unbalanced.

To summarize the requirements:
\begin{itemize}[noitemsep,nolistsep] %
	\item{
	At least one \texttt{Platform} object type must be defined by the implementation and may appear at the top of the system description or immediately underneath a facility object.
	}
	\item{
	All object types in this specification must be defined in any implementation. The use of the object types, with the exception of the \texttt{Platform} object type, is optional.
	}
	\item{
	Objects can have at most one parent but may have many children. Currently the \texttt{Facility} object has no parent as it represents the top of the system description.
	}
        \item{
        If an implementation chooses to add objects not defined in the specification they should only be exposed to the user in a vendor specific context to avoid unpredictable or non-portable behaviour (see \funcref{CntxtInit}).
        }
\end{itemize}

%Figure \ref{fig:BaseSystemMap} depicts a hierarchical arrangement of all of the  platform objects that must be included (defined) in any implementation of the API.
%In addition to the requirement that these object types must be defined in any implementation, they also must be organized in the order shown in figure \ref{fig:BaseSystemMap}.
%For example, an implementation must always position objects of type \textit{Cabinet} under the top level \textit{Platform} object.
%Likewise, objects of type \textit{Board} must appear below an object of type \textit{Cabinet}.
%There can only be a single object of type \textit{Platform}, and that object must be the top of the hierarchy.

The following is a list of the object types currently included in the specification along with a short description of each.
\begin{itemize}[noitemsep,nolistsep] %
	\item{
Facility - The Facility object is the highest level object allowed in a system description. 
It represents an entire computing facility that may be made up of multiple platforms.
Facility objects are useful for including resources like cooling or power delivery that may be shared by multiple platforms.
Facility objects are not necessarily traversible between users on different platforms. 
That is, users may not be able to interact or even discover other platforms within the same facility.
} 
        \item{
Platform - A Platform object is the top level object of the system description exposed to a typical user of the API.
It represents one of the main system divisions in a given compute facility and may be the highest level object in facilities that have only one platform. 
The Platform object is intended to conceptually represent the entire Platform.
For example, if the Platform object has a power or energy measurement or control capability exposed through the Platform objects attributes the scope of these attributes should be platform wide.
}
	\item{
CDU - A Cooling Distribution Unit object represents cooling infrastructure that is part of a facility associated with a platform. The CDU object is intended to be used
for liquid cooling controls and measurements. The CDU object is not strictly between the platform and cabinet objects in the object hierarchy. They can be placed wherever
makes sense for a given cooling solution for a platform.
}
	\item{
Cabinet - Objects of type Cabinet are intended to represent the cabinets or racks that act as enclosures (or logical groupings) for the platform equipment. 
Beyond the utility of convenient groups of lower level objects (equipment) cabinets may have power or energy relevant capabilities which can be exposed through attributes associated with each Cabinet object. 
}
	\item{
Chassis - Objects of type Chassis are intended to be used for finer grained organization of objects within the higher level Cabinet object. Chassis, like cabinets may have power or energy relevant capabilities that can be exposed to the user.
}
	\item{
Board - Board objects offer another method of organization for underlying objects (equipment). 
Boards may also have power and or energy relevant capabilities which can be exposed through associated attributes. 
For example, a board could contain the power supply and the point of instrumentation for collecting power or energy samples for a node or multiple nodes.
}
	\item{
Node - The Node type is probably one of the most universally important object types. 
Measuring and controlling the power and or energy characteristics of a node or multiple nodes (grouped into multiple Boards, Chassis or Cabinets) is important for a many reasons and provides a wide range of flexibility of configuration to the implementor. 
For example, on HPC platforms a single application typically executes on many nodes. 
Understanding the energy use of an application run can be obtained by collecting the energy use (via the appropriate Node attribute) for each node participating in that application execution. 
Node objects will likely have many attributes exposing many power and energy relevant capabilities.
}
	\item{
Socket - The Socket object is intended to represent the one or more processor sockets, or other component types that can be thought of as sockets, that make up a Node. 
For example, a single Node object may be a dual socket (dual CPU) node.
The implementor may choose to enclose other component types (a NIC for example) within a Socket object, or add other object types as they see fit to represent the architecture they are describing.
They can also decide to omit the use of this, or any other object type (currently other than Platform) in the system description.
}
	\item{
Power Plane - The Power Plane object is used to organize lower level objects (any types of objects) within a power domain or single point of measurement and or control.
For example, a pair of cores may share a power plane within a socket. 
This configuration is depicted in figure \ref{fig:BaseSystemMap}. 
This organization allows a pair of cores to be controlled from a single power control point in the hierarchy for convenience. 
This object type allows these power and energy relevant relationships to be expressed anywhere in the system description.
}
	\item{
Core - Core objects are intended to represent the individual processor cores within multi-core CPUs (or possibly GPUs). 
Modern architectures have an increasing number of cores per CPU (or GPU). 
In the near future it is likely that an abstraction between Socket and core would become useful as the number of cores increase. 
Physical and logical groupings of cores already exist in current architectures.
}
	\item{Memory - The Memory object type is included to represent the growing range of memory types that exist on HPC platforms. 
Individual cores, for example, have Memory in the form of cache which the implementor may choose to organize differently from the main memory of the Node or a tertiary level of memory such as NVRAM.
}	
	\item{
NIC - The NIC object is intended to represent the Network Interface Controller.
As with many other object types, the organization of a NIC in relation to Boards, Nodes or even Cores is architecture dependent.
The NIC object type is included in hopes that there are power and energy relevant capabilities included in future NICs.
}	
        \item{
HT - The HT (Hardware Thread) object represents an OS-visible CPU.  
While from a physical perspective frequency and voltage changes occur at the physical core level, it is usually the case that these must be configured by software at the OS-visible CPU level.  
Typically the lowest-common denominator among all OS-visible CPUs is used to configure the physical core.
}

\end{itemize}

Additional object types may be defined by the implementor and placed anywhere in the hierarchy as long as the previously stated rules are not violated.
Ultimately, the  object types defined in this specification, and those added by the implementor, will be used to produce a system description describing the system presented to the user via the context returned upon initialization.
Objects are used as interfaces to underlying functionality.
The specification does not assume state is retained for objects.
Additionally, the specification makes no guarantees with regards to race conditions between processes or threads.

%=============================================================================%
%=============================================================================%
%=============================================================================%
%=============================================================================%
%=============================================================================%
%=============================================================================%
%=============================================================================%
\section{Attributes}\label{sec:TheoryAttributes}
Attributes are an important part of the Power API.
A large amount of basic functionality is exposed through the use of attributes.
The term attribute is used somewhat conceptually since some attributes are implicit while others are explicitly defined as part of a required specification data structure (page \pageref{type:AttrName}).
Attributes are used for a number of reasons such as to navigate through the system description, to access information or a measurement (sensor information for example) and for control (setting a P-state for example).
Global attributes are attributes that are present for every object defined; whether required by the specification or added by the implementor. 

The following is the list of global attributes:
\begin{itemize}[noitemsep,nolistsep] %
\item{name} - Unique identifying name of the object (see \funcref{ObjGetName}).
\item{entry point} - The position in the hierarchy after initialization (see \funcref{CntxtGetEntryPoint}).
\item{type} - The type of the object (see \funcref{ObjGetType}).
\item{parent} - The parent of an object is the object that is above it in the hierarchy (see \funcref{ObjGetParent}).  The only exception is the currently single platform object whose parent is a pointer to NULL. 
\item{children} -  Object or objects directly below an object in the hierarchy (see \funcref{ObjGetChildren}).
\end{itemize}

Note, in the list above all the attributes are implicit. 
Explicit attributes are defined in the \typeref{AttrName} type definition.
The majority of the attributes defined in the specification, and likely those added by an implementor, are, and will be, explicit.
The implicit attributes defined above are primarily used for navigation and are accessed through attribute specific functions which are described in Section \ref{sec:Navigation}.

Explicit attributes are either accessed through the generic attribute interface (Section \ref{sec:Attributes}) or attribute specific functions found in either the section describing the specific interface in which they are used or in Chapter \ref{chap:Common}, \textit{Core (Common) Interface Functions}.

The attribute interface is intended to keep the specification from growing every time additional functionality is either specified or added by an implementor. 
As long as the new functionality fits within the defined attribute interfaces no additional API functions are required to be specified.

%=============================================================================%
%=============================================================================%
%=============================================================================%
%=============================================================================%
%=============================================================================%
%=============================================================================%
%=============================================================================%
\section{Metadata}\label{sec:TheoryMetadata}
Each object and object attribute pair can have additional descriptive metadata associated with it.
This information is often useful for getting a better understanding of the meaning of objects and attributes and how to interpret the values read from attributes.
Examples include a human readable name and description strings, the list of values supported by an attribute, and measurement accuracy and precision.
The metadata interface (see section \ref{sec:METADATA}) returns information relevant to either a specific object or a specific attribute of a specific object.
A given attribute name may have different metadata for different objects, even if the objects are of the same type (e.g., the voltage attribute of two node objects may have different metadata accuracy values).


\section{Thread Safety}\label{sec:ThreadSafety}
Implementations of the Power API are not required to provide thread safety to multiple threads of the same process.  
If necessary, users of the Power API must use locking or some other mechanism to ensure that only one thread per process calls into the Power API at a time.  
This requirement only applies to threads of the same process that may issue conflicting operations.  
Different processes may make simultaneous Power API calls without any coordination.  
If thread concurrency within a process is required, the \funcref{CntxtInit} function can be called multiple times to initialize multiple Power API contexts.  
Multiple threads of the same process may then simultaneously call into the Power API, so long as each thread operates on a different Power API context.  
For example, a process with four threads may create four Power API contexts and associate one context with each thread.  
The threads may then make Power API calls without any additional coordination, so long as each thread operates only on its assigned context and the objects exposed by its assigned context. 
Threads should not operate on objects exposed by another thread's context without employing locking or some other coordination mechanism.



	\chapter{Type Definitions}\label{chap:TypeDefinitions}
	%==============================================================================%
%==============================================================================%
%==============================================================================%
%==============================================================================%
%==============================================================================%

\section{Opaque Types}\label{sec:OpaqueTypes}

The following type definitions are specified to be opaque pointers from the point of view of Power API users.
Power API implementations will typically map these pointers to internal implementation-specific state.
The reason for using opaque pointers is to hide non-portable implementation details from users and give implementors of the API maximum flexibility.

\begin{typedefs}
	\typedef{void*}{Cntxt}
	\typedef{void*}{Grp}
	\typedef{void*}{Obj}
	\typedef{void*}{Status}
	\typedef{void*}{Stat}
\end{typedefs}

%==============================================================================%
%==============================================================================%
%==============================================================================%
%==============================================================================% 
%==============================================================================%

\section{Globally Relevant Definitions}\label{sec:GlobalTypes}

The following definitions are specified on a global basis.
The \MAJORVERSIONrefx and \defrefx{MINORVERSION} definitions are compile time constants that indicate the Power API version supported by the implementation.
The \defrefx{MAXSTRINGLEN} definition is a compile time constant that defines the maximum length of strings that can be returned from Power API calls, with the actual value being a vendor specific length.

\begin{typedefs}
	\pounddefineMAJORVERSION
	\pounddefineMINORVERSION
	\pounddefineMAXSTRINGLEN
\end{typedefs}

%==============================================================================%
%==============================================================================%
%==============================================================================%
%==============================================================================%
%==============================================================================%

\section{Context Relevant Type Definitions}\label{sec:ContextTypeDefinitions}

The \typeref{CntxtType} and \typerefx{Role} types are required to be defined by all implementations of the Power API.
When a new Power API context is created, one value from each of these types is used to determine the kind of context created (see section \ref{sec:Initialization}).
For \typerefx{CntxtType}, the only required value that an implementation must define is \CNTXTDEFAULTrefx.
This indicates that the new context will only contain Power API functionality that is explicitly defined in the specification, with no implementation-specific extentions present.
Implementors may extend \typerefx{CntxtType} with additional values, such as \CNTXTVENDORrefx, to provide contexts with additional functionality.

We anticipate that most implementations of the Power API will define additional \typerefx{CntxtType} values that provide additional functionality, such as vendor, platform, or model specific extentions.
If an implementation extends the specification, the extensions should only be visible to the user when they use a context that was created with an implementation-specific \typerefx{CntxtType} value.
If the implementation-specific extensions are not available to the user, initialization using an implementation-specific \typerefx{CntxtType} value should result in failure.
The user must always be able to initialize a context using \CNTXTDEFAULTref to to get a context containing only the standard specification features.

Differentiation between context types is the mechanism used by the Power API to enable extended vendor, platform or model specific capabilities while, at the same time, allowing portability for applications or tools that only leverage standard specification features. 
For example, a tool that leverages only the object and attribute types defined in the standard specification can initialize a Power API context using \CNTXTDEFAULTrefx and not have to worry about dealing with any implementation-specific functionality.
The context it receives will only provide functionality that is explicitly defined by the Power API specification.

\typeref{Role} is used to specify the role that the user is acting in when they initialize a new context.
Additional roles may not be added by the implementor.
Notice that there is a role defined for every actor in Chapter \ref{chap:Interfaces} - Role/Systems Interfaces.
We intend that the user's role will serve many purposes, such as determining the view of the system that is provided within the context when combined with the system the user is acting on. 
Roles can also be used to help determine the privilege of the user's context for purposes such as resolving the precedence of conflicting operations.


\subsubsection{PWR_CntxtType}\label{type:CntxtType}
\begin{typedefs}
	\typedef{int}{CntxtType}
	\pounddefineCNTXTDEFAULT
	\pounddefineCNTXTVENDOR
\end{typedefs}


%==============================================================================%

\subsubsection{PWR_Role}\label{type:Role}
\begin{center}
\begin{minipage}{.95\linewidth}%
\begin{lstlisting}
typedef enum {
        PWR_ROLE_APP = 0,   /* Application */
        PWR_ROLE_MC,        /* Monitor and Control */
        PWR_ROLE_OS,        /* Operating System */
        PWR_ROLE_USER,      /* User */
        PWR_ROLE_RM,        /* Resource Manager */
        PWR_ROLE_ADMIN,     /* Administrator */
        PWR_ROLE_MGR,       /* HPCS Manager */
        PWR_ROLE_ACC,       /* Accounting */
        PWR_NUM_ROLES,
        /* */
        PWR_ROLE_INVALID       = -1,
        PWR_ROLE_NOT_SPECIFIED = -2
} PWR_Role;
\end{lstlisting}
\end{minipage}
\end{center}

%==============================================================================%
%==============================================================================%
%==============================================================================%
%==============================================================================%
%==============================================================================%


\section{Object Relevant Type Definitions}\label{sec:ObjectTypeDefinitions}

The \typeref{ObjType} type is required to be defined by all implementations of the Power API specification. 
Objects with types defined by \typerefx{ObjType} are used by the implementor to create the system description (see section \ref{sec:PowerAPIBaseSysDesc}) that is exposed to the user upon initialization.
An implementation may extend this type by adding new object enumeration type, which must be added prior to \texttt{PWR_NUM_OBJ_TYPES}.
The added implementation-specific object types will only be used by implementation-specific contexts (see section \ref{sec:ContextTypeDefinitions}).
Contexts that were initialized using the default context, \texttt{PWR_CNTXT_DEFAULT}, will only expose objects types defined in the list below.

%==============================================================================%

\subsubsection{PWR_ObjType}\label{type:ObjType}
\begin{center}
\begin{minipage}{.95\linewidth}%
\begin{lstlisting}
typedef enum {
        PWR_OBJ_PLATFORM = 0,
        PWR_OBJ_CABINET,
        PWR_OBJ_CHASSIS,
        PWR_OBJ_BOARD,
        PWR_OBJ_NODE,
        PWR_OBJ_SOCKET,
        PWR_OBJ_CORE,
        PWR_OBJ_POWER_PLANE,
        PWR_OBJ_MEM,
        PWR_OBJ_NIC,
        PWR_OBJ_HT,
	PWR_OBJ_CDU,
        PWR_NUM_OBJ_TYPES,
        /* */
        PWR_OBJ_INVALID       = -1,
        PWR_OBJ_NOT_SPECIFIED = -2
} PWR_ObjType;
\end{lstlisting}
\end{minipage}
\end{center}



%==============================================================================%
%==============================================================================%
%==============================================================================%
%==============================================================================%
%==============================================================================%

\section{Attribute Relevant Type Definitions}\label{sec:AttributeTypeDefinitions}

The \typeref{AttrName} and \typerefx{AttrDataType} types are required to be implemented. 
Both may be extended by the implementor and exposed using an implementation specified context type (see section \ref{sec:ContextTypeDefinitions}).
If new \typerefx{AttrName} entries are added it is required that the attribute name is specified and commented as shown in the \typerefx{AttrName} structure.
Likewise, new types must be added to the \typerefx{AttrDataType} structure. 
It's important to note that the attribute interface currently supports only numeric types.
Attributes should only be added to this definition if they can be meaningfully supported by the attribute interface (section \ref{sec:Attributes}).
Additional attributes must be added prior to  \texttt{PWR_NUM_ATTR_NAMES}.
The Attributes in \typerefx{AttrName} expose what we consider foundational measurement and control interfaces. 
Additional capabilities are and can be added using additional operations and often interface specific functions.

The \typeref{AttrAccessError} type is used to hold the error returns that are popped from the \texttt{PWR_Status} handle (see section \ref{sec:OpaqueTypes}) using the \funcref{StatusPopError} function.

%==============================================================================%

\subsubsection{PWR_AttrName}\label{type:AttrName}
\begin{center}
\begin{minipage}{.95\linewidth}%
\begin{lstlisting}
typedef enum {
        PWR_ATTR_PSTATE = 0,         /* uint64_t */
        PWR_ATTR_CSTATE,             /* uint64_t */
        PWR_ATTR_CSTATE_LIMIT,       /* uint64_t */
        PWR_ATTR_SSTATE,             /* uint64_t */
        PWR_ATTR_CURRENT,            /* double, amps */
        PWR_ATTR_VOLTAGE,            /* double, volts */
        PWR_ATTR_POWER,              /* double, watts */
        PWR_ATTR_POWER_LIMIT_MIN,    /* double, watts */
        PWR_ATTR_POWER_LIMIT_MAX,    /* double, watts */
        PWR_ATTR_FREQ,               /* double, Hz */
        PWR_ATTR_FREQ_LIMIT_MIN,     /* double, Hz */
        PWR_ATTR_FREQ_LIMIT_MAX,     /* double, Hz */
        PWR_ATTR_ENERGY,             /* double, joules */
        PWR_ATTR_TEMP,               /* double, degrees Celsius */
        PWR_ATTR_OS_ID,              /* uint64_t */
        PWR_ATTR_THROTTLED_TIME,     /* uint64_t */
        PWR_ATTR_THROTTLED_COUNT,    /* uint64_t */
        PWR_ATTR_GOV,                /* uint64_t */
        PWR_NUM_ATTR_NAMES,
        /* */
        PWR_ATTR_INVALID       = -1,
        PWR_ATTR_NOT_SPECIFIED = -2
} PWR_AttrName;
\end{lstlisting}
\end{minipage}
\end{center}

%==============================================================================%

\subsubsection{PWR_AttrDataType}\label{type:AttrDataType}
\begin{center}
\begin{minipage}{.95\linewidth}%
\begin{lstlisting}
typedef enum {
        PWR_ATTR_DATA_DOUBLE = 0,
        PWR_ATTR_DATA_UINT64,
        PWR_NUM_ATTR_DATA_TYPES,
        /* */
        PWR_ATTR_DATA_INVALID       = -1,
        PWR_ATTR_DATA_NOT_SPECIFIED = -2
} PWR_AttrDataType;
\end{lstlisting}
\end{minipage}
\end{center}

%==============================================================================%

\subsubsection{PWR_AttrAccessError}\label{type:AttrAccessError}
\begin{center}
\begin{minipage}{.95\linewidth}%
\begin{lstlisting}
typedef struct {
        PWR_Obj      obj;    /* The object associated with the error */
        PWR_AttrName attr;   /* The attribute associated with the error */
        int          index;  /* The index in the output array where the error occurred */
        int          error;  /* The error code, see Error Return Definitions section */
} PWR_AttrAccessError;
\end{lstlisting}
\end{minipage}
\end{center}

%==============================================================================%

\subsubsection{PWR_AttrGov}\label{type:AttrGov}
\begin{center}
\begin{minipage}{.95\linewidth}%
\begin{lstlisting}
typedef enum {
        PWR_GOV_LINUX_ONDEMAND,
        PWR_GOV_LINUX_PERFORMANCE,
        PWR_GOV_LINUX_CONSERVATIVE,
        PWR_GOV_LINUX_POWERSAVE,
        PWR_GOV_LINUX_USERSPACE
} PWR_AttrGov;
\end{lstlisting}
\end{minipage}
\end{center}
%==============================================================================%
%==============================================================================%
%==============================================================================%
%==============================================================================%
%==============================================================================%


\section{Metadata Relevant Type Definitions}\label{sec:MetadataTypeDefinitions}

The \texttt{PWR_MetaName} type is required to be implemented. 
The type may be extended by the implementor and the additional capabilities may be exposed using an implementation specified context type (see section \ref{sec:ContextTypeDefinitions}).
If new \texttt{PWR_MetaName} items are added, it is required that the metadata name be specified and commented as shown in the \texttt{PWR_MetaName} definition. 
Additional metadata items must be added prior to \texttt{PWR_NUM_META_NAMES}.


%==============================================================================%

\subsubsection{PWR_MetaName}\label{type:MetaName}
\begin{center}
\begin{minipage}{.95\linewidth}%
\begin{lstlisting}
typedef enum {
        PWR_MD_NUM = 0,           /* uint64_t */
        PWR_MD_MIN,               /* either uint64_t or double, depending on attribute type */
        PWR_MD_MAX,               /* either uint64_t or double, depending on attribute type */
        PWR_MD_PRECISION,         /* uint64_t */
        PWR_MD_ACCURACY,          /* double */
        PWR_MD_UPDATE_RATE,       /* double */
        PWR_MD_SAMPLE_RATE,       /* double */
        PWR_MD_TIME_WINDOW,       /* PWR_Time */
        PWR_MD_TS_LATENCY,        /* PWR_Time */
        PWR_MD_TS_ACCURACY,       /* PWR_Time */
        PWR_MD_MAX_LEN,           /* uint64_t, max strlen of any returned metadata string. */
        PWR_MD_NAME_LEN,          /* uint64_t, max strlen of PWR_MD_NAME */
        PWR_MD_NAME,              /* char *, C-style NULL-terminated ASCII string */
        PWR_MD_DESC_LEN,          /* uint64_t, max strlen of PWR_MD_DESC */
        PWR_MD_DESC,              /* char *, C-style NULL-terminated ASCII string */
        PWR_MD_VALUE_LEN,         /* uint64_t, max strlen returned by PWR_MetaValueAtIndex */
        PWR_MD_VENDOR_INFO_LEN,   /* uint64_t, max strlen of PWR_MD_VENDOR_INFO */
        PWR_MD_VENDOR_INFO,       /* char *, C-style NULL-terminated ASCII string */
        PWR_MD_MEASURE_METHOD,    /* uint64_t, 0/1 depending on real/model mesurement */
        PWR_MD_LIQUID_RES_CAP,    /* uint64_t */
        PWR_MD_LIQUID_TYPE,	  /* char *, C-style NULL-terminated ASCII string */
        PWR_MD_LIQUID_MAX_PUMP_FLOW, /* double */
        PWR_MD_LIQUID_MIN_PUMP_FLOW, /* double */
        PWR_MD_LIQUID_RECOMMENED_PUMP_FLOW, /* double */
        PWR_MD_LIQUID_MAX_TEMP,   /* double, degrees Celcius */
        PWR_MD_LIQUID_MIN_TEMP,   /* double, degrees Celcius */
        PWR_MD_LIQUID_MAX_PRESSURE, /* double, PSI */
        PWR_NUM_META_NAMES,	/* uint64_t */
        /* */
        PWR_MD_INVALID       = -1,
        PWR_MD_NOT_SPECIFIED = -2
} PWR_MetaName;
\end{lstlisting}
\end{minipage}
\end{center}


%==============================================================================%
%==============================================================================%
%==============================================================================%
%==============================================================================%
%==============================================================================%

\section{Error Return Definitions}\label{sec:ErrorReturnDefinitions}

The following required definitions are the available status returns for the functions described in this specification. 
It is anticipated that this list will grow. 
The implementor is also free to add status returns to express conditions not currently covered in the specification and expose them using an implementation specified context type (see section \ref{sec:ContextTypeDefinitions}).
The range -127 through 128 are reserved for use by the Power API specification.
Positive numbers greater than zero are to be used for warnings.


\begin{center}
\begin{minipage}{.95\linewidth}%
\begin{lstlisting}
#define PWR_RET_WARN_TRUNC 5
#define PWR_RET_WARN_NO_GRP_BY_NAME 4
#define PWR_RET_WARN_NO_OBJ_BY_NAME 3
#define PWR_RET_WARN_NO_CHILDREN 2
#define PWR_RET_WARN_NO_PARENT 1
#define PWR_RET_SUCCESS 0
#define PWR_RET_FAILURE -1
#define PWR_RET_NOT_IMPLEMENTED -2
#define PWR_RET_EMPTY -3
#define PWR_RET_INVALID -4
#define PWR_RET_LENGTH -5
#define PWR_RET_NO_ATTRIB -6
#define PWR_RET_NO_META -7
#define PWR_RET_READ_ONLY -8
#define PWR_RET_BAD_VALUE -9
#define PWR_RET_BAD_INDEX -10
#define PWR_RET_OP_NOT_ATTEMPTED -11
#define PWR_RET_NO_PERM -12
#define PWR_RET_OUT_OF_RANGE -13
#define PWR_RET_NO_OBJ_AT_INDEX -14
\end{lstlisting}
\end{minipage}
\end{center}

%==============================================================================%
%==============================================================================%
%==============================================================================%
%==============================================================================%
%==============================================================================%

\section{Time Related Definitions}\label{sec:TimeRelatedDefinitions}

\texttt{PWR_Time} is defined as a 64-bit value used to hold timestamps in nanoseconds for a wide range of functionality. 
For those timestamps that are to be used in relation to an epoch, midnight January 1st, 1970 will be considered the beginning of the epoch.
This will provide for hundreds of years to be expressed from the epoch point, which is sufficient for the purposes of the Power API.
\texttt{PWR_Time} is also used for other structures designed to record time values (\texttt{PWR_TimePeriod}, page \pageref{type:TimePeriod} for example). 
\texttt{PWR_TIME_UNINIT} is used as an indicator that the time value has not been initialized. 
This is intended to allow the implementation to make decisions on how a function is being used based on whether a time value has been specified or not (for example, the Statistics functions in section \ref{sec:StatisticsFunctions}).
\texttt{PWR_TIME_UNKNOWN} is an output, which indicates that the time of an event was not recorded. For example, a maximum value for an attribute could be known for a given time period, but the instant at which the maximum occurred is unknown.
The \texttt{PWR_TimePeriod} type allows for three timestamps, start, stop and instant. 
Instant is available to indicate when a statistically significant event occurred within the window delineated by start and stop.
For example, if the user requests the \texttt{PWR_ATTR_STAT_MAX} statistic for \texttt{PWR_ATTR_POWER}, the start and stop times will indicate the window of time over which the maximum value was calculated. 
The instant would indicate the instant in time the maximum value occurred.
Defining \texttt{PWR_Time}, \texttt{PWR_TIME_UNINIT}, \texttt{PWR_TIME_UNKNOWN}, and \texttt{PWR_TimePeriod} as specified is required.



\begin{center}
\begin{minipage}{.95\linewidth}%
\begin{lstlisting}
typedef uint64_t PWR_Time;
#define PWR_TIME_UNINIT 0
#define PWR_TIME_UNKNOWN 0 
\end{lstlisting}
\end{minipage}
\end{center}

%==============================================================================%
\subsubsection{PWR_TimePeriod}\label{type:TimePeriod}

\begin{center}
\begin{minipage}{.95\linewidth}%
\begin{lstlisting}
typedef struct {
        PWR_Time    start;
        PWR_Time    stop;
        PWR_Time    instant;
} PWR_TimePeriod;
\end{lstlisting}
\end{minipage}
\end{center}
%==============================================================================%
%==============================================================================%
%==============================================================================%
%==============================================================================%

\section{Statistics Relevant Type Definitions}\label{sec:StatisticTypeDefinitions}
The \texttt{PWR_AttrStat} type includes the list of currently defined statistics potentially available to the user of an implementation.
Potentially, because this feature requires either direct device or software support.
Statistics are generated on a per-attribute basis (see \texttt{PWR_AttrName} on page \pageref{type:AttrName}). 
The statistics type definitions are required to be implemented and are used with the statistics functions (see section \ref{sec:StatisticsFunctions}).

%==============================================================================%

\subsubsection{PWR_AttrStat}\label{type:AttrStat}

\begin{center}
\begin{minipage}{.95\linewidth}%
\begin{lstlisting}
typedef enum {
        PWR_ATTR_STAT_MIN = 0,
        PWR_ATTR_STAT_MAX,
        PWR_ATTR_STAT_AVG,
        PWR_ATTR_STAT_STDEV,
        PWR_ATTR_STAT_CV,
        PWR_ATTR_STAT_SUM,
        PWR_NUM_ATTR_STATS,
        /* */
        PWR_ATTR_STAT_INVALID       = -1,
        PWR_ATTR_STAT_NOT_SPECIFIED = -2
} PWR_AttrStat;
\end{lstlisting}
\end{minipage}
\end{center}

%==============================================================================%


%We probably don't need this anymore as we can use the stat's type for this
%but I'm leaving this in until version 1.1 is released in case we change our
%minds by finding a relevant use case for having this in the spec.
%--REG
%\subsubsection{PWR_StatReduction}\label{type:StatReduction}
%\begin{center}
%\begin{minipage}{.95\linewidth}%
%\begin{lstlisting}
%typedef enum {
%    PWR_ATTR_STAT_REDC_SUM,
%    PWR_ATTR_STAT_REDC_MIN,
%    PWR_ATTR_STAT_REDC_MAX,
%    PWR_ATTR_STAT_REDC_AVG
%} PWR_StatReduction;
%\end{lstlisting}
%\end{minipage}
%\end{center}

\subsubsection{PWR_ID}\label{type:ID}
\begin{center}
\begin{minipage}{.95\linewidth}%
\begin{lstlisting}
typedef enum {
        PWR_ID_USER = 0,
        PWR_ID_JOB,
        PWR_ID_RUN,
        PWR_NUM_IDS,
        /* */
        PWR_ID_INVALID       = -1,
        PWR_ID_NOT_SPECIFIED = -2
} PWR_ID;
\end{lstlisting}
\end{minipage}
\end{center}
%==============================================================================%
%==============================================================================%
%==============================================================================%
%==============================================================================%
%==============================================================================%

\section{OS Hardware Interface Type Definitions}\label{sec:OSHWTypeDefinitions}
The following definitions are used in the Operating system to Hardware interface described in section \ref{sec:OSHW}.
Each definition will be described below along with its specification. All of the definitions in this section are required, even if the corresponding OS/HW functions are not implemented. 


%==============================================================================%

\subsubsection{PWR_OperState}\label{type:OperState}

The \texttt{PWR_OperState} type is used to describe the state being requested by OS to Hardware interface functions that require power/performance state information such as P-State and C-State information. 
Both \texttt{c_state_num} and \texttt{p_state_num} must be provided.

\begin{center}
\begin{minipage}{.95\linewidth}%
\begin{lstlisting}
typedef struct {
        uint64_t c_state_num;
        uint64_t p_state_num;
} PWR_OperState;
\end{lstlisting}
\end{minipage}
\end{center}



%==============================================================================%
%==============================================================================%
%==============================================================================%
%==============================================================================%
%==============================================================================%

\section{Application OS Interface Type Definitions}\label{sec:AppOSTypeDefinitions}
The following definitions are primarily used in the Application to Operating system interface described in section \ref{sec:AppOS}.
Each definition will be described below along with its specification. All of the definitions in this section are required, even if the corresponding App/OS functions are not implemented.

%==============================================================================%
\subsubsection{PWR_RegionHint}\label{type:RegionHint}

The  \texttt{PWR_RegionHint} type is an abstraction intended to allow the application to communicate power and performance significant information to the operating system. It is used in conjunction with \texttt{PWR_RegionIntensity} to describe the type and extent of the behavior described for a given execution region. 
This information can then be used to \textit{tune} components, with the intent being a more power/performance efficient use of the components results. 
For example, if an application is going into a serial region, the performance of the application may benefit from the core running the serial portion of the code at a higher frequency, thereby completing that serial portion faster. 
Since the application is in a serial portion, the implementation may determine that the remaining cores may be put into a more power efficient state (a sleep state for example), thus possibly resulting in both a performance increase and a decrease in the amount of power/energy the application uses.
Regions may be specified as \texttt{PWR_REGION_DEFAULT} to indicate that the application is no longer providing a hint as to the region characteristics of currently executing code.

\begin{center}
\begin{minipage}{.95\linewidth}%
\begin{lstlisting}

typedef enum {
        PWR_REGION_DEFAULT = 0,
        PWR_REGION_SERIAL,
        PWR_REGION_PARALLEL,
        PWR_REGION_COMPUTE,
        PWR_REGION_COMMUNICATE,
        PWR_REGION_IO,
        PWR_REGION_MEM_BOUND,
        PWR_REGION_GLOBAL_LOOP,
        PWR_NUM_REGION_HINTS,
        /* */
        PWR_REGION_INVALID       = -1,
        PWR_REGION_NOT_SPECIFIED = -2
} PWR_RegionHint;

\end{lstlisting}
\end{minipage}
\end{center}

%==============================================================================%

\subsubsection{PWR_RegionIntensity}\label{type:RegionIntensity}

The \texttt{PWR_RegionIntensity} type is an abstraction of a given level of intensity for a \texttt{PWR_RegionHint}.
It provides five levels of intensity as well as \texttt{PWR_Region_INT_NONE}, which can be used in the case where
the intensity is not known, is not applicable, or in cases where the operating system or runtime may be better
equipped to determine the intensity of a given code region.

\begin{center}
\begin{minipage}{.95\linewidth}%
\begin{lstlisting}
typedef enum {
        PWR_REGION_INT_HIGHEST = 0,
        PWR_REGION_INT_HIGH,
        PWR_REGION_INT_MEDIUM,
        PWR_REGION_INT_LOW,
        PWR_REGION_INT_LOWEST,
        PWR_REGION_INT_NONE,
        PWR_NUM_REGION_INTENSITIES,
        /* */
        PWR_REGION_INT_INVALID       = -1,
        PWR_REGION_INT_NOT_SPECIFIED = -2
} PWR_RegionIntensity;
\end{lstlisting}
\end{minipage}
\end{center}

%==============================================================================%

\subsubsection{PWR_SleepState}\label{type:SleepState}

The \texttt{PWR_SleepState} type is a high level abstraction of the different sleep
state levels that may be provided on a given system. The sleep levels are translated into
the appropriate hardware level constructs by lower layers of the PowerAPI. 

\begin{center}
\begin{minipage}{.95\linewidth}%
\begin{lstlisting}
typedef enum {
        PWR_SLEEP_NO = 0,
        PWR_SLEEP_SHALLOW,
        PWR_SLEEP_MEDIUM,
        PWR_SLEEP_DEEP,
        PWR_SLEEP_DEEPEST,
        PWR_NUM_SLEEP_STATES,
        /* */
        PWR_SLEEP_INVALID       = -1,
        PWR_SLEEP_NOT_SPECIFIED = -2
} PWR_SleepState; 
\end{lstlisting}
\end{minipage}
\end{center}

%==============================================================================%

\subsubsection{PWR_PerfState}\label{type:PerfState}

The \texttt{PWR_PerfState} type is an abstraction meant to describe the different
possible performance states in which hardware may be placed.

\begin{center}
\begin{minipage}{.95\linewidth}%
\begin{lstlisting}
typedef enum {
        PWR_PERF_FASTEST = 0,
        PWR_PERF_FAST,
        PWR_PERF_MEDIUM,
        PWR_PERF_SLOW,
        PWR_PERF_SLOWEST,
        PWR_NUM_PERF_STATES,
        /* */
        PWR_PERF_INVALID       = -1,
        PWR_PERF_NOT_SPECIFIED = -2
} PWR_PerfState;
\end{lstlisting}
\end{minipage}
\end{center}

%==============================================================================%

\subsubsection{PWR_NoteType}\label{type:NoteType}

The \texttt{PWR_NoteType} type is generalized notification classification mechanism,
allowing notifications to carry a basic type for handling at the target.

\begin{center}
\begin{minipage}{.95\linewidth}%
\begin{lstlisting}
typedef enum {
        PWR_NOTE_FATAL_FAILURE = 0,
        PWR_NOTE_JOB_FAILURE,
        PWR_NOTE_RESTART,
        PWR_NOTE_CAP_VIOLATION,
        PWR_NOTE_NODE_FAILURE,
        PWR_NOTE_NODE_UNRESPONSIVE,
        PWR_NOTE_RESUME_NORMAL,
        /* */
        PWR_NOTE_INVALID       = -1,
        PWR_NOTE_NOT_SPECIFIED = -2
} PWR_NoteType; 
\end{lstlisting}
\end{minipage}
\end{center}


%==============================================================================%

\subsubsection{PWR_Note}\label{type:Note}

The \texttt{PWR_Note} type is a structure that encapsulates an entire desirable 
notification between higher lever system layers and lower level interfaces. For example,
such notifications can be passed between a resource manager and an indiviudal node, bypassing
intermediate layers like a job manager and vice versa. \texttt{PWR_NoteType} must be a valid
type as defined in section~\ref{type:NoteType}. The \texttt{note_details} field can contain a 
string with additional information on the notification following the required notation from
the sending/receving software. \texttt{source_id} should contain the source node's numbering,
perferrably the Power API object ID.  

\begin{center}
\begin{minipage}{.95\linewidth}%
\begin{lstlisting}
typedef struct {
        PWR_NoteType note;
        uint64_t version;
        char * note_details;
        uint64_t note_len;
        char * source_id;
        uint64_t source_id_len;
} PWR_Note;
\end{lstlisting}
\end{minipage}
\end{center}




	\chapter{Core (Common) Interface Functions}\label{chap:Common}
	% These have been superceded by the \newenvironment{prototype} macros.  Remove after conversion.
\newcommand{\ArgHdr}{   \begin{center}
			\begin{tabular}{ | p{5cm} | p{2.1cm} | p{7.5cm} |}
			\hline 
			\textbf{Argument(s)} & \textbf{Input } & \textbf{Description} \\  
			 		     & \textbf{and/or} & \\ 
					     & \textbf{Output} & \\ 
			\hline 
}
\newcommand{\ArgLn}[3]{\texttt{#1} & #2 & #3 \\ \hline}
\newcommand{\ArgFtr}{ 	\end{tabular}
			\end{center}
}



Core, or so called Common, interface functions are functions that can be used, at least in par, by most of the interfaces described in the Power API specification. 
Core functions include the following areas:
\begin{itemize}[noitemsep,nolistsep] 
\item{\textbf{Initialization}, required to use any of the functionality described in this specification, }
\item{\textbf{Navigation}  functions allow the user to traverse the system description and discover information about the underlying facility/platform, }
\item{\textbf{Group} functions, primarily a convenience abstraction,}
\item{\textbf{Attribute} functions expose measurement and control functionality, }
\item{\textbf{Metadata} functions allow the user to access additional information about objects and attributes (often device or instrumentation specific information), }
\item{\textbf{Statistics} functions are used to generate statistical information based on fundamental attribute information (measurements),}
\end{itemize}
and other functionality that is common across a number of interfaces.

%=============================================================================%
%=============================================================================%
%=============================================================================%
%=============================================================================%
%=============================================================================%
%=============================================================================%

\section{Initialization}\label{sec:Initialization}

Initialization using \texttt{PWR_CntxtInit} is required to use any of the functionality documented in this specification.
The user supplies the type of the context requested and their role. 
Currently, the specification's only required context type is \texttt{PWR_CNTXT_DEFAULT}.
The context type is intended to be one way in which the implementor can distinguish their implementation from the standard specification and other implementations (see section \ref{sec:ContextTypeDefinitions}).
The user must also supply their role (see page \pageref{type:Role} for the \texttt{PWR_Role} definition).
One purpose of specifying the role is to convey what type of user they intend to be, and therefore, how they would like to interact with or how the underlying implementation manages the privileges granted to the user/role combination.
A system administrator (\texttt{PWR_ROLE_ADMIN}) will desire and require different capabilities, privileges and level of abstraction than the application user (\texttt{PWR_ROLE_APP}), for example.

The user also has the opportunity to specify a name that will be associated with the context. 
This \textit{feature} is anticipated to be useful in supporting advanced functionality.
Initialization returns a context to the user.
The context contains the user's view of the system, dependent on what type of context was requested, the user's role and implementation specifics.
The system description that the user is exposed to must conform to the rules outlined in the specification (see sections \ref{sec:PowerAPIInit} and \ref{sec:PowerAPIBaseSysDesc}).
The context should be destroyed (cleaned up) by using the \texttt{PWR_CntxtDestroy} function when no longer needed.

%=============================================================================%
%int PWR_CntxtInit( PWR_CntxtType type, PWR_Role role, const char* name, PWR_Cntxt* context);
\begin{prototype}{CntxtInit}
	\longdescription{ The \PWR{CntxtInit} function is required to be called before using any other Power API function.  The context returned is passed to other Power API functions either explicitly as an argument or implicitly through an argument associated with the context.
	}
	\pnote{See page \pageref{type:CntxtType} for a discussion of contexts and roles.}
	\returntype{int}
	\parameter{\PWR{CntxtType} type}{\pInput} 	{The requested context type.}
	\parameter{\PWR{Role} role}	{\pInput} 	{The role of the user.}
	\parameter{const char* name}  	{\pInput} 	{User specified string name to be associated with the context.}
	\parameter{\PWR{Cntxt}* context}{\pOutput}	{The user's context.}
	\returnval{\PWR{RET_SUCCESS}}			{Upon SUCCESS, context is set to a valid user context.}
	\returnval{\PWR{RET_FAILURE}}			{Upon FAILURE.}
\end{prototype}

%=============================================================================%
%int    PWR_CntxtDestroy( PWR_Cntxt context );
\begin{prototype}{CntxtDestroy}
	\longdescription{The \PWR{CntxtDestroy} function is used to destroy (clean up) the context obtained with \PWR{CntxtInit}. The implementation is required to clean up, unlink, destroy (as appropriate) all context resources as a result of this call.}
	\returntype{int}
	\parameter{PWR_Cntxt context}{\pInput}{The context obtained using \PWR{CntxtInit} the user wishes to destroy.}
	\returnval{\PWR{RET_SUCCESS}}	{Upon SUCCESS.}
	\returnval{\PWR{RET_FAILURE}}	{Upon FAILURE.}
\end{prototype}

%=============================================================================%
%=============================================================================%
%=============================================================================%
%=============================================================================%
%=============================================================================%
%=============================================================================%
%=============================================================================%


\section{Hierarchy Navigation Functions}\label{sec:Navigation}

Hierarchy navigation (also called discovery) is accomplished using attributes (EntryPoint, Type, Parent and Children) that are implicit to every object in the system description whether defined in the specification or added by the implementor.
Navigation is accomplished using these attributes, through the associated function calls, within the context made available to the user upon initialization.
After initialization the first call will generally be \texttt{PWR_CntxtGetEntryPoint} to determine the user's entry point in the system hierarchy provided within the user's context. 
Depending on the user, the interface and the role, the context could contain a view of the entire system description or a subset of the system description.
Navigating through the hierarchy is accomplished with \texttt{PWR_ObjGetParent} to navigate up and \texttt{PWR_ObjGetChildren} to navigate down.
To understand what kind of object was returned with either of these calls the user can utilize \texttt{PWR_ObjGetType} call.
The name of the object can be discovered using the \texttt{PWR_ObjGetName} function and if the user has a name, the associated object can be discovered using the \texttt{PWR_CntxtGetObjByName} function. 

The Power API does not provide an explicit ``Free Object'' interface.
Specifically, objects returned by Power API interfaces do not need to be later freed or released explicitly.
This design choice was made in order to keep usage of the Power API as simple as possible, with the potential cost of an increased burden on the Power API implementor to limit implementation-internal memory usage.


%=============================================================================%
%int    PWR_CntxtGetEntryPoint( PWR_Cntxt context, PWR_Obj* entry_point );
\begin{prototype}{CntxtGetEntryPoint}
	\longdescription{The \texttt{PWR_CntxtGetEntryPoint} call is typically used immediately following initialization.  Whenever \texttt{PWR_CntxtGetEntryPoint} is called the implementation defined entry point (location) in the system description is returned.  \texttt{PWR_CntxtGetEntryPoint} can always be called to reposition or reorient the user to the initial entry location.}
	\returntype{int}
	\parameter{PWR_Cntxt context}   {\pInput} 	{The user's context.}
	\parameter{PWR_Obj* entry_point}{\pOutput}	{The user's entry point into the system description (the same for the life of the context).}
	\returnval{PWR_RET_SUCCESS} 			{ Upon SUCCESS, entry_point set to system description entry point (object).}	
	\returnval{PWR_RET_FAILURE} 			{ Upon FAILURE.}
\end{prototype}

%=============================================================================%
%int    PWR_ObjGetType( PWR_Obj object, PWR_ObjType* type );
\begin{prototype}{ObjGetType}
	\longdescription{ The \texttt{PWR_ObjGetType} function returns the type of the object specified.  See page \pageref{type:ObjType} for valid object types.}
	\returntype{int}
	\parameter{PWR_Obj object}	{\pInput}	{The object that the user wishes to determine the type of.}
	\parameter{PWR_ObjType* type}	{\pOutput}	{The type of the specified object.}
	\returnval{PWR_RET_SUCCESS} 	{ Upon SUCCESS, type is set to the type of the specified object.}
	\returnval{PWR_RET_FAILURE} 	{ Upon FAILURE, type is set to PWR_OBJ_INVALID.}
\end{prototype}
%=============================================================================%
%int    PWR_ObjGetName( PWR_Obj object, char* dest, size_t len );
\begin{prototype}{ObjGetName}
	\longdescription{The \texttt{PWR_ObjGetName} function copies the name of the specified object into the user provided buffer. See page \pageref{func:CntxtGetObjByName} to get the object based on the unique name using \texttt{PWR_CntxtGetObjByName}.}
	\returntype{int}
	\parameter{PWR_Obj object}	{\pInput}	{The object that the user wishes to determine the name of.}
	\parameter{char* dest}    	{\pInput}	{The address of the user provided buffer.}
	\parameter{size_t len}    	{\pInput}	{The length of the user provided buffer.}
	\returnval{PWR_RET_SUCCESS} 	{Upon SUCCESS, the buffer will contain the name of the object, the string will include a terminating null byte.}	
	\returnval{PWR_RET_WARN_TRUNC} 	{Call succeeded, but the length of object name was longer than the provided buffer and the name was truncated.}
	\returnval{PWR_RET_FAILURE} 	{Upon FAILURE.}
\end{prototype}
%=============================================================================%
%int    PWR_ObjGetSizeOfName( PWR_Obj object, size_t len );
\begin{prototype}{ObjGetSizeOfName}
    \longdescription{The \texttt{PWR_ObjGetSizeOfName} returns the length of an object's name.  The len parameter will contain the length of the name of the specified object including any string terminators upon return. See page \pageref{func:CntxtGetObjByName} to get the object based on the unique name using \texttt{PWR_CntxtGetObjByName}.}
    \returntype{int}
    \parameter{PWR_Obj object}  {\pInput}   {The object that the user wishes to determine the name of.}
    \parameter{size_t* len}     {\pInputOutput} {The length of the user provided buffer.}
    \returnval{PWR_RET_SUCCESS}     {Upon SUCCESS, the len parameter will contain the size of buffer required to successfully call \texttt{PWR_ObjGetName}, including terminating null byte.}
    \returnval{PWR_RET_FAILURE}     {Upon FAILURE.}
\end{prototype}

%==============================================================================%
%int PWR_ObjGetParent( PWR_Obj object, PWR_Obj* parent );
\begin{prototype}{ObjGetParent}
	\longdescription{The \texttt{PWR_ObjGetParent} function is used to find the object immediately above the specified object in the system description available to the user through the current context.  Note, currently, there are some cases where an object has no parent, namely the facility object.}
	\returntype{int}
	\parameter{PWR_Obj object} {\pInput} 		{The object that the user wishes to determine the parent of.}
	\parameter{PWR_Obj* parent}{\pOutput}		{The parent object of the specified input object.}
	\returnval{PWR_RET_SUCCESS}        		{Upon SUCCESS, parent set to parent of specified object.}
	\returnval{PWR_RET_WARN_NO_PARENT} 		{Call succeeded but specified object does not have a parent.}
	\returnval{PWR_RET_FAILURE}        		{Upon FAILURE.}
\end{prototype}

%==============================================================================%
%int     PWR_ObjGetChildren( PWR_Obj object, PWR_Grp* group );
\begin{prototype}{ObjGetChildren}
	\longdescription{The \texttt{PWR_ObjGetChildren} function returns the child or children of the specified object.  The caller is expected to check the return code of \texttt{PWR_ObjGetChildren} to determine if the object has children or not.  If the specified object has one or more children, indicated by a return code of \texttt{PWR_RET_SUCCESS}, a new group (\texttt{PWR_Grp}) is returned that contains the object's children.  The user is responsible for destroying this group when it is no longer needed (see \texttt{PWR_GrpDestroy} on page \pageref{func:GrpDestroy}).  If the specified object has no children, indicated by a return code of \texttt{PWR_RET_WARN_NO_CHILDREN}, no group is returned and the input (\texttt{PWR_Grp}) is not modified.}
	\returntype{int}
	\parameter{PWR_Obj objec}	{\pInput}	{The object that the user wishes to determine the children of.}
	\parameter{PWR_Grp* group}	{\pOutput}	{On input, this should be set to point to an uninitialized \texttt{PWR_Grp} (i.e., the caller should not call \texttt{PWR_GrpCreate} ahead of time).  If \texttt{PWR_RET_SUCCESS} is returned, *group will be set to a newly created group containing the \texttt{object}'s children.  If \texttt{PWR_RET_WARN_NO_CHILDREN} is returned, the input \texttt{PWR_Grp} is not modified.}
	\returnval{PWR_RET_SUCCESS}          		{Upon SUCCESS, group is set to a newly created group containing the child or children of specified object.}
	\returnval{PWR_RET_WARN_NO_CHILDREN} 		{Call succeeded but specified object does not have any children. The input \texttt{PWR_Grp} is not modified.}
	\returnval{PWR_RET_FAILURE}          		{Upon FAILURE.}
\end{prototype}

%=============================================================================%
%int    PWR_CntxtGetObjByName( PWR_Cntxt context, const char* name, PWR_Obj* object);
\begin{prototype}{CntxtGetObjByName}
	\longdescription{The \texttt{PWR_CntxtGetObjByName} function returns the object given the context and unique object name.  See page \pageref{func:ObjGetName} to get the name of a specified object using \texttt{PWR_ObjGetName}.}
	\returntype{int}
	\parameter{PWR_Cntxt context}	{\pInput}{The context containing the object that the user wishes to retrieve given its unique name. Note, the object may be present in the system but not available to the user through the current context.}
	\parameter{const char * name}	{\pInput}{The unique name of the object that the user wishes to retrieve.}
	\parameter{PWR_Obj* object}	{\pOutput}{The object that corresponds to the name specified by the user.}
	\returnval{PWR_RET_SUCCESS}             	{Upon SUCCESS, object is set to object corresponding to name specified by user.}
	\returnval{PWR_RET_WARN_NO_OBJ_BY_NAME} 	{If no object exists corresponding to name provided.}
	\returnval{PWR_RET_FAILURE}             	{Upon FAILURE.}
\end{prototype}

%==============================================================================%
%=============================================================================%
%=============================================================================%
%=============================================================================%
%=============================================================================%
%=============================================================================%
%=============================================================================%
\section{Group Functions}\label{sec:Group}

Group functions are provided as a convenience in situations, for example, where an operation, or operations are required to be executed on multiple objects. 
Rather than executing the same operation multiple times, once for each object, some operations provide a group variant to streamline this type of functionality.
Groups can be dynamically created (\texttt{PWR_GrpCreate}) when needed and can exist for short periods of time and destroyed with \texttt{PWR_GrpDestroy}, or exist for the duration of the users context.
Groups may not contain multiple instances of the same object, i.e. duplicate objects are not allowed.
When a new group is the product of a function (\texttt{PWR_GrpUnion, PWR_GrpIntersection, PWR_GrpDifference}) and the result of the function operation is the empty set (no objects) an empty group (group with no objects)  should be the result and the function should return \texttt{PWR_RET_SUCCESS}. 
It is the responsibility of the user to clean up all groups produced as a result of group functions using \texttt{PWR_GrpDestroy}.
Groups can only contain objects from a single \texttt{PWR_cntxt}.
Group operations that involve multiple groups must be performed with groups from the same context.

%int PWR_GrpCreate( PWR_Cntxt context, PWR_Grp* group );
\begin{prototype}{GrpCreate}
	\longdescription{The \texttt{PWR_GrpCreate} function is used to create a new group which will be associated with and unique to the users context.}
	\returntype{int}
	\parameter{PWR_Cntxt context}	{\pInput} {The user's context that the group, when created, will be associated with.}
	\parameter{PWR_Grp* group}   	{\pOutput}{The new (empty) group.}
	\returnval{PWR_RET_SUCCESS} 	{Upon SUCCESS, group is set to new (empty) group.}
	\returnval{PWR_RET_FAILURE} 	{Upon FAILURE.}
\end{prototype}

%=============================================================================%
%int PWR_GrpDestroy( PWR_Grp group );
\begin{prototype}{GrpDestroy}
	\longdescription{The \texttt{PWR_GrpDestroy} function is used to destroy (clean up) a group created by a user.}
	\returntype{int}
	\parameter{PWR_Grp group}{\pInput}{The group that the user is acting on.}
	\returnval{PWR_RET_SUCCESS} 	{Upon SUCCESS.}
	\returnval{PWR_RET_FAILURE} 	{Upon FAILURE.}
\end{prototype}

%=============================================================================%
%int PWR_GrpAddObj( PWR_Grp group, PWR_Obj object );
\begin{prototype}{GrpAddObj}
	\longdescription{The \texttt{PWR_GrpAddObj} function is used to add a specified object to a specified group.  Duplicate objects are not allowed in groups.  Adding an object that would be a duplicate of one already in the group will result in no insertion and returns \texttt{PWR_RET_SUCCESS}.}
	\returntype{int}
	\parameter{PWR_Grp group} 	{\pInputOutput}	{The group that the user is acting on.}
	\parameter{PWR_Obj object}	{\pInput}       {The object to be added to the specified group.}
	\returnval{PWR_RET_SUCCESS} 	{Upon SUCCESS.}
	\returnval{PWR_RET_FAILURE} 	{Upon FAILURE.}
\end{prototype}

%=============================================================================%
%int PWR_GrpRemoveObj( PWR_Grp group, PWR_Obj object );
\begin{prototype}{GrpRemoveObj}
	\longdescription{The \texttt{PWR_GrpRemoveObj} function is used to remove a specified object from a specified group.  Attempting to remove an object that is not a member of a group will result in \texttt{PWR_RET_SUCCESS}.}
	\returntype{int}
	\parameter{PWR_Grp group} 	{\pInputOutput}	{The group that the user is acting on.}
	\parameter{PWR_Obj object}	{\pInput}       {The object to be removed from the specified group.}
	\returnval{PWR_RET_SUCCESS}  	{Upon SUCCESS.}
	\returnval{PWR_RET_FAILURE}  	{Upon FAILURE.}
\end{prototype}
%=============================================================================%
%int PWR_GrpGetNumObjs( PWR_Grp group );
\begin{prototype}{GrpGetNumObjs}
	\longdescription{The \texttt{PWR_GrpGetNumObjs} function is used to get the number of objects contained in the specified group.} \returntype{int} \parameter{PWR_Grp group}{\pInput}{The group that the user is acting on.}
	\returnval{int} 		{Upon SUCCESS, the number of objects contained in the specified group.}
	\returnval{PWR_RET_FAILURE} 	{Upon FAILURE.}
\end{prototype}

%=============================================================================%
%int PWR_GrpGetObjByIndx( PWR_Grp group, int index, PWR_Obj* object);
\begin{prototype}{GrpGetObjByIndx}
	\longdescription{The \texttt{PWR_GrpGetObjByIndx} is used to get the object from the specified group at the specified index.}
	\returntype{int}
	\parameter{PWR_Grp group}  {\pInput} {The group that the user is acting on.}
	\parameter{int  index}     {\pInput} {The index within the specified group of the desired object.}
	\parameter{PWR_Obj* object}{\pOutput}{The object at the specified index in the specified group.}
	\returnval{PWR_RET_SUCCESS} 		{Upon SUCCESS, object is set to object at specified index.}
	\returnval{PWR_RET_NO_OBJ_AT_INDEX} 	{No object at specified index in specified group.}
	\returnval{PWR_RET_FAILURE} 		{Upon FAILURE.}
\end{prototype}
%=============================================================================%
%int PWR_GrpDuplicate( PWR_Grp group1, PWR_Grp* group2 );
\begin{prototype}{GrpDuplicate}
	\longdescription{The \texttt{PWR_GrpDuplicate} function is used to duplicate an existing group.  The duplicate group is a new separate group from the original group specified.  Actions on the duplicate group do not affect the original group and vice versa.}
	\returntype{int}
	\parameter{PWR_Grp group1}	{\pInput}{The original group (group1).}
	\parameter{PWR_Grp* group2}	{\pOutput}{Duplicate (group2) of the original group (group1) specified by user even if the original group contains no objects.}

	\returnval{PWR_RET_SUCCESS} 	{Upon SUCCESS, duplicate group of original group created.}
	\returnval{PWR_RET_FAILURE} 	{Upon FAILURE.}
\end{prototype}

%=============================================================================%
%int PWR_GrpUnion( PWR_Grp group1, PWR_Grp group2, PWR_Grp* group3);
\begin{prototype}{GrpUnion}
	\longdescription{The \texttt{PWR_GrpUnion} function is used to create a group that is the union ($\cup$) of two specified groups.  The union group created is a new separate group from the original groups specified.  Actions on the union group do not affect the original groups and vice versa.}
	\returntype{int}
	\parameter{PWR_Grp group1}	{\pInput}{The first of the two groups used in the union, ($\cup$) operation.}
	\parameter{PWR_Grp group2}	{\pInput}{The second of the two groups used in the union, ($\cup$) operation.}
	\parameter{PWR_Grp* group3}	{\pOutput}{he output group (group3) is the union, ($\cup$) operation, of the first (group1) and second (group2) groups specified. If the result of the union operation is the empty set group3 is an empty group (valid group with no objects).}
	\returnval{PWR_RET_SUCCESS} 	{Upon SUCCESS, group3 contains the union of group1 and group2.}
	\returnval{PWR_RET_FAILURE} 	{Upon FAILURE.}
\end{prototype}

%=============================================================================%
%int PWR_GrpIntersection( PWR_Grp group1, PWR_Grp group2, PWR_Grp* group3);
\begin{prototype}{GrpIntersection}
	\longdescription{The \texttt{PWR_GrpIntersection} function is used to create a group that is the Intersection ($\cap$) of two specified groups.  The intersection group is a new separate group from the original groups specified.  Actions on the intersection group do not affect the original groups and vice versa.}
	\returntype{int}
	\parameter{PWR_Grp group1}	{\pInput}{The first of the two groups used in the Intersection ($\cap$)  operation.}
	\parameter{PWR_Grp group2}	{\pInput}{The second of the two groups used in the intersection ($\cap$) operation.}
	\parameter{PWR_Grp* group3}	{\pOutput}{The output group (group3) is the intersection, ($\cap$) operation, of the first (group1) and second (group2) groups specified. If the result of the intersection operation is the empty set group3 is an empty group (valid group with no objects).}
	\returnval{PWR_RET_SUCCESS} 	{Upon SUCCESS, group3 contains the intersection of group1 and group2.}
	\returnval{PWR_RET_FAILURE} 	{Upon FAILURE.}
\end{prototype}

%=============================================================================%
%int PWR_GrpDifference( PWR_Grp group1, PWR_Grp group2, PWR_Grp* group3 );
\begin{prototype}{GrpDifference}
	\longdescription{The \texttt{PWR_GrpDifference} function is used to create a group that is the Difference ($\setminus$) of two specified groups.  The difference group is a new separate group from the original groups specified.  Actions on the difference group do not affect the original groups and vice versa.  In the event that the output \texttt{PWR_Grp} contains no objects see~\ref{sec:Group} for the definition of the output, \texttt{PWR_Grp}.}
	\returntype{int}
	\parameter{PWR_Grp group1}	{\pInput}{The first of the two groups used in the difference ($\setminus$)  operation.}
	\parameter{PWR_Grp group2}	{\pInput}{The second of the two groups used in the difference ($\setminus$) operation.}
	\parameter{PWR_Grp* group3}	{\pOutput}{The output group (group3) is the difference, ($\setminus$) operation, of the first (group1) and second (group2) groups specified. If the result of the difference operation is the empty set group3 is an empty group (valid group with no objects).}
	\returnval{PWR_RET_SUCCESS} 	{Upon SUCCESS,  group3 contains the difference of group1 and group2.}
	\returnval{PWR_RET_FAILURE} 	{Upon FAILURE.}
\end{prototype}

%=============================================================================%

\subsubsection{Function Prototype for PWR_GrpSymDifference()}\label{func:GrpSymDifference}
%int PWR_GrpSymDifference( PWR_Grp group1, PWR_Grp group2, PWR_Grp* group3 );
\begin{prototype}{GrpSymDifference}
	\longdescription{The \texttt{PWR_GrpSymDifference} function is used to create a group that is the Symmetric Difference ($\triangle$) of two specified groups.  The symmetric difference group is a new separate group from the original groups specified.  Actions on the symmetric difference group do not affect the original groups and vice versa.  In the event that the output \texttt{PWR_Grp} contains no objects see~\ref{sec:Group} for the definition of the output, \texttt{PWR_Grp}.}
	\returntype{int}
	\parameter{PWR_Grp group1}{Input}{The first of the two groups used in the symmetric difference ($\triangle$)  operation.}
	\parameter{PWR_Grp group2}{Input}{The second of the two groups used in the symmetric difference ($\triangle$) operation.}
	\parameter{PWR_Grp* group3}{Output}{The output group (group3) is the symmetric difference, ($\triangle$) operation, of the first (group1) and second (group2) groups specified. If the result of the symmetric difference operation is the empty set group3 is an empty group (valid group with no objects).}
	\returnval{PWR_RET_SUCCESS} 	{Upon SUCCESS,  group3 contains the symmetric difference of group1 and group2.}
	\returnval{PWR_RET_FAILURE} 	{Upon FAILURE.}
\end{prototype}

%=============================================================================%
% int   PWR_CntxtGetGrpByName( PWR_Cntxt context, const char* name, PWR_Grp* group);
\begin{prototype}{CntxtGetGrpByName}
	\longdescription{The \texttt{PWR_CntxtGetGrpByName} function returns a group in a given context via a unique group name.  This function is included to allow the user to make use of groups that are provided with the initial context by the implementation.  The list of valid group names should be provided by the vendor in their documentation.  Due to the defined group names being vendor specific, use of this function should be considered non-portable.  The group returned by this call must be functionally identical to a group created via \texttt{PWR_GrpCreate()}.  Like a group created with \texttt{PWR_GrpCreate()} groups returned by \texttt{PWR_CntxtGetGrpByName()} must be destroyed with the \texttt{PWR_GrpDestroy()} call.}
	\returntype{int}
	\parameter{PWR_Cntxt context}	{\pInput}	{The context containing the group that the user wishes to retrieve given its unique name.}
	\parameter{const char* name}	{\pInput}	{The unique name of the group that the user wishes to retrieve.}
	\parameter{PWR_grp* group}	{\pOutput}	{The implementation provided group corresponding to the specified name.}
	\returnval{PWR_RET_SUCCESS}        		{Upon SUCCESS, group corresponding to the specified name.}
	\returnval{PWR_RET_WARN_NO_GRP_BY_NAME} 	{If no implementation supplied group exists corresponding to name provided.}
	\returnval{PWR_RET_FAILURE}             	{Upon FAILURE.}
\end{prototype}

%=============================================================================%
%=============================================================================%
%=============================================================================%
%=============================================================================%
%=============================================================================%
%=============================================================================%
%=============================================================================%

\section{Attribute Functions}\label{sec:Attributes}

The Attribute functions make up the foundation of the Power API specification, providing measurement (get) and control (set) interfaces for a wide range of power and energy related functionality.
Get and set interfaces are provided for single attribute/single object, multiple attribute/single object, single attribute/multiple objects (group) and multiple attributes/multiple objects (group).
In each case the user specifies the attribute or attributes to get or set.
The valid attribute names are defined in the \texttt{PWR_AttrName} structure (see page \pageref{type:AttrName}).
A complete list of all the valid attributes and their meanings can be found in table \ref{table:MasterAttributeTable}, section \ref{sec:BLOA}.
The timestamp is a critical part of the get (measurement) interface for power and energy related information.
It is very important that the timestamp returned (\texttt{PWR_Time}) be an accurate representation of when the value returned was measured to the best possible temporal accuracy, not when the function was called.
It is required by the specification that the value returned is the value that was measured as close as possible to when the get function was called.
The quality of the measurement and timestamp are device and implementation dependent.
Information about each attribute can be obtained through the metadata interface, described in section \ref{sec:METADATA}.

%
%==============================================================================%
%int PWR_ObjAttrGetValue( PWR_Obj object, PWR_AttrName attr, void* value, PWR_Time* ts );
\begin{prototype}{ObjAttrGetValue}
	\longdescription{The \texttt{PWR_ObjAttrGetValue} function is provided to get the value of a single specified attribute (\texttt{PWR_AttrName attr}) from a single specified object (\texttt{PWR_Obj object}).  The timestamp returned (\texttt{PWR_Time *ts}) should accurately represent when the value was measured.}
	\returntype{int}
	\parameter{PWR_Obj object}	{\pInput}	{The target object.}
	\parameter{PWR_AttrName attr}	{\pInput}	{The target attribute. See section \ref{type:AttrName} for a list of available attributes}
	\parameter{void* value}		{\pOutput}	{Pointer to caller-allocated storage, of 8 bytes, to hold the value read from the attribute.}
	\parameter{PWR_Time* ts}	{\pOutput}	{Pointer to caller-allocated storage to hold the timestamp of when the value was read from the attribute. Pass in \texttt{NULL} if the timestamp is not needed.}
	\returnval{PWR_RET_SUCCESS} 			{Upon SUCCESS.}
	\returnval{PWR_RET_NOT_IMPLEMENTED} 		{The requested attribute is not supported for the target object.}
	\returnval{PWR_RET_FAILURE} 			{Upon FAILURE.}
\end{prototype}
%==============================================================================%
%int PWR_ObjAttrSetValue( PWR_Obj object, PWR_AttrName attr, const void* value );
\begin{prototype}{ObjAttrSetValue}
	\longdescription{The \texttt{PWR_ObjAttrSetValue} function is provided to set the value of a single specified attribute (\texttt{PWR_AttrName attr}) of a single specified object (\texttt{PWR_Obj object}).}
	\returntype{int}
	\parameter{PWR_Obj object}	{\pInput}	{The target object.}
	\parameter{PWR_AttrName attr}	{\pInput}	{The target attribute. See section \ref{type:AttrName} for a list of available attributes.}
	\parameter{const void* value}	{\pInput}	{Pointer to the 8 byte value to write to the attribute.}
	\returnval{PWR_RET_SUCCESS} 			{Upon SUCCESS.}
	\returnval{PWR_RET_NOT_IMPLEMENTED} 		{The requested attribute is not supported for the target object.}
	\returnval{PWR_RET_BAD_VALUE} 			{The value was not appropriate for the target attribute.}
	\returnval{PWR_RET_OUT_OF_RANGE} 		{The value was out of range for the target attribute.}
	\returnval{PWR_RET_FAILURE} 			{Upon FAILURE.}
\end{prototype}
%==============================================================================%
%int PWR_StatusCreate( PWR_Cntxt context, PWR_Status* status );
\begin{prototype}{StatusCreate}
	\longdescription{The \texttt{PWR_StatusCreate} function is provided to create the \texttt{PWR_Status} object that will be used in functions that perform multiple operations and potentially return individual statuses for each operation.  It is up to the implementation to create the appropriate amount of storage for the \texttt{PWR_Status} structure based on the implementation and the number of statuses that will be held.  \texttt{PWR_Status} objects can only be used in the context in which they are created, attempting to use a \texttt{PWR_Status} object in a context other than the one it was created for will result in an error.  For example see \texttt{PWR_ObjAttrGetValues} on page \pageref{func:ObjAttrGetValues}.  Note, \texttt{PWR_Status} is an opaque handle, its backing definition is determined by the implementor (see \ref{sec:OpaqueTypes}).  It is intended that the implementation only allocate space for failed operations.  Errors are read from the \texttt{PWR_Status} by popping them off the structure which requires the structure to only be as large as the number of error returns require.  When status objects are passed into a function, they are automatically cleared, therefore errors should always be checked on a status object before reuse.  \emph{Note to Users: Caution is advised when reusing status objects in multiple threads. Common thread safety practices must be followed to ensure that errors are properly caught. Creating status objects for each thread is advised to avoid potential race conditions.}}
	\returntype{int}
	\parameter{PWR_Cntxt context}	{\pInput}	{The context in which the new status is to be used.}
	\parameter{PWR_Status* status}	{\pOutput}	{The new status structure.}
	\returnval{PWR_RET_SUCCESS} 	{Upon SUCCESS.}
	\returnval{PWR_RET_FAILURE} 	{Upon FAILURE.}
\end{prototype}
%==============================================================================%
%int PWR_StatusDestroy( PWR_Status status );
\begin{prototype}{StatusDestroy}
	\longdescription{The \texttt{PWR_StatusDestroy} function is provided to 	destroy the \texttt{PWR_Status} object created using \texttt{PWR_StatusCreate} (see page \pageref{func:StatusCreate}.  Note, \texttt{PWR_Status} is an opaque handle, its backing definition is determined by the implementor (see \ref{sec:OpaqueTypes}).}
	\returntype{int}
	\parameter{PWR_Status status}	{\pInput}	{The \texttt{PWR_Status} structure the user wishes to destroy.}
	\returnval{PWR_RET_SUCCESS} 	{Upon SUCCESS.}
	\returnval{PWR_RET_FAILURE} 	{Upon FAILURE.}
\end{prototype}
%==============================================================================%
%int PWR_StatusPopError( PWR_Status status, PWR_AttrAccessError* error );
\begin{prototype}{StatusPopError}
	\longdescription{The \texttt{PWR_StatusPopError} function is provided to iterate through the \texttt{PWR_Status} object created using \texttt{PWR_StatusCreate} (see page \pageref{func:StatusCreate}) and populated using any of the function calls that leverage this structure.  Using this method allows the \texttt{PWR_Status} structure to only grow as large as necessary storing only error returns.  Note, \texttt{PWR_Status} is an opaque handle, its backing definition is determined by the implementor (see \ref{sec:OpaqueTypes}). The \texttt{PWR_AttrAccessError} structure that is returned will always have its \texttt{obj}, \texttt{attr}, and \texttt{error} fields set to the object, attribute, and error code associated with the error.  The \texttt{PWR_AttrAccessError} structure's index field will only be set for attribute get functions (e.g., \texttt{PWR_ObjAttrGetValues}), and indicates the index in the output value array where the error occurred.  For attribute get functions, errors are returned by \texttt{PWR_StatusPopError} in ascending order by index.}
	\returntype{int}
	\parameter{PWR_Status status}		{\pInput}	{The \texttt{PWR_Status} structure the user wishes to examine (iterate over).}
	\parameter{PWR_AttrAccessError* error}	{\pOutput}	{Pointer to a  \texttt{PWR_AttrAccessError} structure (see page \pageref{type:AttrAccessError}) to hold the status that is popped from the \texttt{PWR_Status} structure.}
	\returnval{PWR_RET_SUCCESS} {Upon SUCCESS.}
	\returnval{PWR_RET_EMPTY}   {Returned when all errors have been popped.}
	\returnval{PWR_RET_FAILURE} {Upon FAILURE.}
\end{prototype}

%==============================================================================%
%int PWR_StatusClear( PWR_Status status )
\begin{prototype}{StatusClear}
	\longdescription{The \texttt{PWR_StatusClear} function is provided to clear a previously used \texttt{PWR_Status} object created using \texttt{PWR_StatusCreate}, (see page \pageref{func:StatusCreate}) basically allowing reuse of the same structure if multiple calls are executed and examined in sequence.  Note, \texttt{PWR_Status} is an opaque handle, its backing definition is determined by the implementor (see \ref{sec:OpaqueTypes}).}
	\returntype{int}
	\parameter{PWR_Status status}	{\pInput}	{The \texttt{PWR_Status} structure the user wishes to clear (reuse).}
	\returnval{PWR_RET_SUCCESS} 	{Upon SUCCESS.}
	\returnval{PWR_RET_FAILURE} 	{Upon FAILURE.}
\end{prototype}

%==============================================================================%
%int PWR_ObjAttrGetValues( PWR_Obj object, int count, const PWR_AttrName attrs[], void* values, PWR_Time ts[], PWR_Status status );
\begin{prototype}{ObjAttrGetValues}
	\longdescription{The \texttt{PWR_ObjAttrGetValues} function is provided to get the value of multiple specified attributes listed in the \texttt{PWR_AttrName attrs[]} array from a single specified object -- \textbf{get multiple attribute values from a single object}.  The timestamps returned in the \texttt{PWR_Time ts[]} array should accurately represent, and correspond sequentially, with the time each value returned was measured.  If the function fails for one or more attributes, the \texttt{PWR_Status status} structure returned can be examined for additional information regarding the failure using \texttt{PWR_StatusPopError} (see page \pageref{func:StatusPopError}).}
	\returntype{int}
	\parameter{PWR_Obj object}		{\pInput}{The target object.}
	\parameter{int count}			{\pInput}{The number of elements in the \texttt{attrs[]}, \texttt{*values}, and \texttt{ts[]} arrays.}
	\parameter{const PWR_AttrName attrs[]}	{\pInput}{The array of target attributes to read. See section \ref{type:AttrName} for a list of available attributes.}
	\parameter{void* values}		{\pOutput}	{The array of values read, one value for each target attribute. This should point to caller-allocated storage of at least (\texttt{count * 8}) bytes. Upon success, the value read for attribute \texttt{attrs[i]} will be located at address (\texttt{values+(i*8)}).}
	\parameter{PWR_Time ts[]}		{\pOutput}{The array of timestamps, one timestamp for each value read. This should point to caller-allocated storage of at least (\texttt{count*sizeof(PWR_Time)}). Upon success, the timestamp of the value read for \texttt{attrs[i]} will be located at ts[i]. Pass in \texttt{NULL} if timestamps are not needed.}
	\parameter{PWR_Status status}		{\pOutput}{Upon \texttt{PWR_RET_FAILURE}, \texttt{status} contains information about each failure that occurred. Pass in \texttt{NULL} if failure information is not needed.}

	\returnval{PWR_RET_SUCCESS} 		{Upon SUCCESS, all operations succeeded.}
	\returnval{PWR_RET_FAILURE} 		{Upon FAILURE, one or more operations failed. Examine \texttt{PWR_Status* status} to determine the operations that failed. All other operations succeeded.}
\end{prototype}
%==============================================================================%
%int PWR_ObjAttrSetValues( PWR_Obj object, int count, const PWR_AttrName attrs[], const void* values, PWR_Status status );
\begin{prototype}{ObjAttrSetValues}
	\longdescription{The \texttt{PWR_ObjAttrSetValues} function is provided to set the value of multiple specified attributes in the  (\texttt{PWR_AttrName attrs[]}) array of a specified object -- \textbf{set multiple attribute values of a single object}.  If the function fails for one or more attributes, the \texttt{PWR_Status status} structure returned can be examined for additional information regarding the failure using \texttt{PWR_StatusPopError} (see page \pageref{func:StatusPopError}).}
	\returntype{int}

	\parameter{PWR_Obj object}		{\pInput}{The target object.}
	\parameter{int count}			{\pInput}{The number of elements in the \texttt{attrs[]} and \texttt{*values} arrays.}
	\parameter{const PWR_AttrName attrs[]}	{\pInput}{The array of target attributes to write. See section \ref{type:AttrName} for a list of available attributes.}
	\parameter{const void* values}		{\pInput}{The array of values to write, one value for each target attribute. The value to write to attribute \texttt{attrs[i]} is located at address (\texttt{values+(i*8)}).}
	\parameter{PWR_Status status}		{\pOutput}{Upon \texttt{PWR_RET_FAILURE}, \texttt{status} contains information about each failure that occurred. Pass in \texttt{NULL} if failure information is not needed.}
	\returnval{PWR_RET_SUCCESS} 		{Upon SUCCESS, all operations succeeded.}
	\returnval{PWR_RET_FAILURE} 		{Upon FAILURE, one or more operations failed. Examine \texttt{PWR_Status* status} to determine the operations that failed. All other operations succeeded.}
\end{prototype}
%==============================================================================%
%int PWR_ObjAttrIsValid( PWR_Obj object, PWR_AttrName attr );
\begin{prototype}{ObjAttrIsValid}
	\longdescription{The \texttt{PWR_ObjAttrIsValid} function is used to determine if a specified attribute (\texttt{PWR_AttrName attr}) is valid for the specified object.}
	\returntype{int}
	\parameter{PWR_Obj object}	{\pInput}{The object that the user is acting on.}
	\parameter{PWR_AttrName attr}	{\pInput}{The attribute the user wishes to confirm is valid for the specified object.  See the \texttt{PWR_AttrName} type definition in section \ref{type:AttrName}.}
	\returnval{PWR_RET_SUCCESS} {Upon SUCCESS.}
	\returnval{PWR_RET_FAILURE} {Upon FAILURE.}
\end{prototype}

%==============================================================================%
%int PWR_GrpAttrGetValue( PWR_Grp group, PWR_AttrName attr, void* values, PWR_Time ts[], PWR_Status status );
\begin{prototype}{GrpAttrGetValue}
	\longdescription{The \texttt{PWR_GrpAttrGetValue} function is provided to get the value of a single specified attribute (\texttt{PWR_AttrName attr}) from all the objects in a specified group (\texttt{PWR_Grp group}) -- \textbf{get a single attribute value from multiple objects}.  The timestamps returned in the \texttt{PWR_Time ts[]} array should accurately represent, and correspond sequentially, with the time each value returned was measured.  If the function fails for one or more attributes, the \texttt{PWR_Status status} structure returned can be examined for additional information regarding the failure using \texttt{PWR_StatusPopError} (see page \pageref{func:StatusPopError}).  \texttt{PWR_GrpAttrGetValue} will continue to attempt to gather values for the entire group, even if an error occurs for a subset of the members of that group.}
	\returntype{int}
	\parameter{PWR_Grp group}	{\pInput}{The target group.}
	\parameter{PWR_Attrgame attr}	{\pInput}{The target attribute to retrieve (get) from each object in the target group. See section \ref{type:AttrName} for a list of available attributes.}
	\parameter{void* values}	{\pOutput}{The array of attribute values retrieved, one value for each object in the target group. This should point to caller-allocated storage of at least (\texttt{PWR_GrpGetNumObjs() * 8}) bytes. Upon success, the value retrieved for the object at index \texttt{i} within the group will be located at address (\texttt{values+(i*8)}).}
	\parameter{PWR_Time ts[]}	{\pOutput}{The array of timestamps, one timestamp for each value retrieved. This should point to caller-allocated storage of at least (\texttt{PWR_GrpGetNumObjs()*sizeof(PWR_Time)}). Upon success, the timestamp of the value retrieved for the object at index \texttt{i} within the group will be located at ts[i]. Pass in \texttt{NULL} if timestamps are not needed.}
	\parameter{PWR_Status status}	{\pOutput}{Upon \texttt{PWR_RET_FAILURE}, \texttt{status} contains information about each failure that occurred. Pass in \texttt{NULL} if failure information is not needed.}

	\returnval{PWR_RET_SUCCESS} 	{Upon SUCCESS, all operations succeeded.}
	\returnval{PWR_RET_FAILURE} 	{Upon FAILURE, one or more operations failed. Examine \texttt{PWR_Status* status} to determine the operations that failed. All other operations succeeded.}
\end{prototype}
%=============================================================================%
%int PWR_GrpAttrSetValue( PWR_Grp group, PWR_AttrName attr, const void* value, PWR_Status status );
\begin{prototype}{GrpAttrSetValue}
	\longdescription{The \texttt{PWR_GrpAttrSetValue} function is provided to set the value of a single specified attribute (\texttt{PWR_AttrName attr}) of each object in a specified group -- \textbf{set a single attribute value on multiple objects}.  If the function fails for one or more attributes, the \texttt{PWR_Status status} structure returned can be examined for additional information regarding the failure using \texttt{PWR_StatusPopError} (see page \pageref{func:StatusPopError}).  \texttt{PWR_GrpAttrSetValue} will continue to attempt to set values for the entire group, even if an error occurs for a subset of the members of that group.}
	\returntype{int}
	\parameter{PWR_Grp group}	{\pInput}{The target group.}
	\parameter{PWR_AttrName attr}	{\pInput}{The target attribute to set for each object in the target group. See section \ref{type:AttrName} for a list of available attributes.}
	\parameter{const void* value}	{\pInput}{The pointer to a single 8 byte attribute value to set for each object in the target group.}
	\parameter{PWR_Status status}	{\pOutput}{Upon \texttt{PWR_RET_FAILURE}, \texttt{status} contains information about each failure that occurred. Pass in \texttt{NULL} if failure information is not needed.}
	\returnval{PWR_RET_SUCCESS} 	{Upon SUCCESS, all operations succeeded.} 
	\returnval{PWR_RET_FAILURE} 	{Upon FAILURE, one or more operations failed. Examine \texttt{PWR_Status* status} to determine the operations that failed. All other operations succeeded.}
\end{prototype}
%=============================================================================%
%int PWR_GrpAttrGetValues( PWR_Grp group, int count, const PWR_AttrName attrs[], void* values, PWR_Time ts[], PWR_Status status );
\begin{prototype}{GrpAttrGetValues}
	\longdescription{The \texttt{PWR_GrpAttrGetValues} function is provided to get the value of multiple specified attributes listed in the \texttt{PWR_AttrName attrs[]} array from each object in a specified group -- \textbf{get multiple attribute values from multiple objects}.  The timestamps returned in the \texttt{PWR_Time ts[]} array should accurately represent, and correspond sequentially, with the time each value returned was measured.  If the function fails for one or more attributes, the \texttt{PWR_Status status} structure returned can be examined for additional information regarding the failure using \texttt{PWR_StatusPopError} (see page \pageref{func:StatusPopError}).  \texttt{PWR_GrpAttrGetValues} will continue to attempt to gather values for the entire group, even if an error occurs for a subset of the members or attributes requested in the object group.}
	\returntype{int}
	\parameter{PWR_Grp group}		{\pInput}{The target group.}
	\parameter{int count}			{\pInput}{The number of elements in the \texttt{attrs[]} array.}
	\parameter{const PWR_AttrName attrs[]}	{\pInput}{he array specifying the set of target attributes to read for each object in the target group. See section \ref{type:AttrName} for a list of available attributes.}
	\parameter{void* values}		{\pOutput}{The array of attribute values retrieved. This should point to caller-allocated storage of at least (\texttt{PWR_GrpGetNumObjs()*count*8}) bytes.  Upon success, the value read for attribute \texttt{attrs[i]} for the object at index \texttt{j} within the group will be located at address (\texttt{values+(j*count*8)+(i*8)}).}
	\parameter{PWR_Time ts[]}{Output}{The array of timestamps, one timestamp for each value retrieved. This should point to caller-allocated storage of at least (\texttt{PWR_GrpGetNumObjs()*count*sizeof(PWR_Time)}). Upon success, the timestamp of the value retrieved for attribute \texttt{attrs[i]} for the object at index \texttt{j} within the group will be located at ts[(j*count)+i]. Pass in \texttt{NULL} if timestamps are not needed.}
	\parameter{PWR_Status status}		{\pOutput}{Upon \texttt{PWR_RET_FAILURE}, \texttt{status} contains information about each failure that occurred. Pass in \texttt{NULL} if failure information is not needed.}
	\returnval{PWR_RET_SUCCESS} {Upon SUCCESS, all operations succeeded.}
	\returnval{PWR_RET_FAILURE} {Upon FAILURE, one or more operations failed. Examine \texttt{PWR_Status* status} to determine the operations that failed. All other operations succeeded.}
\end{prototype}
%=============================================================================%
%int PWR_GrpAttrSetValues( PWR_Grp group, int count, const PWR_AttrName attrs[], const void* values, PWR_Status status );
\begin{prototype}{GrpAttrSetValues}
	\longdescription{The \texttt{PWR_GrpAttrSetValues} function is provided to set the value of multiple specified attributes listed in the  (\texttt{PWR_AttrName attrs[]}) array of each object in a specified group -- \textbf{set multiple attribute values on multiple objects}.  If the function fails for one or more attributes, the \texttt{PWR_Status status} structure returned can be examined for additional information regarding the failure using \texttt{PWR_StatusPopError} (see page \pageref{func:StatusPopError}).  \texttt{PWR_GrpAttrSetValues} will continue to attempt to set values for the entire group and requested attributes, even if an error occurs for a subset of the members or attributes of that object group.}
	\returntype{int}
	\parameter{PWR_Grp group}		{\pInput}{The target group.}
	\parameter{int count}			{\pInput}{The number of elements in the \texttt{attrs[]} and \texttt{*values} arrays.}
	\parameter{const PWR_AttrName attrs[]}	{\pInput}{The array specifying the set of target attributes to set for each object in the target group. See section \ref{type:AttrName} for a list of available at     tributes.}
	\parameter{const void* values}		{\pInput}{The array of attribute values to set for each object in the group. The value to write to attribute \texttt{attrs[i]} of each object is located at address (\texttt{values+(i*8)}).}
	\parameter{PWR_Status status}		{\pOutput}{Upon \texttt{PWR_RET_FAILURE}, \texttt{status} contains information about each failure that occurred. Pass in \texttt{NULL} if failure information is not needed.}
	\returnval{PWR_RET_SUCCESS} {Upon SUCCESS, all operations succeeded.}
	\returnval{PWR_RET_FAILURE} {Upon FAILURE, one or more operations failed. Examine \texttt{PWR_Status* status} to determine the operations that failed. All other operations succeeded.}
\end{prototype}
%=============================================================================%
%==============================================================================%
%=============================================================================%
%=============================================================================%
%=============================================================================%
%=============================================================================%
%=============================================================================%
%=============================================================================%
\section{Metadata Functions}\label{sec:METADATA}

The metadata functions provide an interface for getting more descriptive
information about an object or attribute, such as estimated measurement
accuracy or the list of valid values for a given attribute.  This
information is often useful for getting a better understanding of the
meaning of objects and attributes and how to interpret the values read
from attributes.  While most metadata is read-only information, some
metadata is potentially configurable, such as the underlying power
sampling rate used to calculate \texttt{PWR_ATTR_ENERGY} values.

Table~\ref{table:MasterMetadataTable} on page~\pageref{type:AttrStat} 
lists the available types of metadata. Not all of the metadata items
listed will be available for every object and attribute pair.  The exact
set is dependent on the capabilities of the underlying hardware and
Power API implementation. If a requested metadata item is not available
a \texttt{PWR_RET_NO_ATTRIB} error is returned at runtime.

The majority
of metadata items will require that both an object instance and
attribute name pair be specified, but a few may be defined for object
instances alone.  For example, the metadata strings \texttt{PWR\_MD\_NAME},
\texttt{PWR\_MD\_DESC}, and \texttt{PWR\_MD\_VENDOR\_INFO} may be
available for individual object instances, with no associated attribute
name specified.  In these cases, the attribute name requested should
be set to \texttt{PWR_ATTR_NOT_SPECIFIED}. One important use case for
these informational strings, especially the \texttt{PWR\_MD\_VENDOR\_INFO}
string, is for a Power API user to capture these strings with each run
to record configuration and provenance information.  For example, a
user may chose to log the \texttt{PWR\_MD\_VENDOR\_INFO} string for
the top-level platform or facility object in the output of each run.

The metadata interface consists of three functions. The
\texttt{PWR_ObjAttrGetMeta} and \texttt{PWR_ObjAttrSetMeta} functions allow
metadata values to be retrieved and set, respectively. The third function,
\texttt{PWR_MetaValueAtIndex}, provides a way to enumerate through an attribute's
list of available values.  This is useful for attributes that have a small, well-defined
set of discrete values (e.g., \texttt{PWR\_ATTR\_PSTATE}). It is expected that where 
a set of discrete values can be described in a logical order that the index ordering
is from smallest (lowest) to largest (highest) value. The remainder of
this section describes the metadata functions in more detail.

%
%==============================================================================%
%int PWR_ObjAttrGetMeta( PWR_Obj obj, PWR_AttrName attr, PWR_MetaName meta, void* value );
\begin{prototype}{ObjAttrGetMeta}
	\longdescription{The \texttt{PWR_ObjAttrGetMeta} function returns the requested metadata item for the specified object or object and attribute name pair.  The caller must allocate enough storage to hold the returned metadata value and pass a pointer to the storage in the \texttt{value} argument.  The required size can be determined by consulting the type column of Table~\ref{table:MasterMetadataTable}.  In the case of string metadata items (i.e., type \texttt{char *}), the required string length can be determined by getting the appropriate length metadata item, which is the original metadata name with the \texttt{_LEN} suffix added.  For example, the required string length for the \texttt{PWR_MD_VENDOR_INFO} string can be determined by retrieving the \texttt{PWR_MD_VENDOR_INFO_LEN} metadata item.}
	\parameter{PWR_Obj obj}		{\pInput}{The target object.}
	\parameter{iPWR_AttrName attr}	{\pInput}{The target attribute. See the \texttt{PWR_AttrName} type definition in Section \ref{type:AttrName} for the list of possible attributes. If object-only metadata is being requested, this argument should be set to \texttt{PWR_ATTR_NOT_SPECIFIED}.}
	\parameter{PWR_MetaName meta}	{\pInput}{The target metadata item to get. See the \texttt{PWR_MetaName} type definition in Section \ref{type:MetaName} for the list of possible metadata items, with detailed descriptions provided in Table~\ref{table:MasterMetadataTable}.}
	\parameter{void* value}		{\pOutput}{Pointer to the caller allocated storage to hold the value of the requested metadata item. See Table~\ref{table:MasterMetadataTable} for type information.}
	\returnval{PWR_RET_SUCCESS} 	{Upon SUCCESS.}
	\returnval{PWR_RET_NO_ATTRIB} 	{The attribute specified is not implemented.}
	\returnval{PWR_RET_NO_META} 	{The metadata specified is not implemented.}
	\returnval{PWR_RET_FAILURE} 	{Upon FAILURE.}
\end{prototype}

%
%==============================================================================%
%int PWR_ObjAttrSetMeta( PWR_Obj obj, PWR_AttrName attr, PWR_MetaName meta, const void* value );
\begin{prototype}{ObjAttrSetMeta}
	\longdescription{The \texttt{PWR_ObjAttrSetMeta} function sets the specified metadata item for the target object or object and attribute name pair.  The caller must pass a pointer to the new value for the specified metadata item in the \texttt{value} argument.  The required type for the value can be determined by consulting the type column of Table~\ref{table:MasterMetadataTable}.  In the case of string metadata items (i.e., type \texttt{char *}), the maximum string length can be determined by getting the appropriate length metadata item, which is the original metadata name with the \texttt{_LEN} suffix added.  For example, the maximum string length for the \texttt{PWR_MD_VENDOR_INFO} string can be determined by retrieving the \texttt{PWR_MD_VENDOR_INFO_LEN} metadata item.}
	\returntype{int}
	\parameter{PWR_Obj obj}		{\pInput}{The target object.}
	\parameter{PWR_AttrName attr}	{\pInput}{The target attribute. See the \texttt{PWR_AttrName} type definition in Section \ref{type:AttrName} for the list of possible attributes. If object-only metadatais being set, this argument should be set to \texttt{PWR_ATTR_NOT_SPECIFIED}.}
	\parameter{PWR_MetaName meta}	{\pInput}{The target metadata item to set. See the \texttt{PWR_MetaName} type definition in Section \ref{type:MetaName} for the list of possible metadata items, with detailed descriptions provided in Table~\ref{table:MasterMetadataTable}.}
	\parameter{const void* value}	{\pInput}{Pointer to the new value for the metadata item. See Table~\ref{table:MasterMetadataTable} for type information.}
	\returnval{PWR_RET_NO_ATTRIB} 	{The attribute specified is not implemented.}
	\returnval{PWR_RET_NO_META} 	{The metadata specified is not implemented.}
	\returnval{PWR_RET_READ_ONLY} 	{The metadata specified is not settable.}
	\returnval{PWR_RET_BAD_VALUE} 	{The value specified is not valid.}
	\returnval{PWR_RET_FAILURE} 	{Upon FAILURE.}
\end{prototype}

%
%==============================================================================%
%int PWR_MetaValueAtIndex( PWR_Obj obj, PWR_AttrName attr, unsigned int index, void* value, char* value_str ); \end{lstlisting}
\begin{prototype}{MetaValueAtIndex}
	\longdescription{The \texttt{PWR_MetaValueAtIndex} function allows the available values for a given attribute to be enumerated.  It is assumed that the set of valid values is static and has size equal to the value returned by the \texttt{PWR_MD_NUM} metadata item.  Once the value of \texttt{PWR_MD_NUM} is known, \texttt{PWR_MetaValueAtIndex()} can be called repeatedly with index from 0 to \texttt{PWR_MD_NUM} - 1 to retrieve the list of valid values for the target attribute.  Each call will return the value at the specified index as well as a human-readable string representing the value in human readable format.  If an attribute is not enumerable, then \texttt{PWR_MD_NUM} will return 0.  In general any attribute that does not have a small set of discrete valid values will return 0 when \texttt{PWR_MD_NUM} is requested, to indicate that the attribute is not enumerable.}
	\returntype{int}
	\parameter{PWR_Obj obj}		{\pInput}{The target object.}
	\parameter{PWR_AttrName attr}	{\pInput}{The target attribute. See the \texttt{PWR_AttrName} type definition in Section \ref{type:AttrName} for the list of possible attributes.}
	\parameter{unsigned int index}	{\pInput}{The index of the metadata item value to look up. The \texttt{PWR\_MD\_NUM} metadata item returns the number of possible values, indexed from 0 to \texttt{PWR\_MD\_NUM} - 1.}
	\parameter{void* value}		{\pOutput}{Pointer to the caller allocated storage to hold the value of the requested metadata item value. See Table~\ref{table:MasterMetadataTable} for type inform     ation. The storage must be sized appropriately for the metadata value type. If the value is not required, this argument should be set to NULL.}
	\parameter{char* value\_str}	{\pOutput}{Pointer to the caller allocated storage to hold the human-readable C-style NULL-terminated ASCII string representing the metadata item value. The storage passed in must have size in bytes of at least the value returned by the \texttt{PWR\_MD\_VALUE\_LEN} metadata item. If the string representation is not required this argument should be set to NULL.}
	\returnval{PWR_RET_SUCCESS} 	{Upon SUCCESS.}
	\returnval{PWR_RET_NO_ATTRIB} 	{The attribute specified is not implemented.}
	\returnval{PWR_RET_BAD_INDEX} 	{The index specified is not valid.}
	\returnval{PWR_RET_FAILURE} 	{Upon FAILURE.}
\end{prototype}

%==============================================================================%
%=============================================================================%
%=============================================================================%
%=============================================================================%
%=============================================================================%
%=============================================================================%
%=============================================================================%

\section{Statistics Functions}\label{sec:StatisticsFunctions}
%

The statistics functions provide an interface to generate statistics related to specific attributes of an object or group. 
The interface allows for generating statistics somewhat real-time or mining historic statistics, assuming that the necessary data is retained.
The interface for collecting historic statistics is much more straight forward and can be accomplished with a single call, \texttt{PWR_ObjGetStat} for a single object and \texttt{PWR_GrpGetStats} for a group of objects. 
%The \texttt{PWR_ObjGetStat} and \texttt{PWR_GrpGetStats} functions are used to collect historic statistic information.
The interface for collecting real-time statistics is designed to interface with hardware or layers of software that require a notification of when information collection should begin and when it can be terminated. 
The requested statistic can then be mined for this window of time, even while the window remains open. 
The sequence of calls for mining real-time statistics is as follows.
The user creates a statistic object using the \texttt{PWR_ObjCreateStat} call when collecting a statistic on a single object or the \texttt{PWR_GrpCreateStat} call when the statistic is to be collected on a group of objects.
Basically, a tuple of information is provided, an object or group, the attribute (\texttt{PWR_ATTR_POWER} for example, see page \pageref{type:AttrName}) that the user would like the statistic for and the statistic (\texttt{PWR_ATTR_STAT_AVG} for example, see page \pageref{type:AttrStat}).
Notice that the statistic to be collected is part of the required parameters for creating a statistics object, while it is provided at the time of retrieval when collecting historic statistics.
The reason for this approach is that the underlying hardware or software layer needs to understand what information to start collecting to support the requested statistic. 
Buffers are typically a limiting factor in the capabilities that can be supported by an implementation. 
Requiring an implementation to collect the data necessary for any potential statistic could require a great deal of space.
Once a statistics object is created (for an object or a group) the user indicates the beginning of the window by calling \texttt{PWR_StatStart}.
Once \texttt{PWR_StatStart} is called the user can retrieve the statistic information associated with the statistics object by calling \texttt{PWR_StatGetValue}, when the statistics object was created for a single object, or \texttt{PWR_StatGetValues}, when the statistics object was created for a group of objects.
The start time is always the time that the user calls \texttt{PWR_StatStart} on the statistics object.
The user can call \texttt{PWR_StatGetValue} or \texttt{PWR_StatGetValues} as many times as they wish prior to calling \texttt{PWR_StatStop}.
If \texttt{PWR_StatStop} has not been called, the stop time is the time the user calls \texttt{PWR_StatGetValue} or \texttt{PWR_StatGetValues}.
Once \texttt{PWR_StatStop} has been called the stop time if fixed for that statistics object. 
Essentially, the implementation at this time has everything it needs to calculate the return value or values for \texttt{PWR_StatGetValue} or \texttt{PWR_StatGetValues}.
The user is responsible for checking the start and stop times returned along with the statistics value. 
The start and stop times may be different for two reasons.
In the normal case, the implementation is required to return start and stop times that accurately represent when the actual data was sampled that was used in calculting the statistics value.
As such, the returned values could differ from the times set by the real-time statistics functions.
In the abnormal case, the start time, and possibly the stop time, could differ more significantly from the times \texttt{PWR_StatStart} and \texttt{PWR_StatStop} were called or the stop time determined by calling either of the \texttt{PWR_StatGetValue} or \texttt{PWR_StatGetValues} functions before \texttt{PWR_StatStop} has been called. 
If this occurs, due to a resource exhaustion issue for example, the implementation is required to either return a failure or return a statistics value and the accurate time values representing the statistics value returned along with a warning indicating that the time window has been truncated. 
A truncated time-frame is still required to as closely as possible represent the data collection time the statistic is generated based on.
It is then up to the user to determine if the value returned is useful or not.
Statistics objects can be re-used by calling \texttt{PWR_StatClear}, which indicates to the implementation that any data retained associated with the statistic object can be released. 
To begin another statistics window the user repeats the process just outlined.
When the user is done with a statistics object they should call \texttt{PWR_StatDestroy}.

%==============================================================================%
%int PWR_ObjGetStat( PWR_Obj object, PWR_AttrName name, PWR_AttrStat statistic, PWR_TimePeriod* statTime, double* value ); 
\begin{prototype}{ObjGetStat}
	\longdescription{The \texttt{PWR_ObjGetStat} function is used to retrieve a historic statistic using an object, attribute, statistic tuple.  Note that the \texttt{PWR_ObjGetStat} call operates on single objects only, not groups of objects.  The \texttt{PWR_ObjGetStat} is a standalone call is used for historic data collection only.  To retrieve a statistic from a group of objects, the \texttt{PWR_GrpGetStats} call on page \pageref{func:GrpGetStats} should be used.}
	\parameter{PWR_Obj object}		{\pInput}{The object to collect the statistic for (part of the object, attribute statistic triple}
	\parameter{PWR_AttrName name}		{\pInput}{The attribute to act on, see the \texttt{PWR_AttrName} type definition in section \ref{type:AttrName}.}
	\parameter{PWR_AttrStat statistic}	{\pInput}{ified attribute, see \texttt{PWR_AttrStat} type definition in section \ref{sec:StatisticTypeDefinitions}.}
	\parameter{PWR_TimePeriod* statTime}	{\pInputOutput}{Time structure that initially must contain the times (start, stop and instant if appropriate) requested by the user (Input) and the times, possibly      different, representing the period of the statistic data returned (Output), see page \pageref{type:TimePeriod}.}
	\parameter{double* value}		{\pOutput}{pointer to space (double) to store the statistic}
	\returnval{PWR_RET_SUCCESS} {Upon SUCCESS.}
	\returnval{PWR_RET_FAILURE} {Upon FAILURE.}
\end{prototype}

%==============================================================================%
%int PWR_GrpGetStats( PWR_Grp group, PWR_AttrName name, PWR_AttrStat statistic, PWR_TimePeriod* statTime double values[], PWR_TimePeriod statTimes[] ); 
\begin{prototype}{GrpGetStats}
	\longdescription{The \texttt{PWR_GrpGetStats} function is used to retrieve historic statistic for a group of objects.  Each object in the group is combined with the attribute and statistic specified to form the object, attribute, statistic tuple.  Note that the \texttt{PWR_GrpGetStats} call operates on one or more objects in a group.  The \texttt{PWR_GrpGetStats} is a standalone call is used for historic data collection only.  To retrieve a statistic from a single object, the \texttt{PWR_ObjGetStat} call on page \pageref{func:ObjGetStat} should be used.}
	\returntype{int}
	\parameter{PWR_Grp group}		{\pInput}{The group to collect the statistic for. Each object in the group forms the object, attribute, statistic triple.}
	\parameter{PWR_AttrName name}		{\pInput}{The attribute to act on, see the \texttt{PWR_AttrName} type definition in section \ref{type:AttrName}.}
	\parameter{PWR_AttrStat statistic}	{\pInput}{The desired statistic for the specified attribute, see \texttt{PWR_AttrStat} type definition in section \ref{sec:StatisticTypeDefinitions}.}
	\parameter{PWR_TimePeriod* statTime}	{\pInput}{Time structure that must contain the times (start, stop and instant if appropriate) requested by the user. Note this is Input only, see page \pageref{type: TimePeriod}.}
	\parameter{double values[]}		{\pOutput}{Space (of double) allocated by user to store an array of statistic values}
	\parameter{PWR_TimePeriod statTimes[]}	{\pOutput}{Space allocated by user to hold an array of time structures representing the actual times associated with each statistic value returned in values[], see      page \pageref{type:TimePeriod}.}
	\returnval{PWR_RET_SUCCESS} 		{SUCCESS.}
	\returnval{PWR_RET_FAILURE} 		{Upon FAILURE.}
\end{prototype}

%==============================================================================%
%int PWR_ObjCreateStat( PWR_Obj object, PWR_AttrName name, PWR_AttrStat statistic, PWR_Stat* stat); 
\begin{prototype}{ObjCreateStat}
	\longdescription{The \texttt{PWR_ObjCreateStat} function is used to create a statistics object that will be used for real-time statistics gathering operations for a single object.  The user specifies the \textbf{object}, attribute, statistic tuple that all subsequent requests using the statistics object created will be based on.  Note, this call is not used for historic statistic gathering, see \texttt{PWR_ObjGetStat} on page \pageref{func:ObjGetStat} and \texttt{PWR_GrpGetStats} on page \pageref{func:GrpGetStats}.}
	\returntype{int}
	\parameter{PWR_Obj object}		{\pInput}{The object to act on.}
	\parameter{PWR_AttrName name}		{\pInput}{The attribute to act on, see the \texttt{PWR_AttrName} type definition in section \ref{type:AttrName}.}
	\parameter{PWR_AttrStat statistic}	{\pInput}{The desired statistic for the specified attribute, see \texttt{PWR_AttrStat} type definition in section \ref{sec:StatisticTypeDefinitions}.}
	\parameter{PWR_Stat* stat}		{\pOutput}{The stat for the object, attribute, statistic triple specified.}
	\returnval{PWR_RET_SUCCESS} 		{Upon SUCCESS, valid stat is created.}
	\returnval{PWR_RET_FAILURE} 		{Upon FAILURE.}
\end{prototype}
%==============================================================================%
%int PWR_GrpCreateStat( PWR_Grp group, PWR_AttrName name, PWR_AttrStat statistic, PWR_Stat* stat); 
\begin{prototype}{GrpCreateStat}
	\longdescription{The \texttt{PWR_GrpCreateStat} function is used to create a statistics object that will be used for real-time statistics gathering operations for a group of objects.  The user specifies the \textbf{group}, attribute, statistic tuple that all subsequent requests using the statistics object created will be based on.  Note, this call is not used for historic statistic gathering, see \texttt{PWR_ObjGetStat} on page \pageref{func:ObjGetStat} and \texttt{PWR_GrpGetStats} on page \pageref{func:GrpGetStats}.}
	\returntype{int}
	\parameter{PWR_Grp group}		{\pInput}{The group to act on.}
	\parameter{PWR_AttrName name}		{\pInput}{The attribute to act on, see the \texttt{PWR_AttrName} type definition in section \ref{type:AttrName}.}
	\parameter{PWR_AttrStat statistic}	{\pInput}{The desired statistic for the specified attribute, see \texttt{PWR_AttrStat} type definition in section \ref{sec:StatisticTypeDefinitions}.}
	\parameter{PWR_Stat* stat}		{\pOutput}{The stat for the group, attribute, statistic triple specified.}
	\returnval{PWR_RET_SUCCESS} 		{Upon SUCCESS, valid stat is created.}
	\returnval{PWR_RET_FAILURE} 		{Upon FAILURE.}
\end{prototype}
%==============================================================================%
%int PWR_StatStart( PWR_Stat statObj ); 
\begin{prototype}{StatStart}
	\longdescription{The \texttt{PWR_StatStart} function is used to indicate to a device or software layer to start the window of time that the statistic requested will be calculated over.  The \texttt{PWR_StatStart} function is used for real-time statistics gathering only.}
	\returntype{int}
	\parameter{PWR_Stat statObj}	{\pInput}{The statistics object to begin collecting the specified statistic for (specified in \texttt{PWR_ObjCreateStat} or \texttt{PWR_GrpCreateStat}).}
	\returnval{PWR_RET_SUCCESS} {Upon SUCCESS.}
	\returnval{PWR_RET_FAILURE} {Upon FAILURE.}
\end{prototype}
%==============================================================================%
%int PWR_StatStop( PWR_Stat statObj ); 
\begin{prototype}{StatStop}
	\longdescription{The \texttt{PWR_StatStop} function is used to indicate to a device or software layer to stop the window of time that the statistic requested will be calculated over.  The \texttt{PWR_StatStop} function is used for real-time statistics gathering only.}
	\returntype{int}
	\parameter{PWR_Stat statObj}{\pInput}{The statistics object to stop collecting the specified statistic for (specified in \texttt{PWR_ObjCreateStat} or \texttt{PWR_GrpCreateStat}).}
	\returnval{PWR_RET_SUCCESS} {Upon SUCCESS.}
	\returnval{PWR_RET_FAILURE} {Upon FAILURE.}
\end{prototype}
%==============================================================================%
%int PWR_StatClear( PWR_Stat statObj ); 
\begin{prototype}{StatClear}
	\longdescription{The \texttt{PWR_StatClear} function is used to indicate to a device or software layer to clear or reset the window of time that the statistic requested will be calculated over.  The clear effectively restarts the window, so there is no need to call \texttt{PWR_StatStart} again.  The \texttt{PWR_StatClear} function is used for real-time statistics gathering only.}
	\returntype{int}
	\parameter{PWR_Stat statObj}	{\pInput}{The statistics object to clear (effectively reset) for the specified statistic (specified in \texttt{PWR_ObjCreateStat} or \texttt{PWR_GrpCreateStat}).}
	\returnval{PWR_RET_SUCCESS} {Upon SUCCESS.}
	\returnval{PWR_RET_FAILURE} {Upon FAILURE.}
\end{prototype}
%==============================================================================%
%int PWR_StatGetValue( PWR_Stat statObj, double* value, PWR_TimePeriod* statTimes ); 
\begin{prototype}{StatGetValue}
	\longdescription{The \texttt{PWR_StatGetValue} function is used to retrieve the statistic and related time stamp information from the statistics object created using \texttt{PWR_ObjCreateStat}.  Note that the \texttt{PWR_StatGetValue} call operates on single objects only, not groups of objects.  The start time for the window the statistic is calculated over is set by calling \texttt{PWR_StatStart}.  The stop time is set by either calling this function, \texttt{PWR_StatGetValue}, or set for the statistics object by calling \texttt{PWR_StatStop}.  Each time \texttt{PWR_StatGetValue} is called prior to calling \texttt{PWR_StatStop} the time \texttt{PWR_StatGetValue} is called is used as the stop time for the statistics calculation.  From the specification standpoint, there is no limit to how often \texttt{PWR_StatGetValue} can be called.  The start, stop and, depending on the statistic requested, the instant time values returned should as accurately as possible represent the time-stamps of the data used in the statistics value returned.  The \texttt{PWR_StatGetValue} function is used for real-time statistics gathering only.  If a single value return is desired for a group of objects, the \texttt{PWR_StatGetReduce} call on page \pageref{func:StatGetReduce} should be used.}
	\returntype{int}
	\parameter{PWR_Stat statObj}		{\pInput}{The statistics object to collect the statistic for (the object, attribute stat triple is specified in \texttt{PWR_ObjCreateStat}).}
	\parameter{double* value}		{\pOutput}{pointer to space (double) to store the statistic}
	\parameter{PWR_TimePeriod* statTimes}	{\pOutput}{Time structure that contains the timestamps pertinent to the specific statistic value, see page \pageref{type:TimePeriod}.}
	\returnval{PWR_RET_SUCCESS} 		{Upon SUCCESS.}
	\returnval{PWR_RET_FAILURE} 		{Upon FAILURE.}
	\returnval{PWR_RET_WARN_TRUNC} 		{When the time window has been truncated by the implementation, start and stop times may differ significantly from those set by the interface.} 
\end{prototype}
 %==============================================================================%
%int PWR_StatGetValues( PWR_Stat statObj, double values[], PWR_TimePeriod statTimes[] ); 
\begin{prototype}{StatGetValues}
	\longdescription{The \texttt{PWR_StatGetValues} function is used to retrieve the statistic and related time stamp information from the statistics object(s) created using \texttt{PWR_GrpCreateStat}.  Note that the \texttt{PWR_StatGetValues} call operates on one or more objects in the group specified in the \texttt{PWR_GrpCreateStat} call.  The start time for the window the statistic is calculated over is set by calling \texttt{PWR_StatStart}.  The stop time is set by either calling this function, \texttt{PWR_StatGetValues}, or set for the statistics object by calling \texttt{PWR_StatStop}.  Each time \texttt{PWR_StatGetValues} is called prior to calling \texttt{PWR_StatStop} the time \texttt{PWR_StatGetValues} is called is used as the stop time for the statistics calculation.  From the specification standpoint, there is no limit to how often \texttt{PWR_StatGetValues} can be called.  The start, stop and, depending on the statistic requested, the instant time values for each individual object returned (in the \texttt{Output} \texttt{PWR_TimePeriod} structure) should as accurately as possible represent the time-stamps of the data used in the statistics values returned.  The \texttt{PWR_StatGetValues} function is used for real-time statistics gathering only.  If a single value return is desired for a group of objects, the \texttt{PWR_StatGetReduce} call on page \pageref{func:StatGetReduce} should be used.}
	\returntype{int}
	\parameter{PWR_Stat statObj}		{\pInput}	{The statistics object to collect the statistic for (the group, attribute stat triple is specified in \texttt{PWR_GrpCreateStat}).}
	\parameter{double values[]}		{\pOutput}	{Space allocated by user to hold array of values (statistics).}
	\parameter{PWR_TimePeriod statTimes[]}	{\pOutput}	{Space allocated by user to hold array of time structures that contains the timestamps pertinent to each specific statistic value, see page \pageref{type     :TimePeriod}}
	
	\returnval{PWR_RET_SUCCESS} 		{Upon SUCCESS.}
	\returnval{PWR_RET_FAILURE} 		{Upon FAILURE.}
	\returnval{PWR_RET_WARN_TRUNC} 		{When the time window has been truncated by the implementation, start and stop times may differ significantly from those set by the interface.} 
\end{prototype}
%==============================================================================%
%int PWR_StatGetReduce( PWR_Stat statObj, PWR_AttrStat reduceOp, int* index, double* result, PWR_Time* instant);
\begin{prototype}{StatGetReduce}
	\longdescription{The \texttt{PWR_StatGetReduce} function is used to reduce a set of per-object statistics down into a single returned value.  The inputs are a \texttt{PWR_Stat} object, and a reduction operation.  The reduction operation can be thought of as occurring in two phases.  In the first phase, a statistic is calculated for each object associated with the input \texttt{PWR_Stat}, one statistic value per object.  The objects, target attribute, and desired statistic to calculate are specified when the \texttt{PWR_Stat} is created.  In the second phase, the set of statistic values calculated in the first phase are combined into a single result value.  How this occurs is determined by the reduction operation that was specified by the caller.  For example, the \texttt{PWR_ATTR_STAT_AVG} reduction operation returns the average of the per-object statistics calculated in the first phase.  The start time for the window the statistic is calculated over is set by calling \texttt{PWR_StatStart}.  The stop time used for the statistics calculated in the first phase are based on the time this function is called, or set for the statistics object from a previous call to  \texttt{PWR_StatStop}.  \texttt{PWR_StatGetReduce} is provided such that optimizations may be possible when gathering the statistics of each member in a group of objects.  An example of such an operation would be calculating an average, where gathering the values is done through a tree topology overlay network, where averages can be calculated at each parent of multiple children in the tree.  Note that the implementation of \texttt{PWR_StatGetReduce} can be done in its more simplistic form by calling \texttt{PWR_StatGetValues} and performing the required operation on the returned set of values to return the requested reduction operation.}
	\returntype{int}
	\parameter{PWR_Stat statObj}		{\pInput}{The statistics object to collect the statistic for (the object group, attribute, stat triple is specified in \texttt{PWR_GrpCreateStat}).}
	\parameter{PWR_AttrStat reduceOp}	{\pInput}{The reduction operation to perform.}
	\parameter{int* index}			{\pOutput}{The index of the object in the statObj's associated object group that provided the reduction result. This value is only set for reduction operations where it makes sense, such as PWR_ATTR_STAT_MIN and PWR_ATTR_STAT_MAX.}
	\parameter{double* result}		{\pOutput}{The result of the reduction operation, which is always a single double value.}
	\parameter{PWR_Time* instant}		{\pOutput}{For statistics where a point in time that the value occured is valid (e.g. max and min), this is the timestamp when that value was observed.}
	\returnval{PWR_RET_SUCCESS} 	{Upon SUCCESS.}
	\returnval{PWR_RET_FAILURE} 	{Upon FAILURE.}
	\returnval{PWR_RET_WARN_TRUNC} 	{When the time window has been truncated by the implementation.}
\end{prototype}
%==============================================================================%
%int PWR_GrpGetReduce( PWR_Grp group, PWR_AttrName name, PWR_AttrStat statistic, PWR_AttrStat reduceOp, PWR_TimePeriod statTime, int* index, double* result, PWR_TimePeriod* resultTime);
\begin{prototype}{GrpGetReduce}
	\longdescription{The \texttt{PWR_GrpGetReduce} function is used to reduce a set of per-object statistics down into a single returned value.  Unlike \texttt{PWR_StatGetReduce} that is used for real time statistics gathering, \texttt{PWR_GrpGetReduce} is meant to gather statistics for historical data.  Therefore, this call is much like the \texttt{PWR_GrpGetStats} function, with an added reduction.  The inputs are a \texttt{PWR_Grp} object, an attribute, a statistic, a reduction operation and a time period.  The reduction operation can be thought of as occurring in two phases.  In the first phase, a statistic is calculated for each object associated with the input group, one statistic value per object.  The objects, target attribute, and desired statistic to calculate are specified as inputs to this function..  In the second phase, the set of statistic values calculated in the first phase are combined into a single result value.  How this occurs is determined by the reduction operation that was specified by the caller.  For example, the \texttt{PWR_ATTR_STAT_AVG} reduction operation returns the average of the per-object statistics calculated in the first phase.  Upon success, the returned \texttt{PWR_TimePeriod} structure will have its time fields set to the timestamps that are most closely associated with the result of the reduction operation.  For certain reduction operations, some timestamps in the returned \texttt{PWR_TimePeriod} may not be valid output.  For example, in the case of a averaging reduction, an associated ``instant'' timestamp is not a useful value.  For minimum and maximum operations, the ``instant'' timestamp is useful and will represent the time at which the maximum or minimum was observed.  In all cases the start and stop timestamps in the \texttt{PWR_TimePeriod} will represent the time window over which the the value was calculated.  \texttt{PWR_GrpGetReduce} is provided such that optimizations may be possible when gathering the statistics of each member in a group of objects.  An example of such an operation would be calculating an average, where gathering the values is done through a tree topology overlay network, where averages can be calculated at each parent of multiple children in the tree.  Note that the implementation of \texttt{PWR_GrpGetReduce} can be done in its more simplistic form by calling \texttt{PWR_GrpGetStats} and performing the required operation on the returned set of values to return the requested reduction operation.}
	\returntype{int}
	\parameter{PWR_Grp group}		{\pInput}{The group to collect the statistic for. Each object in the group forms the object, attribute, statistic triple.}
	\parameter{PWR_AttrName name}		{\pInput}{The attribute to act on, see the \texttt{PWR_AttrName} type definition in section \ref{type:AttrName}.}
	\parameter{PWR_AttrStat statistic}	{\pInput}{The desired statistic for the specified attribute, see \texttt{PWR_AttrStat} type definition in section \ref{sec:StatisticTypeDefinitions}.}
	\parameter{PWR_AttrStat reduceOp}	{\pInput}{The reduction operation to perform.}
	\parameter{PWR_TimePeriod statTime}	{\pInput}{Time structure that must contain the times (start, stop and instant if appropriate) requested by the user. Note this is Input only, see page \pageref{type:TimePeriod}.}
	\parameter{int* index}			{\pOutput}{The index of the object in the statObj's associated object group that provided the reduction result. This value is only set for reduction operations where it makes sense, such as PWR_ATTR_STAT_MIN and PWR_ATTR_STAT_MAX.}
	\parameter{double* result}		{\pOutput}{The result of the reduction operation, which is always a single double value.}
	\parameter{PWR_TimePeriod* resultTime}	{\pOutput}{The time period that the results are valid for. Note that this may diverge from the input time period if results for the exact time period are not available. This time period will also contain the instant that the statistic was observed for cases where this makes sense, such as PWR_ATTR_STAT_MIN and PWR_ATTR_STAT_MAX.}
	\returnval{PWR_RET_SUCCESS} 	{Upon SUCCESS.}
	\returnval{PWR_RET_FAILURE} 	{Upon FAILURE.}
	\returnval{PWR_RET_WARN_TRUNC} 	{When the time window has been truncated by the implementation.} 
\end{prototype}

%==============================================================================%
%int PWR_StatDestroy( PWR_Stat statObj ); 
\begin{prototype}{StatDestroy}
	\longdescription{The \texttt{PWR_StatDestroy} function is used to destroy (clean up) the statistics pointer created using \texttt{PWR_ObjCreateStat} or \texttt{PWR_GrpCreateStat}.}
	\returntype{int}
	\parameter{PWR_Stat statObj}{\pInput}{The statistics object to destroy (clean up)}
	\returnval{PWR_RET_SUCCESS} {Upon SUCCESS.}
	\returnval{PWR_RET_FAILURE} {Upon FAILURE.}
\end{prototype}
%==============================================================================%
%=============================================================================%
%=============================================================================%
%=============================================================================%
%=============================================================================%
%=============================================================================%
%=============================================================================%

\section{Version Functions}\label{sec:VersionFunctions}

The \texttt{PWR_GetMajorVersion} and \texttt{PWR_GetMinorVersion} functions are used to get the major and minor portions of the specification version supported by the implementation. 
Users can make decisions regarding available functionality based on the version number supported.

%==============================================================================%
%int PWR_GetMajorVersion( ); 
\begin{prototype}{GetMajorVersion}
	\longdescription{The \texttt{PWR_GetMajorVersion} function is used to get the major version number portion of the version number of the specification supported by the implementation.}
	\returntype{int}
	\returnval{int} 		{Upon SUCCESS, integer representation of major portion of version number}
	\returnval{PWR_RET_FAILURE} 	{Upon FAILURE}
\end{prototype}
%==============================================================================%
%int PWR_GetMinorVersion( ); 
\begin{prototype}{GetMinorVersion}
	\longdescription{The \texttt{PWR_GetMinorVersion} function is used to get the minor version portion of the version number of the specification supported by the implementation.}
	\returntype{int}
	\returnval{int}             {Upon SUCCESS, integer representation of minor portion of version number.}
	\returnval{PWR_RET_FAILURE} {Upon FAILURE.}
\end{prototype}

%==============================================================================%
%=============================================================================%
%=============================================================================%
%=============================================================================%
%=============================================================================%
%=============================================================================%
%=============================================================================%
\section{Big List of Attributes}\label{sec:BLOA}

The following is the master list of Attributes available to the user.
The attributes valid for specific interfaces are listed in the appropriate section in Chapter~\ref{chap:Interfaces}. 

\begin{attributetable}{Complete list of all supported attributes}{table:MasterAttributeTable}
\aPstateDesc
\aCstateDesc
\aCstateLimitDesc
\aSstateDesc
\aCurrentDesc
\aVoltageDesc
\aPowerDesc
\aMinPowerDesc
\aMaxPowerDesc
\aFreqDesc
\aFreqLimitMinDesc
\aFreqLimitMaxDesc
\aEnergyDesc
\aTempDesc
\aOSIdDesc
\aThrottledIdDesc
\aThrottledCountIdDesc
\aGovDesc
\end{attributetable}

%==============================================================================%
%=============================================================================%
%=============================================================================%
%=============================================================================%
%=============================================================================%
%=============================================================================%
%=============================================================================%
\section{Big List of Metadata}\label{sec:BLOM}

%\laros{Table needs work. Reduced Metadata to MD but the names are still really long? Probably could use an abbreviation for same as attribute.}
\begin{attributetable}{Complete List of All Metadata Names}{table:MasterMetadataTable}
\mNum
\mMin
\mMax
\mPrecision
\mAccuracy
\mUpdateRate
\mSampleRate
\mTimeWindow
\mTSLatency
\mTSAccuracy
\mMaxLen
\mNameLen
\mName
\mDescLen
\mDesc
\mValueLen
\mVendorInfoLen
\mVendorInfo
\mMeasureMethod
\end{attributetable}

%%%\begin{center}
%%%%\begin{longtable}{ | p{5.0cm} | p{1.2cm} | p{1.6cm} | p{6.2cm} |}
%%%\begin{longtable}{ | p{5.0cm} | p{1.0cm} | p{2.0cm} | p{7.0cm} |}
%%%\caption{Complete List of All Metadata Names}\label{table:MasterMetadataTable}\\
%%%
%%%\hline 
%%%\multicolumn{1}{|c|}{\textbf{Metadata}} & \multicolumn{1}{c|}{\textbf{Set}}        & \multicolumn{1}{c|}{\textbf{Type}} & \multicolumn{1}{c|}{\textbf{Description}} \\
%%%\multicolumn{1}{|c|}{  }                                    & \multicolumn{1}{c|}{\textbf{and/or}}   & \multicolumn{1}{c|}{(SaA = Same}                    & \multicolumn{1}{c|}{  }           \\     
%%%\multicolumn{1}{|c|}{  }                                    & \multicolumn{1}{c|}{\textbf{Get}}   & \multicolumn{1}{c|}{as Attribute)}                    & \multicolumn{1}{c|}{  }           \\                                 
%%%\hline
%%%\hline
%%%\endfirsthead
%%%
%%%\multicolumn{4}{c}%
%%%{{ \tablename\ \thetable{} -- continued from previous page}} \\
%%%\hline
%%%\multicolumn{1}{|c|}{\textbf{Metadata}} & \multicolumn{1}{c|}{\textbf{Set}}        & \multicolumn{1}{c|}{\textbf{Type}} & \multicolumn{1}{c|}{\textbf{Description}} \\
%%%\multicolumn{1}{|c|}{  }                                    & \multicolumn{1}{c|}{\textbf{and/or}}   & \multicolumn{1}{c|}{(SaA = Same}                    & \multicolumn{1}{c|}{  }           \\     
%%%\multicolumn{1}{|c|}{  }                                    & \multicolumn{1}{c|}{\textbf{Get}}   & \multicolumn{1}{c|}{as Attribute)}                    & \multicolumn{1}{c|}{  }           \\                                 
%%%\hline
%%%\hline
%%%\endhead
%%%
%%%\hline \multicolumn{4}{|r|}{{Continued on next page}} \\ \hline
%%%\endfoot
%%%
%%%\hline \hline
%%%\endlastfoot  
%%%
%%%  \hline
%%%\texttt{PWR\_MD\_NUM}                      & Get     & uint64_t   & Number of values supported. This is only relevant for attributes with a discrete set of values (e.g., \texttt{PWR\_ATTR\_PSTATE}). Other attributes return 0. \\
%%%  \hline
%%%\texttt{PWR\_MD\_MIN}                      & Get     & SaA        & Minimum value supported. \\
%%%  \hline
%%%\texttt{PWR\_MD\_MAX}                      & Get     & SaA        & Maximum value supported. \\
%%%  \hline
%%%\texttt{PWR\_MD\_PRECISION}                & Get     & uint64_t   & Number of significant digits in values. \\
%%%  \hline
%%%\texttt{PWR\_MD\_ACCURACY}                 & Get     & double     & Estimated percent error +/- of measured vs. actual values. \\
%%%  \hline
%%%\texttt{PWR\_MD\_UPDATE\_RATE}             & Set/Get & double     & Rate values become visible to user, in updates per second. Getting or setting a value at a rate higher than this is not useful. \\
%%%  \hline
%%%\texttt{PWR\_MD\_SAMPLE\_RATE}             & Set/Get & double     & Rate of underlying sampling, in samples per second. This is only relevant for values derived over time (e.g., \texttt{PWR\_ATTR\_ENERGY}). \\
%%%\hline
%%%\texttt{PWR\_MD\_TIME\_WINDOW}             & Set/Get & \texttt{PWR\_Time}  & The time window used to calculate the value returned or relevant to an attribute. For example, the ``instantaneous'' \texttt{PWR\_ATTR\_POWER} values reported may actually be averaged over a short time window. Power caps are also enforced with respect to a target time window. \\
%%%\hline
%%%\texttt{PWR\_MD\_TS\_LATENCY}                  & Get     & \texttt{PWR\_Time}  & Estimate of the time required to get or set an attribute. This is useful to estimate completion time for an operation \textit{a priori}. A value of zero should be returned when the get/set is instantaneous.\\
%%%  \hline
%%%\texttt{PWR\_MD\_TS\_ACCURACY}             & Get     & \texttt{PWR\_Time}  & Estimated accuracy of returned timestamps, represented as +/- the \texttt{PWR\_Time} value returned. \\
%%%  \hline
%%%\texttt{PWR\_MD\_MAX\_LEN}                 & Get     & uint64_t   & The maximum string length that will be returned by the metadata interface. All other string lengths (metadata items ending in ``_LEN'') will be less than or equal to this value.  The value of \texttt{PWR\_MD\_MAX\_LEN} will be less than or equal to \texttt{PWR\_MAX\_STRING\_LEN}. \\
%%%\hline
%%%\texttt{PWR\_MD\_NAME\_LEN}                & Get     & uint64_t   & Length of the attribute name string, in bytes. This is the buffer length needed to store the string returned when \texttt{PWR\_MD\_NAME} is requested. \\
%%%  \hline
%%%\texttt{PWR\_MD\_NAME}                     & Get     & char *     & Attribute name string. This is a C-style NULL-terminated ASCII string. This provides a human readable name for the attribute. The string length is given by \texttt{PWR\_MD\_NAME\_LEN}. \\
%%%  \hline
%%%\texttt{PWR\_MD\_DESC\_LEN}                & Get     & uint64_t   & Length of the attribute description string, in bytes. This is the buffer length needed to store the string returned when \texttt{PWR\_MD\_DESC} is requested. \\
%%%  \hline
%%%\texttt{PWR\_MD\_DESC}                     & Get     & char *     & Attribute description string. This is a C-style NULL-terminated ASCII string. This provides a human readable description of the attribute that is more descriptive than the attribute's name alone. The string length is given by \texttt{PWR\_MD\_DESC\_LEN}. \\
%%%  \hline
%%%\texttt{PWR\_MD\_VALUE\_LEN}               & Get     & uint64_t   & Maximum length of the value strings returned by \texttt{PWR\_MetaValueAtIndex}. This can be used to discover the buffer size that needs to be passed to \texttt{PWR\_MetaValueAtIndex} via the \texttt{value\_str} argument. \\
%%%  \hline
%%%\texttt{PWR\_MD\_VENDOR\_INFO\_LEN}        & Get     & uint64_t   & Length of the vendor information string, in bytes. This is the buffer length needed to store the string returned when \texttt{PWR\_MD\_VENDOR\_INFO} is requested. \\
%%%  \hline
%%%\texttt{PWR\_MD\_VENDOR\_INFO}             & Get     & char *     & Vendor provided information string. This is a C-style NULL-terminated ASCII string. This may be used to convey part numbers, configuration, or other non-standard information. The string length is given by \texttt{PWR\_MD\_VENDOR\_INFO\_LEN}. \\
%%%  \hline
%%%\texttt{PWR\_MD\_MEASURE\_METHOD}          & Get     & uint64_t   & Denotes the measurement method: an actual measurement (returned value = 0) or a model based estimate (return value = 1). Other values $> 1$ may be used to denote multiple vendor specific models in the situation where multiple models may exist. \\
%%%  \hline
%%%\end{longtable}
%%%\end{center}
%%%

      
	\chapter{High-Level (Common) Functions}\label{chap:HighLevel}
	\input{HighLevel}

	\chapter{Role/System Interfaces}\label{chap:Interfaces}
	\input{Interfaces}

	\chapter{Conclusion}
	\input{Conclusion}

	% References
	\nocite{*}

	\clearpage
	\providecommand*{\phantomsection}{}
	\phantomsection
	\addcontentsline{toc}{chapter}{References}
	\bibliographystyle{plain}
	\bibliography{references}

	%\appendix
	\begin{appendices}

		\chapter{Topics Under Consideration for Future Versions}
		

The following topics are either currently in active discussion or are planned to be addressed in future versions of the specification.
In some cases it will be necessary to solicit additional feedback from the community to ensure we properly address the issue in future versions.

\begin{itemize}[noitemsep,nolistsep]

\item{\textbf{Support for Generic Notifications between Nodes} - Working on capability to send generic notifications between PowerAPI objects. This is expected to be useful for compatibility with other community projects and job management.}

\item{\textbf{Better Support for Frequency Scaling} - Frequency scaling features are not universal between hardware implementations. It is desirable to expose all of the possible behaviors of frequency scaling and we are working towards better descriptive solutions to represent these behaviors in the most accurate manner possible.}

\item{\textbf{Coexistence of Implementations} - One of the driving questions for this future work is - how does one implementation interface with another? It is possible, even likely that an implementor will focus on implementing a portion or portions of the specification. This begs the question of how does implementation A interact with implementation B? Further, what role does the specification play in driving this interaction? We intend to work closely with the community to sort out this issue and document the appropriate guidance in the next version of the specification.
}
\item{\textbf{Language Bindings} - Some roles, system administrator for example, more commonly interface with the platform through shells, shell scripting or other interpretive languages like Perl or Python. We will investigate adding some or all of these capabilities, via specification and possibly prototypes, in future versions of the standard.
}
    \begin{itemize}
        \item{The next version of the specification will include a complete Python specification of all existing functions modified appropriately for the Python language}
    \end{itemize}

\item{\textbf{User Guide} - The addition of a user guide could provide additional useful information to both users and implementors. The addition of a users guide will be considered and if realized will accompany subsequent releases of the specification.
}
\item{\textbf{Hypothetical System Example} - We are considering creating a hypothetical system example to use to discuss and clarify concepts and higher level use cases. This will likely be included in the User Guide.
}
\item{\textbf{Required versus Optional, or Quality of Implementation} - We plan to clarify and document more precisely what portions of the specification are required to be implemented, what portions are optional and the definition of a quality implementation. This topic is complicated by the fact that implementors are free to implement portions of the specification. 
}
    \begin{itemize}
        \item{Some progress has been made on this topic for version 1.1 but additional work is required.}
    \end{itemize}

\item{\textbf{Policies} - Security policies, priority of operations and privileges need to be further vetted and specified when appropriate. This topic has a large amount of intersection with the \textit{Coexistence of Implementations} topic and will be considered jointly.
}
\item{\textbf{Unit Tests} - Development of a unit test infrastructure is under consideration, possibly to be associated with our prototype which will be released open source at a later date. Unit tests might also be a way for the implementation community to assure interaction between implementations of portions of the specification that will be required to work together.
}
\item{\textbf{User Supplied Functions} - We intend to investigate adding the ability for a user to supply a function for the purposes of generating a statistic, for example.
}
\item{\textbf{Multiple Platform Support} - Currently the specification only considers operation on a single platform. There is nothing preventing supporting multiple platforms and exposing multiple platforms in a single context in future versions. This will be considered for the next release in conjunction with the Coexistence of Implementation issue.
}
\item{\textbf{Generation Counter} - We intend to consider the addition of a generation counter capability to be used in conjunction with counters that have the potential for roll over. The generation counter could be used to inform the user that this has taken place. This concept likely has additional utility which is what will be explored for future releases of the specification. Target: 1.X - Implementation should handle overflow internally
}
\item{\textbf{Time Conversion/Overflow} - Time conversion convenience functions are being considered to convert between \texttt{PWR_Time} values and POSIX-compatible time representations. Included in this will be methods of detecting overflow during time value arithmetic.
}
\item{\textbf{Context Refresh} - We are considering adding the ability to refresh a context int he case of a long lived context such as one that is used by a persistent daemon. Yet to be resolved is what happens to existing pointers, more specifically what happens when the user has a pointer to an object that no longer exists after the refresh, or if this can happen. 
}
\item{\textbf{Enhanced Support for ACPI 5.0} - Collaborative Processor Performance Control and Continuous Performance Control are currently not supported. Support will require new attributes and some function calls to allow for the flexible mechanisms provided in the ACPI 5.0 specification to allow expression of desired performance on a sliding, abstract unit-less scale. ACPI 5.0 also supports gathering statistics about the delivery of given performance values and the time spent in certain states, which we intend to address. We anticipate adding this support alongside the P-state and C-state functionality already in the Power API in a future version of the specification. 
}
\item{\textbf{User/Resource Manager Interfaces} - Work needs to be done in this area but is best accomplished in collaboration with resource manager, work load manager experts. We hope to include standard interfaces for the user to query this system in future versions of the specification. 
}
    \begin{itemize}
        \item{Work has begun to develop general report and information mining capabilities}
    \end{itemize}

\item{\textbf{HPCS Manager to Resource Manager Interface} - This interface clearly needs some work. Again it seems that this would benefit greatly from collaborative efforts.
}
    \begin{itemize}
        \item{Work has begun to develop general report and information mining capabilities}
    \end{itemize}

\end{itemize}


		\chapter{Change Log}
		The following list contains changes to the community Power API specification version 0.9

\begin{itemize}
  \item{First release of community Power API}
  \item{PWR_ObjGetSizeOfName - New function added that allows the user to get the size of buffer required to successfully call PWR_ObjGetName for a given object.}
  \item{Features to support Power Stack}
\end{itemize}





		\chapter{Alternative Programming Language Bindings: Python}
		\input{LanguageBindings}

	\end{appendices}

	\clearpage
	\chapter{Index}
	{\scriptsize
	\printindex}
	%\addcontentsline{toc}{chapter}{Index}


\end{document}
