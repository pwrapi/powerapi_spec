The Resource Manager/Operating System Interface is intended to access both low level and abstracted information from the operating system. 
Similar or additional information may be available from the monitor and control system (section \ref{sec:RMMC}) depending on the implementation.
The resource manager is in a somewhat unique position of providing a range of functionality depending on the specific implementation.
The resource manager role includes functionality such as batch schedulers and allocators as well as potential portions of tightly integrated runtime and launch systems.
The resource manager may require fairly low level measurement information to make decisions and potentially store historic information for consumption by the user role (for example). 
The resource manager may also play a very large role in controlling power and energy pertinent functionally on both a application and platform basis in response to facility restrictions (power capping or energy aware scheduling for example).

\subsection{Supported Attributes}\label{sec:RMOSAttributes}
A significant amount of functionality for this interface is exposed through the attribute functions (section \ref{sec:Attributes}).
The attribute functions in conjunction with the following attributes (Table \ref{table:RMOS}) expose numerous measurement  (get) and control (set) capabilities to the resource manager.

\begin{attributetable}{Resource Manager, Operating System - Supported Attributes}{table:RMOS}
	\aPstateDesc
	\aCstateDesc
	\aCstateLimitDesc
	\aSstateDesc
	\aPowerDesc
	\aMinPowerDesc
	\aMaxPowerDesc
	\aFreqDesc
	\aFreqLimitMinDesc 
	\aFreqLimitMaxDesc
	\aEnergyDesc
	\aTempDesc
\end{attributetable}

\subsection{Supported Core (Common) Functions}\label{sec:RMOSSupportedCommon}

\begin{itemize}[noitemsep,nolistsep] 
	\item{Hierarchy Navigation Functions - section \ref{sec:Navigation}}
		\begin{itemize}[noitemsep,nolistsep] 
			\item{ALL}
		\end{itemize}
	\item{Group Functions - section \ref{sec:Group}}
		\begin{itemize}[noitemsep,nolistsep] 
			\item{ALL}
		\end{itemize}
	\item{Attribute Functions - section \ref{sec:Attributes}}
		\begin{itemize}[noitemsep,nolistsep] 
			\item{ALL}
		\end{itemize}
	\item{Metadata Functions - section \ref{sec:METADATA}}
		\begin{itemize}[noitemsep,nolistsep] 
			\item{ALL}
		\end{itemize}
	\item{Statistics Functions - section \ref{sec:StatisticsFunctions}}
		\begin{itemize}[noitemsep,nolistsep] 
			\item{ALL}
		\end{itemize}
\end{itemize}

%==============================================================================%

\subsection{Supported High-Level (Common) Functions}\label{sec:RMOSHighLevel}

%==============================================================================%

\subsection{Interface Specific Functions}\label{sec:RMOSFunctions}

%=============================================================================%
%int    PWR_ObjGetName( PWR_Obj object, char* dest, size_t len );
\begin{prototype}{NotifyToNode}
	\longdescription{The \texttt{PWR_NotifyToNode} function provides data in a notification object to the OS/node level from the system (resource manager/scheduler) level. This function is useful for reporting non-standard operating conditions directly to the node, bypassing intermediate layers For example, in the case where a job is being managed by an intelligent job manager, this function can notify the node that something has gone wrong with the job in the event that the job manager is incapable of providing this information. This is important as the operating conditions that were provided at a job-level may be violated if an error occurs. Therefore this solution allows the system to notify nodes of what to do. This is useufl if there are node level power/energy optimization runtimes in operation.}
	\returntype{int}
	\parameter{PWR_Note note}	{\pInput}	{The notification that the system wishes to transmit to the OS/node layer.}
	\parameter{int agentid}    	{\pInput}	{The system agent target for the notification.}
	\parameter{size_t len}    	{\pInput}	{The length of the user provided notification.}
	\returnval{PWR_RET_FAILURE} 	{Upon FAILURE.}
\end{prototype}


