\section{Theory Of Operation}\label{sec:PythonTheoryOfOperation}
\subsection{Overview}\label{sec:PythonOverview}

This section loosely follows the same tack as the C API chapter
of the same name (chapter \ref{chap:Theory} on page \pageref{chap:Theory}). In
particular, it discusses any remarkable caveats or differences between the
Python Power API bindings and the C Power API specification. Because the
intended audience of this section includes those with limited experience with
Python, seasoned Python developers are forewarned of some pedantic explanations
of Python behavior.

%==============================================================================%
\subsubsection{Python Version Agnosticism}\label{sec:PythonVersionAgnosticism}

This Python specification aims to be agnostic with respect to Python versioning
where possible. It is currently designed to include version 2.7 and on. In
particular, considerations needed to include version 2.7 are made with respect
to integer typing (section \ref{sec:PythonNumericTypes} on page
\pageref{sec:PythonNumericTypes}) and enumerations (section
\ref{sec:Enumerations} on page \pageref{sec:Enumerations}). Enumerations as
explained later in this section follow a C-style syntax instead of the ``enum''
data type available as of Python
3.4.\footnote{https://docs.python.org/3/library/enum.html}

%==============================================================================%
\subsubsection{The Python import statement}\label{sec:PythonImport}

Throughout this appendix the ``\texttt{pwr.}'' prefix is used on API class methods,
functions, and variables so that they are identifiable in the document. The
example below illustrates a typical usage of an Python module implementation of
the API imported as \texttt{pwr}:

\begin{center}\begin{minipage}{.95\linewidth}\begin{lstlisting}
import <your.implementation.library.path> as pwr
myPwrCntxt = pwr.Cntxt(...)
\end{lstlisting}\end{minipage}\end{center}

%==============================================================================%
\subsubsection{Memory Management and Garbage Collection}\label{sec:PythonMemoryManagement}

The C API specification is very precise in defining the enumerations,
structures, and entry-points that are to be used to access the power
measurement and control functionality exposed by the specification. In the
pass-by-value C language, ``objects'' are precisely defined structures that are
allocated (and freed) by the user of the API or the implementation of the API.
The functions defined by the C API use opaque handles to identify or point to
these objects to represent
them and pass them around from place to place. Refer to the C API ``Theory of
Operation'' section \ref{chap:Theory} on page \pageref{chap:Theory} for the
language independent information.

Python is an object-oriented language, and uses ``pass-by-object-reference'' to
reference and transport data. Python treats everything as an object, including
references to other objects, however, it has no native concept of pointers to
locations in memory. Because of this, Python uses a garbage collection
mechanism to clean up previously instantiated objects which have become
completely de-referenced.

For example, the C API call \texttt{PWR_CntxtInit()} returns a handle
``myPwrCntxt'' that may identify or point to a location in memory containing
the context structure:

\begin{center}\begin{minipage}{.95\linewidth}\begin{lstlisting}
/* Example C code */
PWR_Cntxt myPwrCntxt;
rc = PWR_CntxtInit(PWR_CNTXT_DEFAULT, PWR_ROLE_RM, "foo", &myPwrCntxt);
\end{lstlisting}\end{minipage}\end{center}

The memory region pointed to by myPwrCntxt must be returned to the heap once
the application is done using it. This explicit memory management is some of
what necessitates the use of clean-up functions such as
\texttt{PWR_CntxtDestroy(myPwrCntxt)}.

With Python, ``myPwrCntxt'' is a reference to an instance of the
\texttt{pwr.Cntxt} object class. Python counts the number of references to its
objects, and once that count goes to ``0'', the object is garbage collected
(freed). De-referencing an object such as ``myPwrCntxt'' can be implicitly
performed by returning from the method or function in which it is being used,
or explicitly performed by calling ``del'':

\begin{center}\begin{minipage}{.95\linewidth}\begin{lstlisting}
import <your.implementation.library.path> as pwr
def someFunction():
    myPwrCntxt = pwr.Cntxt(pwr.CntxtType.DEFAULT, pwr.Role.RM, "foo")
    localEPObj = myPwrCntxt.GetEntryPoint()
    ...
    # Once this function returns, localEPObj is out of its scope, which
    # causes Python garbage collection to free its memory. myPwrCntxt is
    # not garbage collected as it is needed/returned to the caller.
    return myPwrCntxt

def anotherFunction():
    LocalPwrCntxt = someFunction()
    ...
    # Explicitly releasing the LocalPwrCntxt object returned from
    # someFunction() by calling del.
    del LocalPwrCntxt
    ...  # More to be done, but no further need for the LocalPwrCntxt object.
\end{lstlisting}\end{minipage}\end{center}

If myPwrCntxt is being used elsewhere, such as being passed via the return
statement in a particular method or function, it is not freed (garbage
collected) until all possible usage is out of scope. This automatic garbage
collection eliminates the necessity to explicitly free the memory used by
``myPwrCntxt''.

%==============================================================================%
\subsubsection{Encapsulation - Methods are part of the data classes}
\label{sec:PythonEncapsulation}

Another feature of Python is its object-oriented approach to defining data and
the methods that act on that data. Since much of the C API specification is
written with collections of functions using a few handles or pointers to
identify common data structures as parameters, Python's object-oriented
approach fits well with the existing C API. In Python, classes are defined, and
those classes contain methods appropriate for the particular class. For
instance in C, \texttt{PWR_CntxtInit()} returns an opaque handle,
``myPwrCntxt''. That handle must be passed into functions as a parameter for
that function to know what data to act on:

\begin{center}\begin{minipage}{.95\linewidth}\begin{lstlisting}
/* Example C code: */
PWR_Cntxt myPwrCntxt;
PWR_Obj   myPwrObj;

rc = PWR_CntxtInit(PWR_CNTXT_DEFAULT, PWR_ROLE_RM, "foo", &myPwrCntxt);
if (rc != PWR_RET_SUCCESS)
	exit(-rc);
rc = PWR_CntxtGetEntryPoint(myPwrCntxt, &myPwrObj);
if (rc != PWR_RET_SUCCESS)
	exit(-rc);
\end{lstlisting}\end{minipage}\end{center}

\emph{Note: The example above adds error handling that validates the return code of the two
C API functions called. This causes the C example to exit if one of the calls
fail, resulting in the same ``exit on error'' behavior that the Python example
that follows will exhibit. More on Python error handling in
\ref{sec:PythonErrorHandling} on page \pageref{sec:PythonErrorHandling}.}

In this Python API, \texttt{myPwrCntxt} is an object that is an instance of the
\texttt{pwr.Cntxt} class and that object contains methods for acting on that
object:

\begin{center}\begin{minipage}{.95\linewidth}\begin{lstlisting}
myPwrCntxt = pwr.Cntxt(pwr.CntxtType.DEFAULT, pwr.Role.RM, "foo")
myPwrObj = myPwrCntxt.GetEntryPoint()
\end{lstlisting}\end{minipage}\end{center}

Descriptions and many more examples of class object and methods can be found in
section \ref{sec:PythonCoreInterfaceFunctions} starting on page
\pageref{sec:PythonCoreInterfaceFunctions}.

%==============================================================================%
\subsubsection{Enumerations}\label{sec:Enumerations}

The C API relies heavily on C-style enumerations which, amongst other things,
enforce strict type-checking, and can provide automatic incrementing of the
enumeration's value. Python 2.7 has no direct support for enumerations.  This
Python API specification defines support for enumerations so that they match in
style and functionality with the defined enumerations in the C API. This
enables strict type-checking and automatic incrementing of the enumeration's
value definitions:

\begin{center}\begin{minipage}{.95\linewidth}\begin{lstlisting}
# Example Colors Enumeration (not part of the specification)
class Colors(_EnumerationClass):
    pass
Colors("Blue")
Colors("Green")
Colors("Red")
Colors("Yellow")
# Vendor implementations can add more entries here or look at
# the object-oriented "Extending Existing Enumerations" description below.
Colors("NUM_COLORS")
Colors("INVALID", -1)
Colors("NOT_SPECIFIED", -2)
\end{lstlisting}\end{minipage}\end{center}

\begin{center}\begin{minipage}{.95\linewidth}\begin{lstlisting}
# Example Sizes Enumeration (not part of the specification)
class Sizes(_EnumerationClass):
    pass
Sizes("Small")
Sizes("Medium")
Sizes("Large")
# Vendor implementations can add more entries here or look at
# the object-oriented "Extending Existing Enumerations" description below.
Sizes("NUM_SIZES")
Sizes("INVALID", -1)
Sizes("NOT_SPECIFIED", -2)
\end{lstlisting}\end{minipage}\end{center}

\emph{Implementation Note: Defining a class such as Colors with only the
do-nothing \texttt{pass} statement imposes strict type-checking against the
Colors class instead of the generic (and externally defined) \texttt{_Enumeration}
class. The subsequent ``constructors'' being called to define the specific
enumerations such as Colors(``Green'') actually add the definition into
the Colors class itself, as opposed to instantiating the Colors
class. This is implemented using Python ``MetaClasses''.}

%==============================================================================%
\paragraph{Enumerations Extension}\label{sec:EnumerationsExtension}

Enumerations can be extended in two ways: (1) By adding more entries before the
\texttt{NUM_classname} entry, as shown in the example above, or (2) by extending a
defined \texttt{EnumerationClass} in an object-oriented way, as illustrated here:

\begin{center}\begin{minipage}{.95\linewidth}\begin{lstlisting}
# Likely in some other vendor specific file...
Colors("Black", Colors.NUM_COLORS)
Colors("NUM_COLORS")  # to reset NUM_COLORS
#
Sizes("X-Large", Sizes.NUM_SIZES)
Sizes("NUM_SIZES")  # to reset NUM_SIZES
\end{lstlisting}\end{minipage}\end{center}

The code examples above show how a vendor implementation can extend a defined
\texttt{EnumerationClass} in this specification.

%==============================================================================%
\paragraph{Enumerations Usage}\label{sec:EnumerationsUsage}

The following examples show Python code using some of the enumerations defined
in section \ref{sec:PythonContextReleventTypeDefinitions} starting on page
\pageref{sec:PythonContextReleventTypeDefinitions}:

\begin{center}\begin{minipage}{.95\linewidth}\begin{lstlisting}
# Using enumerations as parameters
myPwrCntxt = pwr.Cntxt(pwr.CntxtType.DEFAULT, pwr.Role.RM, "foo")

# Return the value of an enumeration
enumVal = int(pwr.Role.RM)  # Gives a value of 4 to enumVal

# Return the name of the enumeration:
enumStr = str(pwr.Role.RM)  # Gives "RM" to enumStr

# Strict type-checking can be enforced
class Cntxt():
    def __init__(self, cntxtType, cntxtRole, cntxtName):
    ...
    if not isinstance(cntxtType, CntxtType):
            raise PwrError( ReturnCode.BAD_VALUE,
                "{0:s}: Invalid context type.".format(self.__class__.__name__))
    ...
\end{lstlisting}\end{minipage}\end{center}

%==============================================================================%
\subsubsection{Numeric Types}\label{sec:PythonNumericTypes}

Python represents integer, unlimited length integer, and floating point numbers
with its respective ``int'', ``long'' and ``float'' built-in types. A Python
``float'' is equivalent to a C double. Unlike Python 3.0, Python 2.7 still
distinguishes an ``int'' type from an unlimited-length ``long''. Python
implementations of this API shall use long for all integer types.

%==============================================================================%
\subsubsection{Time Entities}\label{sec:PythonTimeEntities}

The C API specification in section \ref{sec:TimeRelatedDefinitions} on page
\pageref{sec:TimeRelatedDefinitions} describes encapsulating a 64-bit integer
representing a time value in nanoseconds. Python represents a time value in its
\texttt{time} module using a floating point number representing time in seconds.
Since Python time values are a floating point values (e.g., $13.434582349$
seconds) no precision is lost representing time this way. Conversion back and
forth between the C style integer representation and Python's default floating
point format can be done as follows:

\begin{center}\begin{minipage}{.95\linewidth}\begin{lstlisting}
# Convert Python's floating point "seconds" value to a long
# integer representing "nanoseconds":
timeIntNs = long(timeFloatSec * 1000000000.0)

# Convert a long integer representing a count of "nanoseconds" to a Python
# "float" value containing a possible fractional value of "seconds":
timeFloatSec = float(timeIntNs) / 1000000000.0
\end{lstlisting}\end{minipage}\end{center}

The preferred Python implementation for handling time in this API is described
in section \ref{class:Time} on page \pageref{class:Time}.

%==============================================================================%
\subsubsection{Error Handling}\label{sec:PythonErrorHandling}

Error handling in the C API specification generally involves checking for a
non-zero integer return-code value. Python supports two methodologies for
robust error handling: (1) exceptions and (2) the ability of a method or
function to return multiple values (in the form of a tuple). Returning multiple
values enables the return of an error code alongside of an expected (or
unexpected) result. This Python API specification uses Python exceptions for
things that are generally seen as fatal errors. Some functions in the API that
``get'' or ``set'' values use per-value return codes to continue the operation
in the face of non-fatal errors.

Exceptions give the ability to funnel any errors in a Python method or function
through one or more error tracking regions wrapped by Python
\texttt{try/except} clauses. Errors that occur while instantiating objects,
i.e. when the class's \texttt{__new__()} and \texttt{__init__()} methods are
invoked, need to be handled using Python exceptions. This is primarily because
some of the Python library initialization methods raise exceptions themselves
and a class \texttt{__init__()} method does not return any value at all. For
these reasons, this Python specification uses the Exception methodology.

In this specification, some methods return arrays of measurement information
that may contain error information for individual measurements. These methods
may encounter an error reading an attribute on a particular object in the
group, but the rest of the attributes from the operation should not be thrown
away. For this type of functionality exceptions are only raised in the case of
a fatal error, and per-element access errors are handled with error status
information in results data structures.

For ``Get'' methods that return or yield multiple results from multiple
operations such as the \texttt{AttrGetValue(s)} methods and the
\texttt{pwr.Stat.GetValue(s)} methods, the \texttt{pwr.ReturnCode} value is
included with the measurement data for any particular operation. If the return
code value is set to \texttt{pwr.ReturnCode.SUCCESS}, then the measurement data
is valid. Otherwise the data is invalid, and the return code is set to be
something besides \texttt{pwr.ReturnCode.SUCCESS}. See
\ref{meth:ObjAttrGetValue} on page \pageref{meth:ObjAttrGetValue} for more
details.

Similarly, ``Set'' operations will yield failure details when errors occur.  If
there are no errors, nothing will be generated.  See \ref{meth:ObjAttrSetValue}
information returned.

%==============================================================================%
\paragraph{Class PwrError} \label{class:PwrError}

An example is shown below of the definition of the
\texttt{Exception} class that is used throughout this
Python API specification. See section \ref{class:ReturnCode} on page
\pageref{class:ReturnCode} for supported API ReturnCodes. All exceptions
generated by implementation of this specification should use and check for this
Exception class. Other system-level exceptions that may occur should be
accounted for but are not described in this document. Please refer to
system-level Python documentation for details on these other exceptions.

\begin{center}\begin{minipage}{.95\linewidth}\begin{lstlisting}
class PwrError(EnvironmentError):
    def __init__(self, returnCode, errorMsg = None):
        # returnCode      --> e.errno
        # str(returnCode) --> e.strerror
        # errorMsg        --> e.errmsg
        ...
\end{lstlisting}\end{minipage}\end{center}

Below is an example of the \texttt{pwr.PwrError} exception:

\begin{center}\begin{minipage}{.95\linewidth}\begin{lstlisting}
import <your.implementation.library.path> as pwr

def someOptionalMethod():
    # Is not implemented
    raise PwrError( ReturnCode.NOT_IMPLEMENTED,
        "{0:s}: someOptionalMethod not implemented!".format(
            self.__class__.__name__))

def ExampleOfExceptionHandling():
    ...
    defaultResult = None
    try:
        theResult = myPwrObj.someOptionalMethod()
        theResult += myPwrObj.SomeOtherMethod()

    except PwrError as e:
        print "ERROR!"
        print e.errno     # "-2"
        print e.strerror  # "NOT_IMPLEMENTED"
        print e.errmsg    # "someOptionalMethod not implemented!"
        return defaultResult
        # "raise" can also be called here.
    else:
        return theResult
\end{lstlisting}\end{minipage}\end{center}

%==============================================================================%
\paragraph{Multiple Return Values}\label{sec:PythonMultipleReturnValues}

Python seemingly has the capability to return multiple values from a method or
function. These multiple values are actually contained within a single object.
This follows from the fact that everything is an object in Python, including
tuples, which is an object that contains a collection of other objects:

\begin{center}\begin{minipage}{.95\linewidth}\begin{lstlisting}
def SomeMethodOrFunction():
    return 1, 2
    # or "return (1, 2)"
\end{lstlisting}\end{minipage}\end{center}

The method or function can be called two ways: either returning the elements of
the tuple or the entire tuple as one indexed object (an array). Here the
tuple's elements are returned:

\begin{center}\begin{minipage}{.95\linewidth}\begin{lstlisting}
retval1, retval2 = SomeMethodOrFunction()
# or "(retval1, retval2) = SomeMethodOrFunction(arg)"
print str(retval1)  # prints "1"
print str(retval2)  # prints "2"
\end{lstlisting}\end{minipage}\end{center}

Here everything is treated as the singular-indexed tuple object:

\begin{center}\begin{minipage}{.95\linewidth}\begin{lstlisting}
retval = SomeMethodOrFunction()
print str(retval[0])  # prints "1"
print str(retval[1])  # prints "2"
\end{lstlisting}\end{minipage}\end{center}

In the case where a method or function returns a tuple of several arrays that
relate one-to-one with each other element-wise, they can be re-arranged to be an
array of tuples.  This arrangement which may be more programmatically simple to iterate over,
among other things. Below is an example of how to convert a tuple of arrays to
an array of tuples using the standard built-in Python function, \texttt{zip}:

\begin{center}\begin{minipage}{.95\linewidth}\begin{lstlisting}
def SomeGroupMethodOrFunction(arg):
    return [1,2,3], [4,5,6]
array1, array2 = SomeGroupMethodOrFunction(arg)
print array1         # prints "[1,2,3]"
print array2         # prints "[4,5,6]"
combinedArray = zip(array1, array2)
print combinedArray  # prints [(1,4),(2,5),(3,6)]
\end{lstlisting}\end{minipage}\end{center}

%==============================================================================%
\subsubsection{Iterators and Generators}\label{sec:PythonIteratorsGenerators}

Python has a special \texttt{yield} statement which allows a method or function to
return or ``generate'' a result without exiting from the logic flow of that
method or function. This allows an ``Iterator'' method/function to loop and
collect or act upon the particular objects that a ``Generator'' method/function
may yield. This technique can be used to avoid the creation of very long,
memory intensive lists of objects such as those represented by a
\texttt{pwr.Grp} object.

In this Python specification, ``Generators'' are exposed as part of the API.
They complement the normal, list-giving methods as defined in the C API, but do
not replace them. They are prefixed with the name ``Generate''. For example,
the method \texttt{GenerateChildren()} complements the \texttt{GetChildren}
method documented in \ref{meth:GetChildren} on page
\pageref{meth:GetChildren}). The following are some example functions
illustrating the use of iterators and generators versus lists:

\begin{center}\begin{minipage}{.95\linewidth}\begin{lstlisting}
# This function creates a list of objects and returns it to the calling function.
# The entire list is created in memory before it is returned to the caller.
def ObjectLister(numObjs):
    objList = []
    for objNum in numObjs:
        newObj = ExampleObj()
        objList.append(newObj)
    return objList

# This function generates objects on the fly, and yields each object it creates
# to the calling function one at a time. The code flows back into this generator
# on each iteration of the calling functions for loop.
def ObjectGenerator(numObjs):
    for objNum in numObjs:
        newObj = ExampleObj()
        yield newObj

def ObjectConsumer():
    for someObj in ObjectGenerator(1000000):
        # ObjectGenerator has yielded another object for this function to use.
        useObjectOnce(someObj)  # Each individual "someObj" is garbage collected

    # Here ObjectLister() creates/returns the entire list all at once.
    for someObj in ObjectLister(1000000):
        # This function iterates over the list, and then when it is done with the
        # entire list, the list and its objects get garbage collected.
        useObjectOnce(someObj)

\end{lstlisting}\end{minipage}\end{center}

%==============================================================================%
\subsubsection{Shortcuts using Properties}\label{sec:PythonShortcutsProperties}
Python offers the ability to attach ``properties'' to class methods and to
overload operators so that simple ``set'' and ``get'' methods and operators
on a class can appear and behave like standard class-variables. These
shortcuts make for simpler and more concise code. Throughout this Appendix,
these shortcuts will be mentioned for the various functions, methods and
operators for which they are available.
\begin{center}\begin{minipage}{.95\linewidth}\begin{lstlisting}
# The standard way...
myEntryPoint = cntxt.GetEntryPoint()
myAttr = myPwrObj.AttrGetValue(pwr.AttrName.TEMP)
myAttrMeta = myPwrObj.AttrGetMeta(pwr.AttrName.TEMP, pwr.MetaName.SAMPLE_RATE)
unionGroup = myPwrGrp.Union(someOtherPwrGrp)

# With properties...
myEntryPoint = cntxt.entrypoint
myAttr = myPwrObj.TEMP.value
myAttrMeta = myPwrObj.TEMP.SAMPLE_RATE
unionGroup = myPwrGrp | someOtherPwrGrp
\end{lstlisting}\end{minipage}\end{center}

%==============================================================================%
\subsection{Power API Initialization}\label{sec:PythonPowerAPIInitialization}
Initialization is accomplished by instantiating the \texttt{pwr.Cntxt} class
which returns a \texttt{pwr.Cntxt} object:

\begin{center}\begin{minipage}{.95\linewidth}\begin{lstlisting}
import <your.implementation.library.path> as pwr
myPwrCntxt = pwr.Cntxt(CtxtType,  # pwr.CntxtType
                       Role,      # pwr.Role
                       Name)      # Python str
\end{lstlisting}\end{minipage}\end{center}

\subsection{Roles}\label{sec:PythonRoles}
All of the same roles in the C API specification (section \ref{sec:Roles} on
page \pageref{sec:Roles}) should be supported by valid Python implementations.

\subsection{System Description}\label{sec:PythonSystemDescription}

All of the object types in the C API ``System Description'' (section
\ref{sec:PowerAPIBaseSysDesc}  on page \pageref{sec:PowerAPIBaseSysDesc}) are
represented by a Python base class, \texttt{pwr.Obj}. The \texttt{pwr.Obj} base
class supports functionality such as: obtaining an object's parent, obtaining
its children, getting the object's type, and navigating the object tree. Power
object type specific functionality is represented in child classes of the base
\texttt{pwr.Obj} class.

The C API on page \pageref{sec:PowerAPIBaseSysDesc}, states that a variety
of object types are to be defined, but not necessarily used or supported. These
are: ``Facility'', ``Platform'', ``Cabinet'', ``Chassis'', ``Board'', ``Node'', ``Socket'',
``Power Plane'', ``Core'', ``Memory'', and ``NIC''. In the C API specification,
opaque handles, which can be pointers, are used to point to these various abstracted
objects.  However, in Python, separate child classes to the \texttt{pwr.Obj} class
are created to represent these various types of objects. These object types are
represented by respectively named child classes such as
\texttt{pwr.ObjPlatform}, \texttt{pwr.ObjCabinet}, and \texttt{pwr.ObjBoard}.

\subsection{Attributes}\label{sec:PythonAttributes}

Each Python \texttt{pwr.Obj} object has two sets of associated attributes,
``Global'' and ``Explicit''. Refer to the C API ``Attributes'' section
\ref{sec:TheoryAttributes} on page \pageref{sec:TheoryAttributes} for language
independent details. Global attributes are guaranteed to exist on every type of
\texttt{pwr.Obj}.

Global attributes are accessed through methods defined in the base
\texttt{pwr.Obj} class, such as \texttt{GetName}, \texttt{GetType},
\texttt{GetParent}, \texttt{GetChildren/GenerateChildren}, e.g.:

\begin{center}\begin{minipage}{.95\linewidth}\begin{lstlisting}
myType = myPwrObj.GetType()
myName = myPwrObj.GetName()
myParent = myPwrObj.GetParent()
myChildren = myPwrObj.GetChildren()
\end{lstlisting}\end{minipage}\end{center}

The \texttt{GetEntryPoint} context method is accessed via the
\texttt{pwr.Cntxt} class and returns the entry point object of the context's
system description:

\begin{center}\begin{minipage}{.95\linewidth}\begin{lstlisting}
import <your.implementation.library.path> as pwr
myPwrCntxt = pwr.Cntxt(pwr.CntxtType.DEFAULT, pwr.Role.MC, "MonitorAndControl")
myPwrObj = myPwrCntxt.GetEntryPoint()
\end{lstlisting}\end{minipage}\end{center}

Explicit attributes are attributes that may be unique to one or more
\texttt{pwr.Obj} object types. They are accessed via the attribute interface.
For details on Python attribute methods, see section
\ref{sec:PythonAttributeMethods} on page \pageref{sec:PythonAttributeMethods}:

\begin{center}\begin{minipage}{.95\linewidth}\begin{lstlisting}
attrName = pwr.AttrName.POWER
measurement1 = myPwrObj.AttrGetValue(attrName)
measurement2 = myPwrObj.AttrGetValue(pwr.AttrName.VOLTAGE)
\end{lstlisting}\end{minipage}\end{center}

The attribute interface is preferred over explicit methods so that additional
API methods are not necessary to expand functionality for a particular
object type.

\subsection{Metadata}\label{sec:PythonMetadata}
Metadata are supported as detailed in section \ref{sec:TheoryMetadata} on page
\pageref{sec:TheoryMetadata} of the C API.

To access metadata for a particular object attribute:

\begin{center}\begin{minipage}{.95\linewidth}\begin{lstlisting}
attrName  = pwr.AttrName.PSTATE
metaName  = pwr.MetaName.MIN
minPstate = myPwrObj.AttrGetMeta(attrName, metaName)
maxPstate = myPwrObj.AttrGetMeta(pwr.AttrName.PSTATE, pwr.MetaName.MAX)
\end{lstlisting}\end{minipage}\end{center}

\subsection{Thread Safety}\label{sec:PythonThreadSafety}

There is no difference in the way threading is to be handled versus what is described
in the C API.
Please refer to the C API specification (section \ref{sec:ThreadSafety} on
page \pageref{sec:ThreadSafety}) for a discussion on threading and
multiprocessing concerns.
